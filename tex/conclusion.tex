\section{Fazit}

    \subsection{Zusammenfassung}
    Die Entwicklung von Embedded-Systemen mit Mikrocontrollern gewinnt an Popularität. Ein Hauptgrund dafür ist der
    Trend des \glsit{glo:iot}. Durch Plattformen wie Arduino und Raspberry Pi wird diese Entwicklung
    herangetrieben und zugänglicher gemacht.

    Die am weitesten verbreitete Chip-Architektur kommt von \glsit{glo:arm}, welche die zum heutigen Zeitpunkt relevanten
    Prozessorfamilien Cortex-A, -R und -M entwickeln. Diese Architekturen werden an verschiedene Hardware-Hersteller
    verkauft, wie zum Beispiel \textit{Texas Instruments}, \textit{Atmel} oder \textit{STMicroelectronics}.

    Zwischen der Hardware und einer Hardware-Applikation liegen mehrere Abstraktionsschichten.

    Dabei nutzt der Entwickler bzw. die Entwicklungswerkzeuge die Schichten mit den Technologien \acrshortit{glo:jtag}/\acrshortit{glo:swd} und \glsit{glo:cmsisdap}
    (falls unterstützt).

    Eine auf \glsit{glo:arm}-Cortex-basierende Applikation nutzt in der Regel ein Echtzeitbetriebssystem (oft bei Applikationen
    mitintegriert), vom Hersteller zur Verfügung gestellte Treiber und einen \glsit{glo:hal}.
    Diese nutzen die von \glsit{glo:arm} entwickelte unterste Abstraktionsschicht \glsit{glo:cmsis}-Core.

    Der Entwicklungsprozess wird durch folgende Punkte gestört:
    \begin{itemize}
        \item Herstellerspezifische Hardware-Abstraktionen (Applikation schlechter portierbar)
        \item Plattformbindende Entwicklungswerkzeuge
        \item Teure Lizenzen
        \item Herstellerspezifische Entwicklungswerkzeuge
        \item Steile Lernkurve für Neueinsteiger (größtenteils durch die vorherigen Punkte veranlasst)
    \end{itemize}

    Die Analyse eines Programms auf Fehler und Effizienz spielt eine große Rolle im Entwicklungsprozess von
    Applikationen. Dafür gibt es mehrere Ansätze, welche sich in dem Bereich der Embedded-Systeme etabliert haben,
    jedoch meistens die Anschaffung von externer Hardware voraussetzen.

    Der \acrshortit{glo:jtag}-Standard stellt eine einheitliche Schnittstelle zur Logik des Prozessors (oder jedem
    \glsit{glo:ic}, der \acrshortit{glo:jtag} implementiert) dar, mit dem Debugging-Anweisungen, auch zur Laufzeit, an den
    Prozessor zur Verarbeitung gesendet werden können.

    Eine Alternative zur Fehleranalyse mit externer Hardware stellt \glsit{glo:cmsisdap} dar, welches auf einer Debug-Einheit in Form
    eines \glsit{glo:ic} auf der Zielhardware verbaut ist und in den meisten Fällen einen \glsit{glo:usb}-Anschluss besitzt. Somit können
    Debugging-Anweisungen mit Hilfe des \acrshortit{glo:jtag}-Standards vom Entwicklungssystem direkt an die Zielhardware übertragen
    werden.

    Zum Herunterladen des entwickelten Firmware-Programms und anschließend zur Fehlersuche auf der Zielhardware kann
    neben externer Debugging-Hardware auch das Programm \glsit{glo:openocd} verwendet werden, welches durch den \acrshortit{glo:jtag}-Standard ein
    Protokoll und die benötigte Hardware für \textit{On-Chip Debugging} zur Verfügung gestellt bekommt.

    Software-seitige Ansätze, so wie \glsit{glo:openocd}, nutzen diese Technologie, um den Debugger auf dem Entwicklungssystem mit
    der Zielhardware kommunizieren zu lassen. Diese Arbeit thematisiert unter anderem auch die Integration einer
    solchen Software in eine \glsit{glo:ide}.

    Um den genannten Problemen und Hindernissen des Entwicklungsprozesses entgegenzuwirken, wird ein Konzept für eine
    Entwicklungsplattform mit folgenden Komponenten aufgestellt:
    \begin{itemize}
        \item Einfach zu nutzende und herstellerunabhängige Abstraktionsschicht für Funktionalitäten der Hardware, um
        Applikations-Code portabler zu machen und bei der trotzdem die maximale Leistung der verwendeten Hardware
        genutzt werden kann
        \item Designer-\glsit{glo:ui}, welches einem Entwickler eine einfache Alternative bietet Applikationen für
         beliebige Hardware zu entwickeln
        \item Erweiterbare Hardware, welche aber nicht zwangsläufig verwendet werden muss
    \end{itemize}

    \subsection{Ausblick}
    \label{sec:prospect}
    Der nächste Schritt, um den Entwicklungsprozess von Embedded-Systemen zugänglicher zu machen, ist aus dem Konzept
    für die Hardware einen Prototypen zu erstellen.

    An dem Hardware-Prototypen kann anschließend eine Schnittstelle für den herstellerunabhängigen
    \glsit{glo:hal} entwickelt werden. Die Implementierung der Interfaces erfolgt zunächst für die Hardware
    des Prototypen, anschließend soll die Implementierung von so vielen verschiedenen Hardware-Ausführungen wie möglich
    erfolgen.

    Mit der Definierung der Schnittstelle kann das \textit{Designer-}\glsit{glo:ui} entwickelt werden. Unter den in
    Abschnitt \ref{sec:modelbaseddev} aufgeführtem Paradigma zur modellbasierten Entwicklung wird eine graphische
    Entwicklungs- und Konfigurierungsmöglichkeit erstellt.

    Im Hinblick auf das \textit{Designer-}\glsit{glo:ui} soll ein abstrakter Weg gefunden werden, die Kommunikation
    unterschiedlicher Services (Funktionalitäten von Peripherie) mit verschiedenen Protokollen zu regeln.
    In anderen Worten: Wie lassen sich in einem Datenfluss zwischen zwei abstrahierten Hardware-Instanzen beliebige
    Protokolle aneinanderketten, um die Kommunikation zwischen diesen zu ermöglichen?\\
    \textbf{Beispiel:} Ein Sensor übermittelt gemessene Daten an einen Ethernet-Adapter. Zwischen den beiden
    Peripheriegeräten wären die in Abbildung \ref{fig:protocolabstraction} dargestellten Protokollschichten denkbar.
    \begin{figure}
        \centering
        \caption{Protokollschichten}
        \label{fig:protocolabstraction}
        \source{eigene Zeichnung}
        \includegraphics[scale=0.6]{../Graphiken/protocol_konsch}
    \end{figure}
    \texttt{MyProtocol} bezeichnet dabei ein vom Entwickler definiertes Format zur Datenübertragung.

    Im Verlauf des Prozesses ist nach neuen Technologien für potenzielle Alternativen zu den bis jetzt eingesetzten
    Technologien Ausschau zu halten. Eine Vielzahl an verschiedenen Hardware-Herstellern bedeutet gleichzeitig auch
    einen Wettlauf um die Entwicklung neuer Technologien zur Vereinfachung der Hardware-Entwicklung.
    Beispielsweise stellt der Hersteller NXP eine Flash-\glsit{glo:api} zur Verfügung, um ein Firmware-Programm vom
    Dateiverwaltungssystem eines Host-Systems per \glsit{glo:usb} direkt auf die Zielhardware herunterzuladen (``flashen''). Unter
    Windows kann ein Entwickler beispielsweise die kompilierte Datei per \textit{Drag-and-Drop} auf den Flash-Speicher
    der Zielhardware spielen.

    Auch Veränderungen im Bereich der Entwicklungsumgebungen sind zu beachten. In regelmäßigen Abständen werden neue
    Funktionalitäten hinzugefügt, von denen auch ein Entwickler für Embedded-Systeme profitieren könnte.

    Der \glsit{glo:openocd} ist durch eine ähnliche aber selbst entwickelte Software zu ersetzen. Diese soll eine deutlich
    reduzierte Version darstellen und auf die Verwendung als \glsit{glo:ide}-Plugin optimiert werden.

    Die zu entwickelnden Komponenten verfolgen das Ziel, den Entwickler in allen Aspekten des Entwicklungsprozesses für
    Embedded-Software zu unterstützen und ergeben zusammen eine neue Plattform für die Entwicklung von Embedded-Systemen.