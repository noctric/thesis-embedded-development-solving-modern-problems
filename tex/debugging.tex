\section{Debugging}
\label{sec:debugging}
In diesem Kapitel sollen zunächst die Grundprinzipien der Fehlersuche in Firmware-Programmen erläutert werden. Im
Vordergrund stehen hier die in der \glsit{glo:arm}-Cortex-M-Familie unterstützten Debugging-Features, das durch
\acrshortit{glo:jtag} abgelöste \glsit{glo:ice}, sowie ein Einblick in die Debugging-Technologien auf Host-Seite. Es soll
weiterhin auf den für die Fehlersuche in Programmen für Embedded-Systeme relevanten Unterschiede zwischen Soft- und
Hardware Breakpoints eingegangen werden.
Anschließend folgt eine Ausführung der Funktionsweise von vorhandenen Host-seitigen Debugging-Lösungen.
	\subsection{Debugging-Features}
	\label{sec:debugfeatures}
	Debugging ist für die meisten Software-Entwickler ein fester Bestandteil des Entwicklungsprozesses. Jede von
	Menschen geschriebene Software enthält vorhersehbare und nicht vorhersehbare Fehlerfälle, welche idealerweise
	im Vorhinein gefunden und korrigiert werden können bzw. müssen. Debugger sind ein mächtiges Werkzeug, das dem
	Entwickler bei der Fehlersuche behilflich ist.
	Verschiedene Debugger stellen verschiedene Funktionalitäten und Methoden zur Fehlerfindung bereit. Die klassischen
    Debugging-Methoden, welche von \glsit{glo:arm}-Cortex-M-Prozessoren unterstützt werden, sind im Folgenden aufgelistet
    \begin{description}
        \item[Breakpoint] Breakpoints sind prinzipiell Punkte innerhalb eines Programms, an denen die Ausführung
        des Programms angehalten wird. Dabei lassen sich wichtige Unterscheidungen treffen:
        \begin{description}
            \item[Software Breakpoints] Die für den Software\hyp{}Entwickler bekannte Art von Breakpoints sind
            Software\hyp{}Breakpoints. Diese können bei einer gewünschten Anweisung im Programm\hyp{}Code platziert werden.
            Der Debugger ersetzt die Anweisung (oder je nach Architektur nur einen Teil der Anweisung) mit einer
            speziellen Breakpoint-Anweisung. Bei der Cortex-M-Prozessorfamilie werden Software Breakpoints mit der
            Anweisung \texttt{BKPT} gesetzt. Die Anzahl der Software Breakpoints ist in der Theorie unbegrenzt (logische
            Begrenzung durch den Speicher). Wird die Anweisung \texttt{BKPT} ausgeführt, generiert der Prozessor
            ein Debug-Event, auf welches den Umständen entsprechend reagiert werden kann
            \citep[vgl. Kap. 14.5]{Yiu2013}.
            Üblicherweise hält der Prozessor die Ausführung des Programms an und der Entwickler kann mit Hilfe des
            Debuggers den Speicherinhalt auslesen und ggf. verändern.
            \item[Hardware Breakpoints] Hardware\hyp{}Breakpoints sind für viele Software\hyp{}Entwickler zunächst unintuitiv.
            Ein Hardware Breakpoint enthält eine Zuordnungslogik für Adressen. Sie beobachten einen internen
            Bus oder Befehlszähler und generieren bei einer Übereinstimmung ein Debug-Event\\\citep[]{Styger}.

            Bei der Cortex-M-Prozessorfamilie steuert die \glsit{glo:fpb} die
            Zuordnungslogik, die selbst bei temporär unveränderbarem Speicher (z.B. Flash-Speicher) Debug-Events
            generieren kann\citep[vgl.~Kap.~14.5]{Yiu2013}. Hardware Breakpoints sind allerdings limitiert. Bei
            Cortex-M-Prozessoren können bis zu acht Breakpoints gesetzt werden, von denen sechs für Befehlsadressen sind.
        \end{description}
        \item[Single Stepping]
        \textit{Single Stepping} bezeichnet die Ausführung eines einzelnen Schrittes des Programms. Ein \textit{Schritt}
        bedeutet in diesem Fall die Ausführung einer Code-Zeile oder eines Maschinenbefehls
        \footnote{\href{https://sourceware.org/gdb/onlinedocs/gdb/Continuing-and-Stepping.html}
        {GDB Dokumentation}, Zugriff: 19.07.2017}.
        Die Ausführung eines Programms kann durch ein \textit{Continue} auch normal weiter geführt werden.
        Üblicherweise ist in Debugging-Software auch ein \textit{Step-Over}-Befehl verfügbar, der es ermöglicht
        verschachtelte Aufrufe als einen einzigen Schritt auszuführen.
        \textit{Single Stepping} wird bei vielen Architekturen durch temporäre Software Breakpoints realisiert.
        \item[Watchpoint] \textit{Watchpoints} spielen eine wichtige Rolle in der Fehlerfindung von
        \\Hardware-Applikationen. Sie ermöglichen das Anhalten des Programms, sobald sich der Wert eines Ausdrucks ändert.
        Als Ausdrücke werden jegliche semantische Konstrukte definiert, die einen Wert liefern.
        Watchpoints werden auch \textit{Data Breakpoints} genannt.

        Prozessoren der Cortex\hyp{}M\hyp{}Familie enthalten die \glsit{glo:dwt}, welche für die
        Verwaltung von Watchpoints zuständig ist. ``Es können Breakpoints funktional identisch zum Verhalten der
        \texttt{bkpt} Anweisung [...] ausgelöst werden, wenn eine definierbare Operation [...] auf einem definierbaren Speicherbereich auf dem Adressbus erkannt
        wird''\citep[vgl. Kap. 10.1.3]{Asche2017}.

        Außerdem können bestimmte Restriktionen auf den Watchpoint gesetzt werden, so dass dieser nur bei bestimmten
        Werten oder anderen definierten Umständen ein Debug-Ereignis auslöst.
        \item[On the fly memory access] \acrshortit{glo:jtag} ermöglicht es dem Debugger während der Ausführung eines Programms
        Speicherinhalte zu verändern ohne das Programm zu stopppen.
    \end{description}
	\subsection{GDB}
	\label{sec:gdb}
	Eine der gängigsten Debugging\hyp{}Softwares ist der frei verfügbare \glsit{glo:gdb}. Die Debugging\hyp{}Software
	wird für die Fehlersuche in Programmen vieler Programmiersprachen eingesetzt und wird von vielen
	Entwicklungsumgebungen als Debugger-Logik verwendet. Dabei wird dem Nutzer oft eine graphische Bedienung durch eine
	\glsit{glo:ide} zur Verfügung gestellt.\\

	Der \glsit{glo:gdb} unterstützt mit den folgenden vier Grundfunktionalitäten den Anwendungsentwickler bei der Fehlersuche in
	seinem Programm
	\footnote{\href{https://sourceware.org/gdb/}{GDB Produkt Seite}}:
	\begin{itemize}
	    \item Starten des Programms unter verschiedenen Umständen und Konfigurationen
	    \item Ausführung des Programms bei definierten Bedingungen stoppen
	    \item Zustand des Programms bei gestoppten Zustand analysieren
	    \item Speicherinhalte verändern, um die Ausführung des Programms zu beeinflussen
	\end{itemize}

	Außerdem definiert \glsit{glo:gdb} ein \glsit{glo:rsp}, welches die Fehlersuche auf einem externen Gerät
	ermöglicht. Dabei sendet eine auf dem Host-System laufende Software (zum Beispiel eine \glsit{glo:ide}) die vom Nutzer
	eingegebenen Befehle über einen Kommunikationsweg (zum Beispiel \glsit{glo:tcp}) an ein Target-System. Dort werden die
	Befehle von einer Debugging-Software empfangen und ausgeführt.

	\subsubsection{GDB-Remote-Server-Protocol-Beispiel}
	\textbf{Beispiel:} Der Nutzer setzt einen Breakpoint im Programm-Code. Übermittelt wird die Information über das
	definierte Protokoll als Nachricht in der Form:\\
	\begin{Verbatim}[frame=single]
Z<TYPE><ADDR><LEN>
    \end{Verbatim}

    \texttt{Z} entspricht der in dem \glsit{glo:rsp} definierten Anweisung zum Setzen von Breakpoints.
    Zum Löschen eines Breakpoints lautet der Anfang der Nachricht \texttt{z}. In das Feld \texttt{TYPE} wird ein Wert
    $0$ oder $1$ geschrieben, wobei $0$ einen Software Breakpoint und $1$ einen Hardware Breakpoint setzt. Das darauf folgende
    Feld \texttt{ADDR} gibt die Adresse an der der Breakpoint gesetzt werden soll an. Schließlich wird in das Feld
    \texttt{LEN} bei Software Breakpoints die Länge der zu ersetzenden Anweisung (also die Anweisung hinter dem
    Breakpoint) und bei Hardware Breakpoints die zu betrachtende Speicherstelle geschrieben
    \footnote{\href{http://davis.lbl.gov/Manuals/GDB/gdb_31.html}{Dokumentation des GDB RSP}}.

    \begin{Verbatim}[frame=single]
Z011504
    \end{Verbatim}

    Das oben aufgeführte Beispiel zeigt den Inhalt einer Nachricht, in der ein Software\hyp{}Breakpoint (\texttt{Z0}) an der
    Adresse 0x1150 (\texttt{1150}) mit der Länge der zu ersetzenden Anweisung (\texttt{4}) gesetzt wird.

    Der \glsit{glo:gdb}-Server muss auf die Anweisung entweder mit einem \texttt{Enn}, wobei \texttt{nn} für eine Error-Nummer
    steht, oder einem \texttt{Ok} antworten.

    \subsection{In-Circuit-Emulation}
    \label{sec:ice}
        \glsit{glo:ice} ermöglicht einem Entwickler Firmware durch die Emulation von Hardware auf Fehler zu untersuchen. Dabei
        handelt es sich um \textit{Non Intrusive Debugging}. Als  nicht intrusiv werden alle Debugging-Techniken
        bezeichnet, die weder die Größe des Firmware-Codes noch die Schnelligkeit der Ausführung dieses Codes beeinflussen.\\
        \textbf{Beispiel:} Das klassische \texttt{printf()}-Debugging beeinflusst beide genannten Aspekte und ändert
        durch die zusätzlichen Anweisungen außerdem den Programmverlauf. Somit ist dieses Verfahren intrusiv/störend.
        \\

        Da Speicher in einem Embedded\hyp{}System oft ein kritischer Punkt ist, kann ein durch Debug\hyp{}Anweisungen vergrößerter
        Speicherbedarf ein Problem darstellen.
        Ein weiterer kritischer Punkt sind die häufig an ein Embedded-System gestellten Echtzeitanforderungen, die
        durch eine erhöhte Laufzeit bedroht werden.

        Ein In-Circuit-Emulator ist eine auf Debugging spezialisierte Hardware, welche den Zielprozessor ersetzt.
        Der Emulator wird dann an den Host-PC angeschlossen und kann über Debugging-Software benutzt werden. Dies
        ermöglicht einen ungestörten Einblick in den Programmablauf
        \citep[vgl.~Kap.~1.1]{Rath2005}.

        Moderne \glsplit{glo:ice} nutzen keinen eigenen verbauten Prozessor um die eigentliche Hardware zu emulieren, sondern greifen
        direkt auf den Zielprozessor mit Hilfe des \acrshortit{glo:jtag}-Standards zu. Hierbei spricht man auch von
        \textit{On Chip Debugging}.

	\subsection{Tracing}
	\label{sec:trace}
	Wie bereits in Abschnitt \ref{sec:requirements} angesprochen, bezeichnet Tracing eine passive Analyse des Programms.
	Mit Hilfe von Werkzeugen können ausgegebene Tracing\hyp{}Informationen, ``also Programmzähler und sowie Inhalte der durch
	sie manipulierten Speicherstellen''\citep[vgl.~Kap.~10.1.6]{Asche2017}, analysiert werden. Hierbei werden lediglich
	Sprungbefehle, also nicht sequentiell aufeinanderfolgende Befehle, aufgeführt.

	Tracing birgt einige Vorteile gegenüber anderen Debugging-Techniken.

	\begin{itemize}
	    \item Es hat keinen relevanten Einfluss auf die tatsächliche Laufzeit eines Programms.
	    \item Inkonsistentes Verhalten eines Programms kann nach auftreten direkt analysiert werden. Ein aufgetretener
	    Fehler, welcher sich nicht einfach reproduzieren lässt, kann durch eine post-portem-Analyse gefunden werden.
	    \item Zeitinformationen (wie lange die Ausführung eines Programms in einem bestimmten Programmabschnitt verweilt)
	    werden in Tracing-Informationen mitgeliefert. Somit kann eine umfassendere Leistungsanalyse durchgeführt werden.
	\end{itemize}

	Der große Nachteil bei Tracing liegt bei der Hardware-Anforderung. Tracing-Hardware benötigt zusätzlichen Platz auf dem
	\glsit{glo:soc}. Es bewirkt außerdem einen höheren Stromverbrauch\citep[vgl.~S.~3]{Campbell2014}.
	``bis zu sieben zusätzliche Pins müssen dafür beschaltet werden. Diese Pins sind normalerweise mit anderen
	Funktionalitäten gemultiplext, woraus folgt, dass diese anderen Funktionalitäten während des Tracens nicht zur
	Verfügung stehen''\citep[vgl.~Kap.~10.1.6]{Asche2017}.

	Tracing wird oft in Kombination mit einer externen Hardware verwendet, die entweder speziell auf Tracing ausgelegt
	ist oder einer normalen \acrshortit{glo:jtag}-Debug-Probe.

	\subsection{OpenOCD/PyOCD}
	\label{sec:openOCD}
	\glsit{glo:openocd} \footnote{\href{http://openocd.org/}{OpenOCD Webseite \texttt{openocd.org}}}
	und PyOCD \footnote{\href{https://github.com/mbedmicro/pyOCD}{PyOCD auf github \texttt{github.com/mbedmicro/pyOCD}}}
	sind zwei Software-Lösungen zum Debuggen von Firmware\hyp{}Programmen auf \glsit{glo:arm}-Cortex-Prozessoren.
	Das primäre Ziel beider Programme ist es, mit Hilfe einer
	verbauten Debug-Einheit und dem \glsit{glo:cmsisdap}-Protokoll, ein Firmware-Programm auf den Speicher spielen zu können und
	entsprechend auf Fehler zu untersuchen. Da in beiden Fällen das \glsit{glo:rsp} verwendet wird, können auf Host-Seite
	\glsit{glo:gdb}-Clients benutzt werden um das Programm auf Fehler zu untersuchen.

	\begin{figure}
        \centering
        \caption{OpenOCD-Module}
        \label{fig:openocdmodules}
        \source{\citep[S.~40~Kap.~6.1]{Rath2005}}
        \includegraphics[scale=0.25]{../Graphiken/openocd_modules}
    \end{figure}

	Abbildung \ref{fig:openocdmodules} zeigt eine abstrakte modulare Aufteilung des \glsit{glo:openocd}-Programms. Es geht daraus
	hervor, dass das Kernstück des Programms der \texttt{Daemon} (Hintergrunddienst) ist. Dieser dient als Vermittler zwischen Host- und
	Target-Seite. Die Module \texttt{Target} und \texttt{Flash} sind Abstraktionen, welche für die Kommunikation bzw.
	Datenübertragung genutzt werden und prozessorspezifische Informationen und Konfigurationen enthalten.

	Auf Hostseite ist ein \texttt{\glsit{glo:gdb}}-Modul vorhanden, welches vom \glsit{glo:cli}
	zum Aufruf von Debugging-Befehlen verwendet werden kann, aber ebenso auch ein direktes Interface zu einem
	\glsit{glo:gdb}-Frontend, also zum Beispiel eine \glsit{glo:ide}, bereitstellt.
	Über das \glsit{glo:cli} kann das Firmware-Programm durch einen
	von den \texttt{Flash}- oder \texttt{Target}-Modulen zur Verfügung gestellten Befehle auf den Prozessor gespielt werden.

	Auf Target-Seite (abstrahiert im Programm) ist ein \texttt{JTAG}-Modul vorhanden, welches für die Umsetzung der
	\acrshortit{glo:jtag}-Befehle zuständig ist. Alle vom Daemon empfangenen Befehle werden in das \acrshortit{glo:jtag}-spezifische Protokoll übertragen
	und an die Ziel-Hardware gesendet. Das \texttt{JTAG}-Modul enthält außerdem die Umsetzung des
	\glsit{glo:cmsisdap}-Protokolls für die Übertragung über \glsit{glo:usb}.

	Grundsätzlich ist eine Entwicklungsumgebung mit \glsit{glo:openocd} folgendermaßen aufgestellt:

	\begin{itemize}
	    \item \glsit{glo:openocd} Daemon wird gestartet
	    \begin{itemize}
	        \item ein lokaler \glsit{glo:gdb}-Server wird gestartet
	        \item ein lokaler \glsit{glo:telnet}-Server wird gestartet
	        \item ein lokaler \glsit{glo:tcl}-Server wird gestartet
	    \end{itemize}
	    \item Der Entwickler kann sein kompiliertes Firmware-Programm über das \glsit{glo:cli} auf den Chip herunterladen
	    \item Der Entwickler kann Breakpoints oder andere Debugging-Anweisungen über einen \glsit{glo:gdb}-Client
	    (z.B. \glsit{glo:ide} oder \glsit{glo:cli}), welcher mit dem \glsit{glo:gdb}-Server kommunizieren muss, an den Prozessor senden
	    \item Der Daemon empfängt alle Befehle und ``verpackt'' sie in die entsprechenden Protokolle
	    (\acrshortit{glo:jtag}/\acrshortit{glo:swd} und anschließend \glsit{glo:cmsisdap})
	\end{itemize}

	\glsit{glo:openocd} ist neben PyOCD eine der wenigen frei verfügbaren Softwares für On-Chip-Debugging und ist
	deshalb sehr weit verbreitet.