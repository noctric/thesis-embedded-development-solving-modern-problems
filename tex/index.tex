\documentclass[12pt,a4paper, twoside]{article}

% language
\usepackage[utf8]{inputenc}
\usepackage[ngerman]{babel}
\usepackage[T1]{fontenc}
\usepackage{lmodern}

% import this before anything else
\usepackage{hyperref}


% symbols
\usepackage{amsmath}
\usepackage{amsfonts}
\usepackage{amssymb}

% misc
\usepackage[acronym, xindy, toc]{glossaries}
\usepackage[square]{natbib}
\usepackage{makeidx}
\usepackage{fancyhdr}
\usepackage{listings}
\usepackage{xcolor}
\usepackage{textcomp}
\usepackage{todo}
\usepackage{dirtree}

\usepackage{caption}

% cuz we love Rust
\lstdefinelanguage{Rust}{
  keywords={let, mut, move, in, use, match, fn, true, false, catch, function, return, null, catch, switch, var, if, in, while, do, else, case, break},
  keywordstyle=\color{blue}\bfseries,
  ndkeywords={struct, enum, export, boolean, throw, import, this, self},
  ndkeywordstyle=\color{orange}\bfseries,
  identifierstyle=\color{black},
  sensitive=false,
  comment=[l]{//},
  morecomment=[s]{/*}{*/},
  commentstyle=\color{purple}\ttfamily,
  stringstyle=\color{red}\ttfamily,
  morestring=[b]',
  morestring=[b]"
}

\lstdefinelanguage{CMake}{
  keywords={include, set, find\_program, string},
  keywordstyle=\color{blue}\bfseries,
  ndkeywords={struct, enum, export, boolean, throw, import, this, self},
  ndkeywordstyle=\color{darkgray}\bfseries,
  identifierstyle=\color{black},
  sensitive=false,
  comment=[l]{\#},
  morecomment=[s]{/*}{*/},
  commentstyle=\color{purple}\ttfamily,
  stringstyle=\color{red}\ttfamily,
  morestring=[b]',
  morestring=[b]"
}

% include, set, find\_program, string

\fancyhead[L]{\rightmark}
\fancyhead[R]{}


% layout
\usepackage{graphicx}
\usepackage{float}
\usepackage{fancyvrb}
\usepackage{geometry}
\geometry{
	a4paper,left=35mm,right=25mm, top=3.5cm, bottom=4cm
	}
\usepackage{tabularx}
\usepackage{hyphenat}
\usepackage{float}

\newcolumntype{L}[1]{>{\raggedright\arraybackslash}p{#1}} % linksbündig mit Breitenangabe
\newcolumntype{C}[1]{>{\centering\arraybackslash}p{#1}} % zentriert mit Breitenangabe
\newcolumntype{R}[1]{>{\raggedleft\arraybackslash}p{#1}} % rechtsbündig mit Breitenangabe

% silbentrennung bei Wörtern mit Umlauten
\hyphenation{voll-stän-dig Ad-res-sen öf-fent-lich ?ber-set-zung Um-stän-de Au-then-ti-fi-zie-rungs-mög-lich-keit Mi-kro-con-trol-ler
Em-bed-ded De-bug-ging Ab-schnitt}

% caption cmd
\newcommand{\source}[1]{\caption*{Quelle: {#1}} }

% glossary in italic
\newcommand{\glsit}[1]{\textit{\gls{#1}}}
\newcommand{\glsplit}[1]{\textit{\glspl{#1}}}
\newcommand{\acrshortit}[1]{\textit{\acrshort{#1}}}
\newcommand{\acrfullit}[1]{\textit{\acrfull{#1}}}

% blank pages
\usepackage{afterpage}

\newcommand\blankpage{%
    \null
    \thispagestyle{empty}%
    \addtocounter{page}{-1}%
    \newpage}


% listings conf
\lstset{
  language=C,
  showstringspaces=false,
  basicstyle=\footnotesize\ttfamily,
  keywordstyle=\bfseries\color{green!40!black},
  commentstyle=\itshape\color{purple!40!black},
  identifierstyle=\color{black},
  stringstyle=\color{orange},
  frame=single,
  numbers=left,                    % where to put the line-numbers; possible values are (none, left, right)
  numbersep=5pt,                   % how far the line-numbers are from the code
  numberstyle=\tiny\color{gray}, % the style that is used for the line-numbers
  breaklines=true,
  postbreak=\mbox{\textcolor{red}{$\hookrightarrow$}\space},
}

\author{}
\title{}
\date{}
\makeindex
\makeglossaries
\loadglsentries{glossary.tex}
\glsaddall

\begin{document}

	\pagestyle{empty}
	
	%-------------------------------------------------------------
	\begin{center}
		{
		\Large\bf
		Implementierung einer Software zur Vereinfachung des Entwicklungsprozesses in
		der Firmware-Entwicklung für eingebettete Systeme mit ARM-Prozessoren
		}\\[3cm]

		{\bf Bachelorarbeit von Michael Luis Calvert}\\[1.5cm]

		an der\\
		Fachhochschule Aachen, Campus Jülich\\
		Fachbereich Medizintechnik und Technomathematik\\
		Studiengang {\em Scientific Programming}\\[3cm]

		%vorgelegt von\\
		%Michael Luis Calvert\\
		%Matrikelnummer: 4012564\\[3cm]
		Köln, \today\\[3cm]

		Diese Arbeit wurde betreut von:\\[0,2cm]
		1. Prüfer: Prof.~Dr.~Volker Sander\\
		2. Prüfer: B.~Sc.~Lukas Knuth

	\end{center}


	%\maketitle

	%{
	%	\vspace{30mm}
	%	\centering
	%	Fachhochschule Aachen \\
	%	Scientific Programming\\
%	}

	\newpage
	\blankpage
	%----------------------- Versicherung
	Diese Arbeit ist von mir selbstständig angefertigt und verfasst.
	Es sind keine anderen als die angegebenen Quellen und Hilfsmittel benutzt worden.\\[3cm]
	Michael Luis Calvert\\[3cm]
	\begin{tabular}{p{10cm}@{}l@{}}\hline
	Datum, Unterschrift
	\end{tabular}
	
	\newpage
	\blankpage
	
	\begin{abstract}
Das Ziel der vorliegenden Bachelorarbeit ist es einen Überblick über die Entwicklung von Embedded-Systemen zu schaffen
und ein Konzept zur Vereinfachung der Entwicklung von Hardware-Applikationen aufzustellen.

Dazu werden zunächst unterschiedliche Technologien auf verschiedenen Abstraktionsebenen erklärt, aus denen ein
Schichtenmodell für Embedded-Systeme aufgestellt wird. Berücksichtigt werden dabei alle Aspekte des Entwicklungsprozesses.

Anschließend erfolgt die Dokumentation der Entwicklung einer Erweiterung für integrierte Entwicklungsumgebungen. Diese
ermöglicht es einem Entwickler die Hardware-Applikation mit Hilfe der vorgestellten Technologien und dem Programm
OpenOCD auf die Zielhardware aufzuspielen und über die Debugging-Software GDB auf Fehler zu untersuchen.
\end{abstract}
	
	%\newpage

	\tableofcontents
	
	\listoffigures
	
	\listoftables
	
	\lstlistoflistings
	
	\newpage
	
	% begin fancy headers
	\pagestyle{fancy}

	% --- for full literature list without using \cite command
	%\nocite{*}

	\section{Einleitung}
	\subsection{Steigende Popularität von Embedded-Entwicklung}
	In den letzten Jahren hat der Bereich der Embedded-Systeme deutlich an Popularität gewonnen.
	Embedded-Systeme bezeichnen prozessorbasierte Hardware, die in ein bestehendes System \textit{eingebettet} sind
	und eine dedizierte Aufgabe übernehmen.
	Vorangetrieben durch Trends, wie \glsit{glo:iot}, nehmen immer mehr IT-Unternehmen Embedded-Systeme
	in ihr Repertoire auf.
	``These embedded systems are increasingly being joined together into an ``Internet of things.'' In the future,
	machines that are linked together in networks could revolutionize manufacturing, energy distribution, and
	transportation systems.''\citep[]{Siemens2014}

    Mitverantwortlich für den Aufschwung der Embedded-Entwicklung ist die Arduino\hyp{}Plattform, welche bereits seit 2005
    den Fokus auf einfach zu verwendende Software und Hardware setzt. Ziel der Plattform ist es, vor allem Studenten die
    Möglichkeiten zu geben, schnelles Hardware-Prototyping ohne große Vorkenntnisse im Hardware- und Programmierbereich
    durchzuführen
    \footnote{Laut der \href{https://www.arduino.cc/en/Guide/Introduction}{Hersteller-Seite} Zugriff: 5. Juli 2017}.
    Während die Einfachheit und Schnelligkeit ein großer Verkaufspunkt der Arduino\hyp{}Plattform ist, birgt sie auch große
    Nachteile, die es Entwicklern schwer macht damit zu arbeiten. Genauer wird dies in \ref{sec:arduino} analysiert.

	``Das Gesamtmarktvolumen für Embedded-Systeme für Deutschland wird auf über 18,7 Mrd. im vergangenen Jahr geschätzt.
      Die Anbieter tragen in 2007 dazu 3,7 Mrd. Euro bei und die Wertschöpfung in den Anwendersektoren umfasst mehr als
      15 Mrd. Euro.''\citep[vgl. S. 1]{Bitkom2009}

	Die wachsende Relevanz der Mikrocontroller-Programmierung resultiert darin, dass sich mehr Entwickler mit der
	Thematik beschäftigen und einen Einstieg finden wollen. Vor allem im Bereich der Programmierung ist der Bedarf an
	Entwicklern groß.
	``Von den vier Bereichen Hardware Design, Produktion, Programmierung und Integration von Embedded-Systemen
    ist die Softwareentwicklung und -programmierung der mitarbeiterintensivste Bereich.''\citep[vgl. S. 24]{Bitkom2009}

    In der Regel bilden E-Techniker und Studierte der Technischen Informatik den Großteil der Entwickler für den
    Embedded-Bereich, jedoch finden sich immer mehr Quereinsteiger (unter denen sich viele Software-Entwickler befinden)
    in die Programmierung für Mikrocontroller ein.

    Zusätzlich zu der wachsenden Anzahl an Entwicklern steigt auch die Anzahl an Entwicklungswerkzeugen und es stehen
    immer mehr davon frei zur Verfügung. Die Nähe zur Hardware bringt allerdings das Problem mit sich,
    dass diese Entwicklungswerkzeuge meistens auf ein spezifisches Modell und somit auch auf einen spezifischen
    Hersteller eines Prozessors zugeschnitten sind.

    Zudem mangelt es an kostenfreien Möglichkeiten die Firmware eines Prozessors auf Fehler zu untersuchen.

    Das Open-Source-Programm
    \glsit{glo:openocd} verfolgt das Ziel die Firmware eines Prozessor über \glsit{glo:gdb} debuggen zu können.
    ``The GNU Debugger
      (gdb) offers excellent debugging support, but covers only some areas of embedded systems debugging.
      Low level tasks require additional hard- and software, and existing open source solutions for these tasks
      are incomplete or at least partially deficient.''\citep[vgl. Introduction]{Rath2005}

    Diese Arbeit beschäftigt sich mit dem Entwicklungsprozess für Embedded\hyp{}Systeme, speziell für die von \glsit{glo:arm}
    entwickelten Cortex\hyp{}M\hyp{}Serien.
    Dabei stehen die Aspekte der Chip-Programmierung und der Fehleranalyse im Vordergrund. Das Ziel dieser Arbeit
    ist es, die Probleme und Herausforderungen, vor allem für Neueinsteiger, der Entwicklung für Embedded-Systeme herauszufinden und
    einen allgemeinen Lösungsansatz dieser Problematiken sowohl für Neueinsteiger, als auch für fortgeschrittene
    Entwickler im Embedded-Bereich zu erarbeiten.

    \subsection{Gliederung}
    Um bestimmte Design-Entscheidungen besser verstehen zu können, folgen zunächst theoretische Hintergründe zu
    \glsit{glo:arm}-Cortex-M-Microcontrollern. Die Betrachtungsweise ähnelt dabei der eines Software-Entwicklers.

    Im zweiten Abschnitt wird zunächst die \glsit{glo:arm}-Cortex-M-Familie vorgestellt und ein allgemeiner Entwicklungsprozess
    beschrieben. Hierbei wird auf Nachteile und übliche Probleme dieses Prozesses eingegangen.

    Der dritte Abschnitt befasst sich mit vorhandenen Abstraktionen (nicht Betriebssystemen) und Protokollen, welche
    dem Entwickler einen einfacheren Zugriff auf die Hardware ermöglichen sollen oder bestimmte Funktionalitäten, wie
    zum Beispiel Debugging, erfüllen.

    Der vierte Teil geht auf die Software-seitige Fehleranalyse mit \glsit{glo:gdb} und vorhandene Software für On-Chip-Debugging
    ein. Unterschiede zwischen Software- und Hardware-Debugging sollen hier dargelegt werden.

    Als fünfter Teil folgt ein eigener Lösungsvorschlag zur Vereinfachung der Entwicklung von Embedded-Systemen.
    Ziel ist es, auf Software- wie auf Hardware-Seite, eine Lösung zu finden, die den Entwicklungsprozess für
    Embedded-Systeme vereinfacht. Der praktische Teil dieser Arbeit ist es, den ersten Schritt auf der Software-Seite
    zu implementieren. Hierzu wird mit Hilfe von \glsit{glo:openocd} ein \acrshort{glo:ide}-Plugin speziell für die Embedded-Entwicklung
    implementiert. Es folgt außerdem eine Einordnung der Rolle dieses Arbeitsschrittes in Hinblick auf die Gesamtlösung.

    Anschließend wird der Inhalt der Arbeit zusammengefasst und ein Ausblick über die Weiterentwicklung der zu
    implementierenden Plattform für Embedded-Systeme gegeben.
    \\

    Als zu testende Hardware, wird ein von Atmel hergestelltes Board mit einem Cortex-M7 Prozessor (\texttt{AT SAM E70 Q21})
    verwendet. Das Board enthält außerdem eine Debug-Einheit mit \acrshortit{glo:cmsisdap}-Unterstützung.
	
	\section{Einführung in Embedded-Systeme}
		\subsection{Arduino}
		\label{sec:arduino}
		Da Arduino mitverantwortlich für den Anstieg des Interesses an Microcontroller-Programmierung in der
		Entwicklerszene ist, wird im Folgenden eine kurze Übersicht über die angebotene Technologie geliefert.

		Wie oben schon aufgeführt, ist Arduino eine Plattform für die Entwicklung mit Mikrocontrollern. Dabei werden
		sowohl Software- als auch Hardware-Komponenten angeboten, um Embedded-Programmierung zu betreiben.

		Es ist ein kostenfreies \glsit{glo:ide} verfügbar, über die sogenannte \glsplit{glo:sketch} entwickelt
		werden können. Über \glsit{glo:usb} werden diese Programme auf das Board heruntergeladen.
	    Außerdem stehen viele Ressourcen und Programmbeispiele online sowie in zahlreichen Lehrbüchern zu Verfügung.
	    Programmiert wird mit einer reduzierten Version der Sprache C. Um einen einfachen \glsit{glo:sketch} zu schreiben,
	    müssen lediglich die beiden Funktionen \texttt{setup()} und \texttt{loop()} implementiert werden.

        Die meisten Prozessoren der Arduino-Hardware stammen aus der Atmel-AVR Familie.
        \\

	    Während Arduino die allgemeine Entwicklung mit Mikrocontrollern sehr stark vereinfacht, sind diese
	    Vereinfachungen oft mit Kosten verbunden.
	    \\

	    Die vereinfachte Sprache und das Framework insgesamt ermutigt den Nutzer den gesamten Code in einen \glsit{glo:sketch}, also
	    eine Datei zu schreiben. Dies ist für viele Programmierer das genaue Gegenteil von dem, was in der Theorie
	    gelehrt wird: Logische Aufteilung von Programmcode in Module, sprich Modularisierung.
	    Möchte man trotzdem ein modularisiertes Programm entwerfen, wird der Nutzer gezwungen auf andere Sprachen und
	    Werkzeuge zurückzugreifen (zum Beispiel \textit{Atmel Studio}).

		Weiterhin ist das Debuggen von Firmware nur mit entsprechender Hardware möglich (siehe \ref{sec:jtag}). Wichtige
		Funktionalitäten, wie zum Beispiel Breakpoints, werden dem Arduino-Nutzer (angenommen man verwendet das
		Arduino-Software-Framework) vorenthalten. Hier wird in der Regel über serielle Ausgaben, ähnlich wie \texttt{printf()},
		ein Fehler gesucht.

		Arduino vereinfacht also den Einstieg in die Entwicklung von Mikrocontrollern enorm, aber macht dafür eine
		etwas anspruchsvollere Entwicklung um einiges schwieriger.

		\subsection{Raspberry Pi}
        Raspberry Pi ist im Vergleich zu Arduino keine gesamte Entwicklungsplattform, sondern bezeichnet lediglich
        einen \textit{Single Board Computer}, also einen kleinen (minimalen) Heim-Computer. Dieser verfolgt das Ziel eine
        einfache Anlaufstelle für das Experimentieren mit Computer-Systemen, vor allem für Schüler und
        Studenten, zu schaffen.

        Die Prozessoren variieren je nach Modell, gehören aber meistens der \glsit{glo:arm}-Cortex-A-Familie an.

        Da der Raspberry Pi einen Heimcomputer darstellt, wird in der Regel ein vollständiges und meistens angepasstes
        Betriebssystem (beispielsweise \textit{Raspbian}, ein auf Debian basierendes Betriebssystem) installiert.

        Mit Peripheriegeräten, welche sich nicht direkt auf dem Board befinden, kann über die \glsplit{glo:gpio}
        kommuniziert werden.

        Die Entwicklung mit dem Raspberry Pi ist durch die Abstraktion mit Betriebssystem nicht sehr Hardware-nah.
        Außerdem bietet der Raspberry Pi im Vergleich zu Personal Computers einen deutlichen Preisvorteil, ist jedoch
        kostspieliger als andere Mikrocontroller.

        Für die meisten Anwendungsbereiche innerhalb der Entwicklung für Embedded-Systeme ist ein Raspberry Pi
        letztendlich ein Übermaß, da die meisten Projekte einen sehr spezifischen Fall oder eine sehr spezifische
        Funktion erfüllen müssen und das Raspberry Pi aber darauf ausgelegt ist mit einem Betriebssystem zu
        laufen und somit die Komplexität deutlich erhöht wird. In Abschnitt \ref{sec:os} wird genauer auf die Thematik
        von Betriebssystemen bei Embedded-Systemen eingegangen.

	\subsection{Die ARM-Cortex-Prozessorfamilien}
	\label{sec:cortex_fam}
	In der \glsit{glo:arm}-Cortex-Prozessorfamilie gibt es 3 grundlegende Subklassen.
	\begin{description}
	    \item[Cortex-A] Das A steht für \textit{Application} und beschreibt die auf Hochleistung fixierte Prozessorfamilie.
	    Sie werden meistens in mobilen Geräten (Smartphones, Tablets, etc...) eingesetzt und können mit einer
	    relativ hohen Taktung betrieben werden.
	    ``While the Cortex-A processors have high performance, the processor is not designed to
          provide rapid response time to hardware events (i.e., real-time requirements)''\citep[Kap.~0]{Yiu2015}
	    \item[Cortex-R] Prozessoren der Cortex-R-Familie sind ebenfalls als Hochleistungsprozessoren gedacht und
	    vor Allen für Echtzeitsysteme entwickelt. Sie garantieren ein deterministisches Verhalten und werden in
	    Systemen verwendet, welche zuverlässige Antworten in Echtzeit benötigen
	    \footnote{siehe \href{https://www.arm.com/products/processors/cortex-r}{ARM Produkt-Seite}}.
	    Praktische Verwendung finden die Prozessoren beispielsweise in der Automobilbranche, im industriellen Bereich
	    und in der Medizintechnik.
	    Die Prozessoren gelten aufgrund ihrer komplexen Architekturen als nicht sehr energieeffizient
	    \citep[vgl.~Kap.~0]{Yiu2015}.
	    \item[Cortex-M] Cortex-M-Prozessoren besetzen den \glsit{glo:mainstream}-Markt für Mikrocontroller und gelten als
	    energieeffizient. Sie besitzen eine kürzere Pipeline zum Abarbeiten von Befehlen und laufen mit einer geringeren
	    Taktung. Sie zeichnen sich außerdem hauptsächlich durch ihre Benutzerfreundlichkeit, kurze Antwortzeit und ihren
	    geringen Platzaufwand aus.\\
	    Der eingebaute \glsit{glo:nvic} bietet ein mächtiges und doch einfach zu nutzendes
	    Interrupt Management\citep[vgl. Kap. 0]{Yiu2015}.
	    Cortex-M-Prozessoren werden am häufigsten in Embedded-Systemen eingesetzt und besitzen eine Variation von
	    verschiedenen Modellen für Anwendungsgebiete deren Anforderungen von Energieeffizienz und minimalen Kosten bis
	    zu Hochleistung reichen.
	    Abbildung \ref{fig:arm_cortex_m_rel_performance} zeigt die Leistungsunterschiede der Cortex-M-Modelle relativ
	    zum Cortex-M0-Prozessor-Typ.

%TODO CORMARK TEST
	    \begin{figure}[h]
	        \caption{Relative Leistung der verschiedenen Cortex-M Modelle}
	        \source{
	            \href{
	                https://www.arm.com/-/media/arm-com/products/processors/Cortex-M-series-performance-graph.jpg?la=en
	            }{
	                ARM Produkt Seite
	            }, Zugriff: 17.08.2017
	        }
	        \label{fig:arm_cortex_m_rel_performance}
            \centering
            \includegraphics[scale=0.3]{../Graphiken/Cortex-M-series-performance-graph.jpg}
	    \end{figure}

	\end{description}

	\subsection{Erläuterung allgemeiner Begriffe}
	Im folgenden Abschnitt werden häufig verwendete Begriffe kurz erläutert, um Missverständnisse bei
	Formulierungen vorzubeugen und den späteren Kontext besser verstehen zu können.
	    \begin{description}
	        \item[Prozessor] Der Prozessor ist ein programmierbarer digitaler integrierter Schaltkreis und gilt als
	        innerstes Modul eines Systems. Er übernimmt die Ausführung von Maschinenbefehlen und steuert andere
	        elektrische Schaltungen\citep[vgl.~Kap.~1]{Asche2017}.
	        \item[Mikrocontroller] Ein Mikrocontroller ist ein zusammenfassender Begriff für Prozessoren, Speicher und
	        Ein- und Ausgabeperipherie.
	        \item[SoC] \glsit{glo:soc} bezeichnet die Zusammenfassung oder Integration von
	        mehreren Funktionen eines programmierbaren Systems auf einem integrierten Schaltkreis. Auch Mikrocontroller
	        lassen sich als \glsit{glo:soc} bezeichnen.
	        \item[RAM] Bei \glsit{glo:ram} kann auf jede Speicherzelle über die Speicheradresse direkt
	        zugegriffen werden. Dabei ist das als Arbeitsspeicher verwendete \glsit{glo:dram} ein
	        flüchtiger Speicher. Im Gegensatz dazu ist \glsit{glo:sram} nicht flüchtig und wird in Caches, also Zwischenspeicher,
	        genutzt. Der Zugriff auf diesen Speicher erfolgt im Gegensatz zu Massenspeichern schnell.
	        \item[ROM] \glsit{glo:rom} bezeichnet einen permanenten Speicher. Der Zugriff auf diesen
	        Speicher erfolgt im Vergleich zum \glsit{glo:ram} langsam.
	        \item[Instruction Set] Als \textit{Instruction Set} wird die Menge der vom Prozessor verstandenen Anweisungen
	        bezeichnet. Anweisungen können den Kontrollfluss verändern, Daten aus dem Speicher lesen, schreiben oder
	        manipulieren\citep[]{Asche2017}.
	        \item[Firmware] Als Firmware wird das vom Prozessor auszuführende Programm bezeichnet. Dabei kann es, falls
	        vorhanden, vom Arbeitsspeicher oder vom Flash-Speicher ausgeführt werden. Das Programm wird dabei als
	        \textit{Image} in den Speicher geladen und ausgeführt.
	        \item[Peripherie] Alle externen Geräte und Anschlüsse, die sich auf dem Board bzw. innerhalb des
	        eingebetteten Systems befinden und mit dem Prozessor kommunizieren (siehe auch \textit{Pins}).
	        \item[Pin] Pins sind Kontakte, die zum Beispiel vom Prozessor für die Kommunikation mit der ``Außenwelt''
	        genutzt werden. Die Pins eines Mikroprozessors können in der Regel mehrere Funktionen übernehmen. Beim
	        Hochfahren muss dem Prozessor mitgeteilt werden, wie er bestückt ist und wie er deswegen die Pins
	        behandeln muss\citep[vgl.~Kap.~1.3]{Asche2017}.
	        \item[Serielle Schnittstelle] Ein serieller Datenfluss bezeichnet das Versenden von Daten in Form von
	        aufeinanderfolgenden Bits. Das Gegenstück zu seriell ist die parallele Übertragung von Daten, in der über
	        mehrere Datenströme parallel gesendet wird.
	        \item[Flash-Speicher] Ein nicht-flüchtiger Speicher, welcher elektronisch gelöscht und programmiert werden
	        kann.
	        \item[RTOS] \glsit{glo:rtos} bezeichnet ein Betriebssystem mit
	        Echtzeitanforderungen. Das heißt, es muss innerhalb eines bestimmten Zeitraums antworten und Daten ohne
	        Pufferverzögerungen bearbeiten.
	        Im Embedded\hyp{}Bereich kommen Echtzeitbetriebssysteme oft zum Einsatz, da sie ``im Gegensatz zu anderen
	        Betriebssystemen dem Entwickler einen hohen Grad der Kontrolle darüber, welche Ereignisse mit welcher
	        Priorität behandelt werden können[, geben]''\citep[Kap.~3.1]{Asche2017}.
	        Genaueres wird in Abschnitt \ref{sec:os} erläutert.
	        \item[Interrupt] Ein Interrupt bezeichnet ein unterbrechendes Ereignis.
	        Diese spielen vor allem bei Echtzeitsystemen eine große Rolle. Sie signalisieren der Hardware sich sofort
	        auf ein anderes Ereignis zu konzentrieren. Der Prozessor unterbricht daraufhin alle Aktivitäten und
	        wendet sich durch einen \textit{Interrupt Handler} dem besagtem Ereignis zu.
	        ``In vielen Prozessoren genießen Interrupts einen privilegierten Status, indem sie in einer
              höheren Sicherheitsstufe und möglicherweise in einem anderen Kontext als Applikations-Code ausgeführt
              werden.''\citep[Kap.~1.5]{Asche2017}
	        \item[Host] Als Host wird die Instanz bezeichnet, von der das Programm auf die Zielhardware gespielt wird.
	        Hierbei handelt es sich meistens um den Entwicklungsrechner.
	        \item[Target] Die Zielhardware wird als Target (also ``Ziel'') bezeichnet.
	        \item[PCB] \glsit{glo:pcb} ist englisch für Leiterplatte. Ein \glsit{glo:pcb} enthält
	        mehrere \glsplit{glo:ic} oder andere elektronische Bauteile, die miteinander Verbunden sind.
	    \end{description}
	\subsection{Betriebssysteme}
	\label{sec:os}
	Wie in Abschnitt \ref{sec:cortex_fam} erwähnt, gibt es für verschiedene \glsit{glo:arm}-Cortex-Prozessorserien unterschiedliche
	Anwendungsgebiete. Damit stellt sich auch die Frage, welches Betriebssystem am besten geeignet ist und ob
	überhaupt ein solches benötigt wird.

	Die Aufgabe eines Betriebssystems ist es die Hard- und Software eines Systems zu verwalten. Es stellt eine
	Abstraktionsebene zwischen System-Hardware und Applikationen dar und ist für die Bereitstellung von verschiedenen
	Diensten zuständig. Dabei arbeitet das Betriebssystem logisch unterteilte Aufgaben ab. Diese können unabhängig
	voneinander ausgeführt und, falls nötig, bei Abschluss wieder miteinander synchronisiert werden.

	Bei Embedded-Systemen ist die Frage, ob und welches Betriebssystem benötigt wird, oft mit mehreren Faktoren
	verbunden. Ein großer Entscheidungspunkt ist die zur Verfügung stehende Hardware. Oft wird bei der Entwicklung
	von Embedded-Systemen versucht so viel wie möglich an Kosten zu sparen, d.h. die Hardware wird oft so minimal wie
	möglich gestaltet. Da ein Betriebssystem durch seine Abstraktion und Bereitstellung von verschiedenen Diensten und
	Programmen großen \glsit{glo:overhead} erzeugt, gestaltet sich die Benutzung eines Betriebssystems bei
	Embedded-Systemen oft als problematisch.
	``Wenn also ein Abstraktionslayer zwischen der konkreten Implementation und dem sie benutzenden Code liegt, bedeutet
	das oft größeren Stack- und CPU-Zeitbedarf allein durch den Aufruf der Abstraktionsfunktion, der bei knapp
	geschneiderten Plattformen durchaus schmerzhaft sein kann!''\citep[Kap.~2.2]{Asche2017}.

	Prinzipiell lässt sich also die Regel aufstellen, dass eine ``einfache'' und nicht parallel laufende Anwendung auf
	einem Embedded-System kein Betriebssystem benötigt, da die meisten Dienste des Betriebssystems nicht gebraucht werden
	und somit unnötigen \glsit{glo:overhead} erzeugen.

	Auch komplexere Anwendungen können vollständig ohne Betriebssystem laufen. Die normalerweise von einem
	Betriebssystem logisch unterteilten und abgearbeiteten Teilaufgaben werden von der eigens entwickelten Firmware
	übernommen,
	``in der Praxis [wird] in diesen Fällen normalerweise die Gliederung in Teilaufgaben von im Hause entwickelten
	Komponenten wie message pumps [...] übernommen [...]. Diese Komponenten lassen sich bereits als kleine
	Betriebssysteme bezeichnen''\citep[Kap.~3.2]{Asche2017}.

	Anders ist die Lage bei mobilen Geräten, wie zum Beispiel Smartphones und Tablets. Dieser Anwendungsbereich
	erfordert viele Funktionen eines Betriebssystems, wie zum Beispiel ein File-System, Speicherverwaltung und Weiteres.
	Die hier verwendeten leistungsfähigen Prozessoren der Cortex-A-Serie bieten sich deshalb für
	die Verwendung eines Betriebssystems, beispielsweise Linux, iOS, Android, etc an.

	Für Echtzeitsysteme existieren die oben schon erwähnten \glsplit{glo:rtos}. Diese sind in frei
	verfügbaren Ausführungen, wie z.B. FreeRTOS, vorhanden oder auch kostenpflichtig verfügbar.

	Ein Echtzeitsystem ist dadurch definiert, dass es genau festgelegte Zeitschranken für Prozesse erfüllen muss, um
	ernsthafte Konsequenzen (wie zum Beispiel ein Systemausfall) oder Misserfolge zu vermeiden. Als Misserfolg wird
	jedes Systemverhalten definiert, welches die formale Systemspezifikation nicht erfüllt
	\citep[vgl.~Kap.~2.2.5]{Brause2017}.

    Zwei Verfahren, die beim dynamischen Prozess-Scheduling für Echtzeitsysteme verwendet werden, sind Polling und
    Interrupts\footnote{\citep[vgl.~Kap.~2.2.5]{Brause2017}}:
    \begin{description}
        \item[Polling] Der Prozessor fragt alle verbundenen Geräte in einer Schleife nacheinander ab, ob ein Ereignis
        oder neue Daten vorhanden sind. Tritt ein dringendes Ereignis während der Bearbeitung eines anderen Ereignisses
        auf, scheitert diese Strategie.
        \item[Interrupt] Der Prozessor befindet sich in einer \textit{Idle Loop} und kann von
        \glsplit{glo:irq} unterbrochen werden. Dabei arbeitet die \glsit{glo:isr} das
        unterbrechende Ereignis ab, wobei der entsprechenden \glsit{glo:isr} auch eine Priorität zugewiesen werden kann. In diesem
        Fall findet zusätzlich ein Prioritätsscheduling statt. Hierbei liegt die Problematik bei einer Anhäufung von
        Interrupts mit hoher Priorität, was dazu führen kann, dass Interrupts mit geringer Priorität ignoriert oder
        überschrieben werden.
    \end{description}

    Für den restlichen Verlauf der Arbeit wird angenommen, dass kein Betriebssystem benötigt wird.

    \subsection{Interrupthandling bei ARM-Cortex-Prozessoren}
    Interrupts stellen ein Kernkonzept der Firmware-Entwicklung dar. Durch sie kann der geregelte Programmablauf
    beeinflusst werden. Ein Interrupt kann genutzt werden, um einen Dienst vom Prozessor anzufordern (zum Beispiel von
    einem Peripheriegerät).
    Folgender Ablauf ergibt sich bei dem Senden eines Interrupts\footnote{\citep[vgl.~Kap.~7]{Yiu2013}}:
    \begin{enumerate}
        \item Es wird ein \glsit{glo:irq} an der Prozessor geschickt
        \item Der Prozessor suspendiert den aktuellen Task (bei geringerer Priorität)
        \item Der Prozessor arbeitet den mit dem Interrupt gekoppelten \textit{Interrupt Handler} (also eine Funktion) als
        \glsit{glo:isr} ab
        \item Der Prozessor führt nach Abschluss den ursprünglichen Task aus
    \end{enumerate}

    Verantwortlich für das Verarbeiten der Interrupts ist der \glsit{glo:nvic}, der für die
    Abarbeitung von Interrupts und Exceptions zuständig ist, wobei in den Architekturen der \glsit{glo:arm}-Cortex-Serien Interrupts
    als Exceptions gelten. Die verschiedenen \textit{Interrupt Handler} sind in der \glsit{glo:ivt} gelistet,
    welche am Anfang eines jeden Programms initialisiert und an der korrekten Stelle im Speicher hinterlegt werden muss.
    Bei den meisten Chip-Architekturen der \glsit{glo:arm}-Cortex-M-Serie bietet diese \glsit{glo:ivt} Platz für 256 Interrupt-Vektoren.

    \begin{table}[h]
        \centering
        \caption{Beispiel für eine \glsit{glo:ivt}}
        \label{tab:itvexample}
        \begin{tabular}{|c|c|}
            \hline
            Interrupt-Nummer    & Speicheradresse des \glsit{glo:isr}       \\ \hline
            1                   & Interrupt-Service-Routine 1   \\ \hline
            2                   & Interrupt-Service-Routine 2   \\ \hline
            3                   & Interrupt-Service-Routine 3   \\ \hline
            $\vdots$            & $\vdots$                      \\ \hline
        \end{tabular}
    \end{table}

    Tabelle \ref{tab:itvexample} zeigt eine beispielhafte \glsit{glo:ivt}. Jeder Eintrag besteht aus einer Interrupt Nummer und
    einem Funktions-Pointer, der auf den \textit{Interrupt Handler} zeigt.

    Jedem Interrupt kann eine Priorität zugewiesen werden. Dabei können diese zwischen 0 und 255 (8 Bit) liegen.
    Hierbei ist Priorität 0 die höchste und 255 die geringste. Wie viele programmierbare Prioritäten letztendlich
    möglich sind, entscheidet der Chip-Hersteller. Eine hohe Anzahl an programmierbaren Prioritäten kann die Komplexität
    eines \glsit{glo:nvic}s, sowie auch den Stromverbrauch erhöhen und gleichzeitig die Leistung bzw. die Geschwindigkeit der
    Architektur verringern\citep[vgl.~Kap.~7.4]{Yiu2013}.

    Die ersten 16 Einträge der \glsit{glo:ivt} sind für den Prozessorkern reserviert. Sie besitzen feste negative Prioritäten.
    Ein Beispiel ist der Vektor 12 bei Adresse \texttt{0x0000\_0030}, der den \textit{Debug Monitor Interrupt Handler} enthält.
    Dieser wird von Debuggern zum Verwalten von Breakpoints benutzt, kann aber auch durch die Maschinenanweisung
    \texttt{bkpt} erzwungen werden\citep[vgl.~Kap.~2.8]{Asche2017}.
    Die restlichen Einträge werden vom Prozessorhersteller definiert.

    Die Stelle, an der sich der gesuchte Interrupt-Vektor im Speicher befindet, lässt sich mit Hilfe der
    Interrupt-Nummer berechnen. Standardmäßig fängt die \glsit{glo:ivt} bei Adresse 0 (also \texttt{0x0} in hexadezimal) an und die
    Adresse eines Vektors lässt sich durch $x \cdot 4$ errechnen, wobei $x$ die Interrupt-Nummer ist
    \citep[vgl.~Kap.~7.5]{Yiu2013}.\\
    \textbf{Beispiel:} Der erste nicht reservierte Interrupt besitzt die Nummer 16. Somit liegt der Funktions-Pointer
    für den Interrupt Handler bei der Adresse $(16 \cdot 4)_2 = (64)_2 = (40)_{16}$, also \texttt{0x40}. Je nach
    Architektur kann diese Rechnung variieren.

	\subsection{Allgemeine Voraussetzungen zur Entwicklung von Embedded\hyp{}Systemen}
	\label{sec:requirements}
	Ohne Betriebssystem tun sich allerdings im Hinblick auf die Entwicklung einige Schwierigkeiten auf.

    Abgesehen davon, dass ein Entwickler die Prozessorarchitektur genau kennen muss, stellt sich die Frage, wie denn eine
    entwickelte Firmware aufgespielt werden und eine Fehlersuche stattfinden kann.

    Das Programm-Image muss an die richtige Startadresse des Speichers geschrieben werden, damit
    der Prozessor alle im Programm vorkommenden Befehle in korrekter Reihenfolge abarbeiten kann.
    \\\textbf{Beispiel:} Beim \texttt{ATSAM E70 Q21} lautet diese Adresse \texttt{0x00400000}.

    Besagte Startadresse geht aus der \glsit{glo:ivt} hervor. Das Herunterladen der
    Firmware in den Speicher kann von Hand implementiert werden, es bieten sich allerdings eine Reihe an Werkzeugen an,
    welche diese Aufgabe übernehmen.

    Außerdem erfordert die Abwesenheit eines Betriebssystems eine Schnittstelle um die installierte Firmware
    auf Fehler untersuchen zu können. Hier gibt es verschiedene Möglichkeiten:
    \begin{itemize}
        \item Target-seitig ist ein Debug-Interface vorhanden und kann über \glsit{glo:usb} angesprochen werden. Genaueres wird in
        Abschnitt \ref{sec:cmsisdap} erklärt.
        \item Externe Debugging-Hardware. Diese Hardware konvertiert Befehle, die über \glsit{glo:usb} gesendet werden,
        in \acrshortit{glo:jtag}- und \acrshortit{glo:swd}-Befehle. Genaueres hierzu in Abschnitt \ref{sec:jtag} und \ref{sec:swd}.
        \item Eine weitere Debugging-Technik ist \glsit{glo:rtt}. Hierbei nimmt ein \glsit{glo:rtt}-Modul alle Aktivitäten
        des Prozessors zur Laufzeit der Programms auf, zum Beispiel Lese- und Schreibzugriffe auf den Speicher. Mit den richtigen
        Werkzeugen kann dann post mortem eine Analyse des Programms stattfinden. Dies ist besonders bei
        Echtzeitsystemen von Nutzen\citep[vgl.~S.~3]{Campbell2014}. Genaueres hierzu in Abschnitt \ref{sec:trace}.
    \end{itemize}

    Die Fehleranalyse bei Embedded-Systemen unterscheidet sich stark von dem für einen Software-Entwickler gewohnten
    Debug-Prozess. Beispielsweise muss bei Breakpoints zwischen Hard- und Software Breakpoints unterschieden werden und
    darauf geachtet werden, wie viele Hardware Breakpoints vom Chip unterstützt werden. Wie diese Breakpoints
    funktionieren und wann man welche Art verwenden kann wird in Abschnitt \ref{sec:debugfeatures} erklärt.

    Doch bevor es zur Fehleranalyse kommt, muss die Firmware erst einmal entwickelt werden. Dazu benötigt man vorerst einen
    Compiler. Das Programm kann in jeder Programmiersprache geschrieben werden, die einen Compiler für die richtige
    Chip-Architektur besitzt und Binärdateien erzeugt. Üblicherweise wird hier C bzw. C++ genutzt, andere Sprachen werden
    wegen ihrer zu hohen Abstraktion oft nicht in Erwägung gezogen. Natürlich besteht auch die Möglichkeit, direkt
    in Assembler zu programmieren.

    ``Im Vergleich zu C und C++ „höhere“ Programmiersprachen wie Java spielen sich auf einer höheren Abstraktionsebene
    ab, das heißt sie verstecken bewusst Details wie Speicherverwaltung, Datenorganisation oder Hardware-Ansteuerung vor
    dem Entwickler. Diese erst einmal nicht schlechte Eigenschaft steht aber im direkten Gegensatz zu den Anforderungen,
    die an eingebettete Systeme gestellt werden.''\citep[Kap.~1.1]{Asche2017}

    Ein weiterer Grund für die Verwendung von C als Programmiersprache für die Entwicklung eingebetteter Systeme ist die
    Bereitstellung von C-Header-Dateien und Treiberbibliotheken durch einen Großteil an Chip-Hersteller.

    Für den weiteren Verlauf dieser Arbeit wird angenommen, dass die Firmware in der Sprache C entwickelt wird, da dies
    zum heutigen Zeitpunkt die relevanteste Sprache im Embedded-Bereich ist.

    Betrachtet man nun die bisher erarbeiteten Voraussetzung zur Entwicklung eines Embedded-Systems, kommt man zu
    folgender Liste:
    \begin{itemize}
        \item Host-PC zum Entwickeln der Software
        \begin{itemize}
            \item Cross-Compiler
            \item \acrshortit{glo:usb}-Verbindung zur Zielhardware
            \item Software zum Aufspielen der Firmware
            \item Software zur Fehleranalyse (genaueres hierzu in Abschnitt \ref{sec:gdb})
        \end{itemize}
        \item Debugging-Hardware
        \begin{itemize}
            \item Debug-Chip auf dem Board mit \acrshort{glo:cmsisdap}-Interface (siehe Abschnitt \ref{sec:cmsisdap}) \textbf{oder}
            \item Debug-Probe zum umwandeln von \acrshortit{glo:jtag}\hyp{}/\acrshortit{glo:swd}\hyp{}Befehlen (siehe\\Abschnitt \ref{sec:jtag} und \ref{sec:swd})
        \end{itemize}
        \item Zielhardware
        \begin{itemize}
            \item Das \glsit{glo:datasheet} liefert einen umfangreichen Überblick über den Aufbau und die Funktionalitäten des
            Prozessors
        \end{itemize}
    \end{itemize}
		\subsubsection{Einrichtung der Entwicklungsumgebung}
        Die Auswahl der richtigen Werkzeuge für die Entwicklung von Embedded-Systemen ist oft eine herausfordernde
        Aufgabe. Die Anzahl an verfügbaren Werkzeugen wirkt überwältigend und unübersichtlich.
        Zunächst ist die Frage zu klären, mit welcher Hardware man arbeiten wird. Viele Chip-Hersteller bieten online
        einen Assistenten an, der anhand von ausgewählten Kriterien die große Auswahl an verschiedenen Prozessoren auf
        die für den Nutzer relevanten beschränkt.

        Es besteht auch die Möglichkeit ganze Development Boards zu kaufen, welche zusätzliche Peripherie und
        zum Beispiel eine Debug-Einheit enthalten.

        Wofür man sich letztlich entscheidet, hängt von den Anforderungen des Projekts ab.

        In der Regel ist es üblich, sich erst einmal für einen Chip-Hersteller zu entscheiden (am besten für ein
        konkretes Modell) und anschließend die vom Hersteller bereitgestellten Werkzeuge zu nutzen.

        Wie in Abschnitt \ref{sec:requirements} bereits erwähnt benötigt man für die Firmware-Entwicklung auf Seiten des Hosts
        folgende Komponenten:
        \begin{description}
            \item[Cross-Compiler] Der geschriebene Source\hyp{}Code muss spezifisch für die Chip\hyp{}Architektur kompiliert werden.
            \item[Programmiergerät] Der kompilierte Source-Code in Form einer Binärdatei muss auf den Speicher des
            Prozessors heruntergeladen werden.
            \item[Debugger] Bei selbst geschriebener Firmware ist die Fehlersuche im Code so gut wie unausweichlich.
            Hierbei helfen einem Debug-Funktionalitäten wie Breakpoints, um Fehler im Code zu finden.
        \end{description}

        Verwendet man unterschiedliche Programme für die oben genannten Prozesse, bezeichnet man die Sammlung dieser
        Programme als \glsit{glo:toolchain}. Die Komplexität besteht nun darin, herauszufinden welche Werkzeuge mit der
        Zielhardware kompatibel sind und welche Werkzeuge am besten miteinander funktionieren.

        Atmel bietet beispielsweise ein Visual-Studio-Plugin mit dem Namen \textit{Atmel Studio} an. Genanntes Plugin
        funktioniert mit vielen von Atmel hergestellten Chips und unterstützt diverse Debugging-Features, wie zum Beispiel
        \textit{Trace} oder \glsit{glo:cmsisdap}. Auch das Aufspielen von Firmware funktioniert problemlos.
        Das Plugin erfüllt also alle oben genannten Voraussetzungen für die Entwicklung eines Embedded-Systems auf
        Seiten des Hosts, sofern das Board entsprechend mit debug-fähiger Hardware ausgestattet ist. Wie diese Hardware
        aussieht und funktioniert wird im Abschnitt \ref{sec:protocols} erarbeitet.

        Da Atmel Studio ein Plugin der von Microsoft entwickelten IDE \textit{Visual Studio} ist, bindet es den
        Entwickler an die Windows-Plattform. Für Entwickler auf anderen Systemen, wie zum Beispiel Linux oder Mac ist
        dies ohne Virtualisierung keine Option. Des weiteren kommt mit der benötigten Installation von Visual Studio ein
        großer \glsit{glo:overhead}, da dieses Programm sehr umfassend ist und somit die Entwicklungsumgebung deutlich verlangsamt.

        Eine Open-Source-Lösung zum Programmieren und Debuggen von Firmware bietet das Programm \glsit{glo:openocd}
        \footnote{Das Projekt wurde 2005 im Rahmen einer Masterarbeit von Dominic Rath
        angestoßen und seitdem von diversen Mitwirkenden weiterentwickelt}.
        Es unterstützt sowohl Windows als auch Unix-ähnliche
        Systeme. Lediglich der richtige Cross-Compiler fehlt, und muss separat vom Entwickler betrieben werden.

        Für die Produktion bietet sich das kostenpflichtige Toolset von Keil an, welches eine IDE mit
        Debugging-Unterstützung enthält. Keil ist eine der wenigen Lösungen, die eine sehr große Menge an
        Chip-Architekturen von verschiedenen Herstellern unterstützt.

	\subsection{Negative Aspekte des Enticklungsprozesses von Embedded\hyp{}Systemen}
	\label{sec:probleme}
	Durch eine genauere Betrachtung des Entwicklungsprozesses zeichnen sich mehrere Hürden und Probleme ab, die es zu
	bewältigen gilt:
	\begin{itemize}
	    \item Für Neueinsteiger in den Bereich der Embedded-Systeme, seien es bereits ausgebildete Software-Entwickler
	    oder Anfänger, ist die Entwicklung mit einer steilen Lernkurve verbunden. Teilweise verantwortlich
	    dafür ist die große Anzahl an unterschiedlicher Hardware und verschiedene Entwicklungsumstände (zum Beispiel
	    auf welchem Host-System entwickelt wird).
	    \item Fehlende oder nicht eingehaltene Standards zwischen Chip-Herstellern sind dafür verantwortlich, dass
	    Applikations-Code schwieriger zu portieren ist. Das heißt auch, dass bei einem Wechsel der Hardware ein neues
	    Interface studiert und in den bestehenden Applikations-Code integriert werden muss.
	    \item Der vorherige Punkt hat zu Folge, dass Entwickler oft an einen Hersteller gebunden sind, da die
	    \glsit{glo:hal} und somit auch die Treiber herstellerspezifisch sind. Ein Wechsel der
	    Hardware ist also mit größeren Umständen verbunden.
	    \item Auch besitzt fast jeder Hersteller seine eigene Sammlung an Enwicklungswerkzeugen, zusätzlich zu anderer
	    Software, die von Drittanbietern angeboten wird. Bestimmte Entwicklungswerkzeuge binden den Entwickler an
	    Plattformen, wohingegen frei verfügbare Software oft unübersichtlich gestaltet ist und somit vor allem für
	    Anfänger eine große Hürde darstellt.
	    \item Bestehende Abstraktionen auf einer höheren Ebene, wie zum Beispiel das Wiring-Projekt (unterstützt nur
	    AVR Prozessoren) \footnote{\href{http://www.wiring.org.co/}{Offizielle Seite \texttt{wiring.org.co}}}, binden den Entwickler an
	    bestimmte Hardware und schränken die Möglichkeiten der Erweiterung und Leistung von Hardware ein. Bei der
	    Wiring-Plattform ist dies dem Ziel geschuldet, möglichst einfaches und sicheres Hardware-Prototyping für unerfahrene
	    Entwickler anzubieten. Die Menge an Funktionalitäten um potenzielle Fehlerquellen abzufangen resultiert in
	    starken Leistungseinbußen.\\
	    \textbf{Beispiel:} Arduino-Shields sind Hardware-Erweiterungen, die auf ein Arduino-\glsit{glo:pcb} gebaut werden können.
	    Dabei müssen die Shields über unterschiedliche Pins angesteuert werden. Dies beschränkt die möglichen
	    Erweiterungen.
	    \item Es herrscht ein Mangel an effizienten und modernen Entwicklungsumgebungen, die ein Software-Enwickler
	    wahrscheinlich gewohnt ist, mit Unterstützung für Embedded-Systeme. Bereits erwähnte \glsplit{glo:ide}, wie zum Beispiel
	    Keil oder Atmel Studio sind entweder an eine Plattform und einen Hersteller gebunden oder nicht frei verfügbar.
	\end{itemize}

    Die aufgeführten Punkte stellen gleichzeitig die Anforderungen an eine Software-Plattform zur Vereinfachung der
    Entwicklung im Embedded-Bereich und werden in Abschnitt \ref{sec:demands} formuliert.


	\section{Hardware-Debugging, Protokolle und Abstraktionen}
\label{sec:protocols}
Bestimmte Protokolle und Abstraktionen bieten dem Entwickler einen einheitlichen Weg auf bestimmte Funktionen eines
Prozessors zuzugreifen. Es bleibt die Frage, wie ein Entwickler diese Technologien für den Entwicklungsprozess nutzen
kann.

In diesem Kapitel werden alle für den Entwicklungsprozess wichtigen Target-seitigen Technologien, vor allem in Hinblick
auf Debugging, erläutert.

    \subsection{Ein Schichtenmodell für Embedded-Systeme}
    \label{sec:layers}
    Die betrachteten Technologien funktionieren auf verschiedenen Abstraktionsebenen und stellen im Entwicklungskontext
    ein Schichtenmodell zwischen der tatsächlichen Zielhardware, der vom Nutzer entwickelten Applikation und den
    Entwicklungswerkzeugen dar.

    %TODO
    \begin{figure}[h]
        \centering
        \caption{Entwicklungsbasiertes Schichtenmodell der Hardware-Abstraktionen}
        \label{fig:layerhwab}
        \source{eigene Zeichnung}
        \includegraphics[scale=0.5]{../Graphiken/hwschichten_v2}
    \end{figure}

    Abbildung \ref{fig:layerhwab} zeigt den groben Aufbau des Schichtenmodells. \acrshortit{glo:jtag} und \acrshortit{glo:swd} (siehe Abschnitt
    \ref{sec:jtag} und \ref {sec:swd}) sind dabei direkt an die Hardware gebunden. Grund dafür ist, dass für den
    \acrshortit{glo:jtag}-Standard sowohl zusätzliche Hardware als auch eine Abstraktion durch das Protokoll vorhanden ist. Das \glsit{glo:cmsisdap}
    liegt dabei genau über dem \acrshortit{glo:jtag} bzw. \acrshortit{glo:swd}, da es die \acrshortit{glo:jtag}-Funktionalitäten direkt benutzt
    (siehe Abschnitt \ref{sec:cmsisdap}). \glsit{glo:cmsisdap} gehört zum \glsit{glo:cmsis}, welches auch andere Hardware-Abstraktionen,
    auf verschiedenen Schichten, wie zum Beispiel \glsit{glo:cmsis}-Core oder \glsit{glo:cmsis}-Driver, enthält (siehe Abschnitt
    \ref{sec:cmsis}).

    \acrshortit{glo:jtag}, \acrshortit{glo:swd} (optional) und \glsit{glo:cmsisdap} werden primär für das Testen und Aufspielen, sowie die Fehlersuche in
    Applikationen verwendet.

    Die restlichen aufgeführten Schichten dienen als vom Entwickler in der Applikation zu verwendende Abstraktionen.

	\subsection{JTAG}
	\label{sec:jtag}
	\acrshortit{glo:jtag} steht für \textit{Joint Test Action Group} und bezeichnet einen Standard zum Testen von Hardware, welcher 1985
	in Zusammenarbeit von verschiedenen Hardware-Herstellern erarbeitet wurde und als Standard des \glsit{glo:ieee} festgehalten ist.
	Der Standard wurde ursprünglich entwickelt, um Komponentenbasiertes Testen, Vernetzung und Interaktionen von
	verschiedenen Komponenten auf einem \glsit{glo:ic} zu vereinfachen bzw. zu ermöglichen\citep[vgl.~Kap.~2]{Rath2005}.
	Das Testverfahren wird \textit{Boundary Scan} genannt.
	Am häufigsten benutzt wird \acrshortit{glo:jtag} zum Programmieren und Debuggen eines \glsit{glo:ic}.

    \acrshortit{glo:jtag} umfasst sowohl ein Protokoll als auch die entsprechende Hardware. Die Hardware sitzt auf dem \glsit{glo:ic} und ermöglicht
    zusammen mit dem \acrshortit{glo:jtag}-Protokoll direkten Zugriff auf bestimmte Funktionalitäten eines Prozessors.

    Ein \acrshortit{glo:jtag}-komformes Gerät enthält ein \glsit{glo:ir}, \glsit{glo:dr} und
    ein \glsit{glo:tap}. Die beiden Register sind als Schieberegister
    (engl. ``shift register'') implementiert, d.h. der Speicherinhalt von aneinandergeketteten \glsplit{glo:flipflop} wird
    pro Takt ins nächste \glsit{glo:flipflop} verschoben.
    Als \glsit{glo:dr} können je nach Befehl verschiedene Register eingesetzt werden.

    Jeder Pin auf dem \glsit{glo:ic} enthält eine \acrshortit{glo:jtag}- bzw. Boundary-Scan-Zelle, die sowohl Daten vom Pin ``aufnehmen'' als auch
    auf den Pin schreiben kann. Aufgenommene, so wie geschriebene Daten werden seriell über die \acrshortit{glo:jtag}-Zellen durch den
    \glsit{glo:tap} verschoben\citep[]{Corelis}.

    Der \glsit{glo:tap} stellt die \acrshortit{glo:jtag}-Hardware-Schnittstelle dar und besteht aus den folgenden Anschlüssen
    \begin{description}
        \item[TDI] Die Testdaten werden über \texttt{Test Data Input} in die \acrshortit{glo:jtag}-Zellen geschoben
        \item[TCK] Das Signal-Interface \texttt{Test Clock} taktet die \acrshortit{glo:jtag}-State-Machine
        \item[TMS] Kontrolliert wird die \acrshortit{glo:jtag}-State-Machine (\glsit{glo:tap}-Controller) von dem \texttt{Test Mode Select}
        \item[TDO] Die Testdaten werden über das \texttt{Test Data Output} zurück gelesen
        \item[TRST] Das \texttt{Test Reset} ist optional und ermöglicht asynchrones Zurücksetzen des \glsit{glo:tap}-Controllers
    \end{description}

    \texttt{TDI}, \texttt{TCK}, \texttt{TMS} und \texttt{TRST} sind für den Prozessor also Output-Signale und
    \texttt{TDO} ein Input-Signal.
    Diese Pins sind üblicherweise dediziert, d.h. sie werden nicht für andere Zwecke genutzt.

    Im \glsit{glo:datasheet} der Test-Hardware \texttt{AT SAM E70 Q21}, stehen die unterstützten \acrshortit{glo:jtag}-Befehle. Dem Standard nach
    sind drei Befehle verpflichtend, sie müssen also vom entsprechenden Chip-Hersteller implementiert werden\footnote{
    \citep[]{Corelis}}:
    \begin{description}
    \item[\texttt{BYPASS}] Beim BYPASS wird das \textit{Bypass-Register} zwischen \texttt{TDI} und \texttt{TDO}
    geschaltet. Das Bypass-Register hält ein einzelnes Bit (engl. ``single bit register'') und umgeht das bei EXTEST
    verwendete längere Boundary-Scan-Register. Dieser Befehl wird genutzt, um Geräte bei einer Verkettung von
    \acrshortit{glo:jtag}-Geräten zu umgehen. Dabei bleibt das betroffene Gerät funktionsfähig.
    \item[\texttt{EXTEST}] Der EXTEST-Befehl wird verwendet, um die zusammengeschalteten Verbindungen zwischen den
    Geräten durchzuführen. Das Boundary-Scan-Register (eins der \glsit{glo:dr}) wird zwischen \texttt{TDI} und
    \texttt{TDO} geschaltet.
    Dabei wird das Gerät in den \textit{externen} Modus versetzt, d.h. Testdaten werden über die
    \acrshortit{glo:jtag}-Zellen auf die Pins geschoben. Gleichzeitig nehmen die \acrshortit{glo:jtag}-Zellen auch die von den Pins geschriebenen Daten
    auf. EXTEST ist der Hauptbefehl um Boundary-Scans durchzuführen.
    \item[\texttt{SAMPLE/PRELOAD}] SAMPLE ähnelt dem EXTEST\hyp{}Befehl, indem das Boundary\hyp{}Scan\hyp{}Register zwischen
    \texttt{TDI} und \texttt{TDO} geschaltet wird. Der Unterschied liegt darin, dass das Boundary-Scan-Gerät trotzdem
    funktionsfähig bleibt. Das Boundary-Scan-Register ist also zugreifbar, ohne dass das Gerät seine
    Funktion unterbrechen muss.
    \end{description}

    Der in Abbildung \ref{fig:tapcontroller} skizzierte \glsit{glo:tap}-Controller bezeichnet eine \glsit{glo:statemachine}
    mit 16 verschiedenen Zuständen.
    Diese \glsit{glo:statemachine} enthält Zustände, welche sowohl das \glsit{glo:dr} als auch das \glsit{glo:ir}
    aufnehmen und Daten auf besagte Register schieben können.
    \begin{figure}[h]
        \centering
        \caption{Endlicher Automat des TAP-Controllers}
        \source{
            \href{
                https://commons.wikimedia.org/w/index.php?curid=45147866}{Von Rudolph H 17:30, 25. Jul. 2009 (CEST)
                - Eigene Zeichung, Copyrighted free use
                }
        }
        \label{fig:tapcontroller}
        \includegraphics[scale=0.3]{../Graphiken/JTAG_TAP_Controller_State_Diagram.png}
    \end{figure}
    \acrshortit{glo:jtag} stellt also eine einheitliche Schnittstelle zur internen Logik des \glsit{glo:ic} dar. Somit können vom Chip-Hersteller
    implementierte Funktionalitäten, wie zum Beispiel das Setzen von Hardware-Breakpoints durch eine externe Instanz
    (sei es zusätzliche Hardware), genutzt werden.

    Eine von \glsit{glo:arm} entwickelte Alternative zu \acrshortit{glo:jtag} ist das im nächsten Abschnitt \ref{sec:swd} behandelte
    \acrshortit{glo:swd}, welches den ``Vorteil hat, dass es pinkompatibel zu \acrshortit{glo:jtag} ist, es aber erlaubt, softwaregesteuert
    zwischen \acrshortit{glo:jtag} und \acrshortit{glo:swd} zu wechseln''\citep[]{Asche2017}.

    Ein \acrshortit{glo:jtag}-Adapter ist ein externer Mikrocontroller mit spezieller Firmware, der vom Host-System gesendete
    Debug-Befehle (z.B. über \glsit{glo:usb}) in \acrshortit{glo:jtag}-Befehle umsetzt und direkt an das \acrshortit{glo:jtag}-Interface weiterleitet.
    Eine Alternative zu diesen externen Hardware-Debuggern ist ein auf dem \glsit{glo:pcb} integrierter \glsit{glo:ic} mit einer
    \glsit{glo:cmsisdap}-fähigen Firmware. Weiteres dazu in Abschnitt \ref{sec:cmsisdap}.
	\subsection{SWD}
	\label{sec:swd}
	Im Gegensatz zu dem im vorherigen Abschnitt \ref{sec:jtag} behandelten \acrshortit{glo:jtag}-Standard, ist
	\acrfullit{glo:swd} eine von \glsit{glo:arm} implementierte proprietäre Technologie, die \acrshortit{glo:jtag}-Funktionalität zu
	effizienteren Umständen verspricht.

	\acrshortit{glo:swd} benötigt im Vergleich zu \acrshortit{glo:jtag} nur 2 Pins für die Signal-Interfaces.
	\begin{description}
	    \item[\texttt{SWDIO}] Bidirektionaler Daten-Pin zum Schreiben und Lesen von Daten. Ersetzt die in \acrshortit{glo:jtag}
	    verwendeten \texttt{TDI}- und \texttt{TDO}-Kontakte, wobei der in \acrshortit{glo:jtag} als \texttt{TMS} bezeichnete Pin bei
	    \acrshortit{glo:swd} in den \texttt{SWDIO} umgewandelt wird \citep[vgl.~S.~6]{Booth}
	    \item[\texttt{SWCLK}] Ähnlich wie bei \acrshortit{glo:jtag} ein Taktungs-Pin zum Takten des Test-Controllers. Dieser ersetzt
	    den \acrshortit{glo:jtag}-\texttt{TCK}-Pin \citep[vgl.~S.~6]{Booth}
	\end{description}

    \acrshortit{glo:swd} nutzt ein nach \glsit{glo:arm} standartisiertes bidirektionales Protokoll, welches im \glsit{glo:adi} v5
    definiert ist, um Daten vom Debugger zum Target-System zu übertragen
    \footnote{\href{https://www.arm.com/products/processors/serial-wire-debug.php}
    {ARM SWD Produkt Seite, Zugriff: 15.07.2017}}.
    Das \glsit{glo:adi} ist eine allgemeine Abstraktion und Definierung eines standartisierten
    Debug-Interfaces für \glsit{glo:arm} Prozessoren. Version 5 des \glsit{glo:adi} ist jedoch darauf ausgelegt, nicht
    \glsit{glo:arm}-prozessorspezifisch zu sein, sondern eine weitere Bandbreite an Geräten zu unterstützen
    \citep[vgl.~Kap.~1]{ArmAdi}.
    \acrshortit{glo:swd} ist kompatibel mit allen \glsit{glo:arm} Prozessoren und anderen Prozessoren, welche \acrshortit{glo:jtag} zur Fehlersuche nutzen. Es verschafft
    außerdem Zugang zu den Debug-Registern der oben genanntent \glsit{glo:arm}-Cortex-Prozessorfamilien
    \footnote{\href{https://www.arm.com/products/processors/serial-wire-debug.php}
    {ARM SWD Produkt Seite, Zugriff: 15.07.2017}}.

    \subsection{ARM CMSIS}
    \label{sec:cmsis}
    Die \glsit{glo:cmsis}-Bibliothek wird von \glsit{glo:arm} entwickelt und stellt eine
    herstellerunabhängige Hardware-Abstraktion für \glsit{glo:arm}-Cortex-Architekturen dar. Es ist der Versuch eine einheitliche
    und konsistente Schnittstelle für den Zugriff auf die Hardware zu schaffen. ``Creation of software is a major cost
    factor in the embedded industry. Standardizing the software interfaces across all Cortex-M silicon vendor products,
    especially when creating new projects or migrating existing software to a new device, means significant cost
    reductions.''\footnote{\href{https://developer.arm.com/embedded/cmsis}{ARM cmsis Produktbeschreibung}}

    Die \glsit{glo:cmsis}-Bibliothek stellt eine Vielzahl an Funktionen für den Zugriff auf die Hardware auf unterschiedlichen
    Ebenen bereit.

    ``Prozessorhersteller (oder ggf. Drittfirmen oder die Public Domain) werden dazu ermutigt, diese Bibliotheken für
    ihre Produkte zu realisieren; gleichzeitig sollen Middlewarehersteller und Entwickler gerätetreiberseitig ebenfalls
    diese eingekapselten Aufrufe benutzen, so dass theoretisch von \glsit{glo:arm} „angestupst“ eine sehr große Interoperabilität
    zwischen Soft- und Middleware und verschiedenen ACPs besteht.''
    \footnote{ACP steht für \textit{ARM Cortex based processor}}
    \citep[Kap.~2.1]{Asche2017}

    \glsit{glo:cmsis} besteht zu heutigem Zeitpunkt aus verschiedenen Paketen auf verschiedenen Abstraktionsstufen.
    Das \glsit{glo:cmsis}-Core ist der Grundbaustein der \glsit{glo:cmsis}-Bibliothek und stellt Kernfunktionen für den Zugriff auf die
    Peripherie dar. Außerdem enthält es bestimmte intrinsische Funktionen.

    \textbf{Beispiel:} Um einen Interrupt zu aktivieren, kann die Funktion
    \begin{lstlisting}
void NVIC_EnableIRQ(IRQn_Type IRQn)
    \end{lstlisting}
    mit \texttt{IRQn} als Interrupt-Nummer verwendet werden.

    Für die Fehleranalyse sind vor Allen die beiden Pakete \glsit{glo:cmsis}-SVD und \glsit{glo:cmsisdap} relevant.
    Das \glsit{glo:cmsis}-SVD soll es Debuggern ermöglichen, eine genaue Übersicht über Peripherie zu liefern, ohne auf
    die tatsächliche Implementierung dieser achten zu müssen.

    Auf \glsit{glo:cmsisdap} wird in Abschnitt \ref{sec:cmsisdap} genauer eingegangen.

    \textit{\glsit{glo:cmsis}\hyp{}\glsit{glo:rtos}} stellt ein generisches \glsit{glo:rtos}\hyp{}Interface für \glsit{glo:arm}-Cortex basierte Geräte bereit.
    Es ist eine Schnittstelle für Software-Komponenten, welche \glsit{glo:rtos}-Funktionalität benötigen und ermöglicht den
    einheitlichen Zugriff auf \glsit{glo:rtos}\hyp{}Funktionalitäten verschiedener Echtzeitbetriebssysteme.

    \begin{figure}[h]
    	\caption{Die CMSIS Softwarepakete}
    	\source{
    	    \href{
    	        https://developer.arm.com/embedded/cmsis
    	    }{
    	        ARM CMSIS Produkt Seite
    	    }, Zugriff: 17.08.2017
    	}
    	\label{fig:cmsis_packages}
        \centering
        \includegraphics[scale=0.3]{../Graphiken/CMSIS_block_diagram.png}
    \end{figure}

    Eine Bibliothek für das Verarbeiten von Digitalen Signalen ist unter dem Namen \glsit{glo:cmsis}-DSP vorhanden und
    enthält viele auf Cortex-M optimierte Verarbeitungs- und Berechnungsfunktionen.

    Als letztes ist ein Peripherie-Treiber-Interface \glsit{glo:cmsis}-Driver für die Verwendung in Middleware und
    Nutzerapplikationen verfügbar. Es ist ein von \glsit{glo:arm} definiertes Interface, welches von unabhängigen
    Hardware-Herstellern implementiert werden kann. Code-Basen, die mit dem \glsit{glo:cmsis}-Driver-Interface arbeiten sind in der
    Theorie portabler als diejenigen, welche direkt die herstellerspezifischen Treiber
    nutzen. Da es sich hierbei lediglich um einen vorgeschlagenen Standard handelt, sind die Hardware-Hersteller nicht
    dazu verpflichtet diesen Standard einzuhalten bzw. zu implementieren.

	\subsection{CMSIS-DAP}
	\label{sec:cmsisdap}
	\acrfullit{glo:cmsisdap} ist ein Produkt in der \glsit{glo:cmsis}-Reihe (siehe Abschnitt \ref{sec:cmsis}) und bezeichnet die Umsetzung eines
	von \glsit{glo:arm} definierten Protokolls für den Zugriff auf die Debug-Funktionalitäten eines Prozessors.

	Der \glsit{glo:dap} ist eine Implementierung des in Abschnitt \ref{sec:swd} erwähnten \glsit{glo:adi}.
	Der \glsit{glo:dap} wird in \glsplit{glo:dp} und \glsplit{glo:ap} aufgeteilt.
	Die \glsplit{glo:dp} können von externen Debuggern genutzt werden, um den \glsit{glo:dap} zu erreichen, über die \glsplit{glo:ap} kann
	auf On-Chip-Ressourcen zugegriffen werden. \acrshortit{glo:jtag} und \acrshortit{glo:swd} nutzen \glsit{glo:dap} für die Kommunikation mit der entsprechenden
	Hardware.

	\glsit{glo:cmsisdap} läuft als Firmware auf einer Debug-Einheit, ein zweiter \glsit{glo:ic}, welcher sich auf dem \glsit{glo:pcb} befindet.
	Dieser Chip wird üblicherweise direkt per \glsit{glo:usb} mit einem Host-System verbunden. Über die Debug-Einheit können
	mit Hilfe des \glsit{glo:cmsisdap}-Protokolls auf direktem Weg \acrshortit{glo:jtag}-Befehle an die Ziel-Hardware gesendet werden. Es stellt
	also ein unmittelbares On-Chip-Debug-Interface dar, welches die Verwendung von externen Hardware-Debuggern obsolet
	macht. Falls unterstützt, kann auch \acrshortit{glo:swd} genutzt werden, um mit dem Prozessor zu
	kommunizieren (siehe Abschnitt \ref{sec:swd}).

	\begin{figure}[h]
	    \centering
	    \caption{CMSIS-DAP Übersicht}
	    \source{\citep[Übersicht]{CMSISDAP}}
	    \label{fig:cmsisdapoverview}
	    \includegraphics[scale=0.7]{../Graphiken/cmsis_dap}
	\end{figure}

	Das \glsit{glo:cmsisdap}-Protokoll liefert Befehle zum Kommunizieren mit der Ziel-Hardware. Nebenbei können auch bestimmte
	Informationen über die Debug-Einheit abgerufen werden. Listing \ref{cmsisdapled} zeigt beispielhaft einen
	Kommunikationsaufruf mit einer auf dem \textit{Evaluation Board} von Atmel verbauten Debug-Einheit.

	Das \glsit{glo:cmsisdap}-Protokoll definiert Befehle, die über die entsprechende Befehls-ID ausgeführt werden.
	In Listing \ref{cmsisdapled} lautet die Befehls-ID für den LED-Befehl (\texttt{dap\_led}) \texttt{0x01} und wird in
	\texttt{buf[1]} geschrieben.
	Die Felder \texttt{buf[2]} und \texttt{buf[3]} enthalten die entsprechenden Parameter für den LED-Befehl.
	\texttt{buf[0]} ist die \texttt{report id} für das \acrshortit{glo:hid}-Protokoll und hat nichts mit \glsit{glo:cmsisdap} zu tun.

	\newpage

	\begin{lstlisting}[
	    language=Rust,
	    caption=CMSIS-DAP-Befehl um die LEDs auf der Debug-Einheit zu bedienen (Rust-Code),
	    label=cmsisdapled,
	    float,
	    floatplacement=H
	]
// HELPER Send LED Command to CMSIS_DAP Dev --> write
fn dap_led(dev: &hidapi::HidDevice, led: u8, output: u8) {
    let mut buf: [u8; 512 + 1] = [0xff; 512 + 1];
    buf[0] = 0x00;      // report id
    buf[1] = 0x01;      // dap_led
    buf[2] = led;       // LED: specifies the LED
                        // that is addressed: 0 = Connect LED;
                        // 1 = Running LED.
    buf[3] = output;    // Output: specifices the status of
                        // the LED after the command: 1 = LED on;
                        // 0 = LED off.

    println!("writing to device.");

    let res = dev.write(&buf).unwrap();

    println!("wrote {:?} Byte(s).", res);
}
	\end{lstlisting}

	Verdeutlicht werden kann die Verwendung von \glsit{glo:cmsisdap} in Kombination mit \acrshortit{glo:jtag} und \acrshortit{glo:swd} an einem Beispiel. Hierzu
	wird zunächst der Quelltext des \glsit{glo:openocd} genauer untersucht.
	Das folgende Beispiel zeigt eine Kette von Aufrufen der verschiedenen Programmmodule innerhalb des \glsit{glo:openocd}.

	Es wird angenommen, dass das Host-System mit einer Debugging-Software einen Breakpoint in der auf dem Prozessor
	laufenden Firmware setzen möchte (Genaueres zu Debugging in Abschnitt \ref{sec:debugging}).
	Der Fokus bei dem Beispiel liegt auf der letzten Instanz, in der der Breakpoint Befehl vom \glsit{glo:cmsisdap}-Programmmodul
	empfangen und schließlich per \glsit{glo:usb} an die Debug-Einheit gesendet wird.

    \begin{enumerate}
    \item \texttt{gdb\_server.c}
    \begin{itemize}
        \item Empfängt Breakpoint-Befehl von Debugging-Software (zum Beispiel eine \glsit{glo:ide} oder lokaler \glsit{glo:gdb})
        \item Übergibt den Breakpoint-Befehl weiter an ein allgemeines Hardware-Interface \texttt{target.c}
        \begin{lstlisting}
static int gdb_write_memory_binary_packet(struct connection *connection, char const *packet, int packet_size){...}
        \end{lstlisting}
    \end{itemize}
     \item \texttt{target.c}
     \begin{itemize}
        \item Interface\hyp{}ähnliche Programmlogik sucht die korrekte Prozessorfamilie heraus und leitet den Befehl weiter
        an die konkrete Hardware\hyp{}Abstraktion
        \begin{lstlisting}
static int target_write_buffer_default(struct target *target, uint32_t address, uint32_t count, const uint8_t *buffer){...}
        \end{lstlisting}
     \end{itemize}
     \item \texttt{cortex\_m.c}
     \begin{itemize}
        \item Berücksichtigt Besonderheiten, die bei der Cortex-M-Prozessorfamilie auftreten. Hier sind die
        Debug-spezifischen Methoden enthalten, die über den Befehl
        \begin{lstlisting}
static int cortex_m_write_memory(struct target *target, uint32_t address, uint32_t size, uint32_t count, const uint8_t *buffer) {...}
        \end{lstlisting}
        weiter an das Debug-Interface gereicht werden
     \end{itemize}
     \item \texttt{arm\_adi\_v5.h/.c}
     \begin{itemize}
        \item Enthält Methoden zum Ausführen der im \glsit{glo:adi} v5 (siehe Abschnitt \ref{sec:swd}) von \glsit{glo:arm}
        definierten Befehle
        \item In diesem Fall ist es der Schreibbefehl für den Speicherpuffer, welcher über die Methode
        \begin{lstlisting}
int mem_ap_sel_write_buf(struct adiv5_dap *swjdp, uint8_t ap, const uint8_t *buffer, uint32_t size, uint32_t count, uint32_t address) {...}
        \end{lstlisting}
        und anschließend
        \begin{lstlisting}
static inline int dap_queue_ap_write(struct adiv5_dap *dap, unsigned reg, uint32_t data){...}
        \end{lstlisting}
        den Breakpoint-Befehl schreibt
     \end{itemize}
     \item \texttt{adi\_v5\_swd.c}
        \begin{itemize}
        \item Der Breakpoint-Befehl wird in das \acrshortit{glo:swd}-Protokoll überführt
        \begin{lstlisting}
static int swd_queue_ap_write(struct adiv5_dap *dap, unsigned reg, uint32_t data){...}
        \end{lstlisting}
     \end{itemize}
     \item \texttt{cmsis\_dap\_usb.c}
     \begin{itemize}
        \item Auf Software-Ebene ist dies die letzte Schicht, die den Breakpoint-Befehl endgültig an die Ziel-Hardware
        sendet
        \item Hier wird zunächst das Kommando in das \glsit{glo:cmsisdap}-Protokoll überführt und in eine Warteschlange geschrieben
        \item Über das \glsit{glo:cmsisdap}-Kommando \texttt{CMSIS\_DAP\_TFER}, welches die definierte ID \texttt{0x05} besitzt,
        wird nun der Breakpoint-Befehl per \glsit{glo:usb} an den Prozessor gesendet
        \item Der Transfer-Befehl ähnelt dem oben gezeigten Code-Beispiel \ref{cmsisdapled} mit dem Unterschied, dass
        bei Listing \ref{cmsisqueue} eine gesamte Warteschlange an Befehlen hintereinander abgearbeitet wird.
        Dieser sieht (in gekürzter Fassung) wie folgt aus:
        \begin{lstlisting}[caption={OpenOCD, \texttt{cmsis\_dap\_usb.c}}, label=cmsisqueue]
static int cmsis_dap_swd_run_queue(struct adiv5_dap *dap)
{
    uint8_t *buffer = cmsis_dap_handle->packet_buffer;
    size_t idx = 0;
    buffer[idx++] = 0;	/* report number */
    buffer[idx++] = CMD_DAP_TFER;
    buffer[idx++] = 0x00;	/* DAP Index */
    buffer[idx++] = pending_transfer_count;

    for (int i = 0; i < pending_transfer_count; i++) {
        uint8_t cmd = pending_transfers[i].cmd;
        uint32_t data = pending_transfers[i].data;

        buffer[idx++] = (cmd >> 1) & 0x0f;
        if (!(cmd & SWD_CMD_RnW)) {
        	buffer[idx++] = (data) & 0xff;
        	buffer[idx++] = (data >> 8) & 0xff;
        	buffer[idx++] = (data >> 16) & 0xff;
        	buffer[idx++] = (data >> 24) & 0xff;
        }
    }

    queued_retval = cmsis_dap_usb_xfer(cmsis_dap_handle, idx);
}
    \end{lstlisting}
    \end{itemize}
    \end{enumerate}

    \subsection{Hardware Abstraction Layer}
    Der \glsit{glo:hal} spielt mit der wachsenden Anzahl an verschiedenen Variationen von Hardware
    eine immer größer werdende Rolle in der Entwicklung von Embedded-Systemen. Der \glsit{glo:hal} stellt als weitere
    Abstraktionsschicht ein Bindeglied zwischen Applikation und Hardware dar. Diese verwenden in der Regel das
    \glsit{glo:cmsis}-Core als unterste Abstraktionsschicht und bauen darauf auf.

    Chip-Hersteller bieten ihren eigenen \glsit{glo:hal} an, um einen einfacheren Zugriff auf die
    entsprechende Peripherie eines \glsit{glo:pcb}s zu ermöglichen. Beispielsweise stellt Atmel das
    \glsit{glo:asf} zur Verfügung.

    \subsection{Applikationsentwicklung mit Abstraktionsschichten}
    \label{sec:abstractionoverview}
    Die vorgestellten Technologien ergeben, wie das Code-Beispiel in Abschnitt \ref{sec:cmsisdap} verdeutlichen soll,
    in Kombination miteinander ein Schichtenmodell indem Hardware-Schnittstellen und Protokolle aufeinander aufbauen.

    Tabelle \ref{tab:hwabstractionlayer} zeigt eine vereinfachte Strukturierung der verschiedenen Schichten. Es ist
    anzumerken, dass \acrshortit{glo:swd} oder \glsit{glo:cmsis}-Driver nicht zwangsläufig verwendet werden. Die Tabelle dient lediglich der
    Verdeutlichung der Struktur der beschriebenen Technologien.

    Prinzipiell kann ein Entwickler seine Anwendung auch ohne jegliche Abstraktionen direkt auf der Hardware
    (``bare metal'') entwickeln. Dies hat aber zu Folge, dass der Aufwand um ein Vielfaches gesteigert wird und
    möglicherweise bei verschiedenen Applikationen viel redundante Arbeit entsteht. Einfache Zugriffe auf
    bestimmte Funktionalitäten müssen für jede neue Applikation neu implementiert werden.

    \begin{table}[h]
        \centering
        \caption{Abstraktionsschichten bei der Entwicklung von Embedded-Systemen}
        \label{tab:hwabstractionlayer}
        \begin{tabular}{| c | C{7cm} |}
        \hline
        \textbf{6.} & Applikation                                           \\ \hline
        \textbf{5.} & \glsit{glo:cmsis}-Driver / herstellerunabhängige Schnittstelle   \\ \hline
        \textbf{4.} & \glsit{glo:hal} / Treiber / Betriebssystem                        \\ \hline
        \textbf{3.} & \glsit{glo:cmsis}-Core und \glsit{glo:cmsisdap}                              \\ \hline
        \textbf{2.} & \acrshortit{glo:swd}                                                   \\ \hline
        \textbf{1.} & \acrshortit{glo:jtag}                                                  \\ \hline
        \end{tabular}
    \end{table}

    Da das \acrshortit{glo:jtag}-Protokoll der einzige Standard in dem Bereich ist, bauen viele Technologien und Abstraktionen darauf
    auf. Dies führt dazu, dass es auf vielen Prozessoren verbaut und unterstützt wird.

    \acrshortit{glo:swd} und \glsit{glo:cmsis} sind die von \glsit{glo:arm} entwickelten Technologien und sind deshalb, wegen der weiten Verbreitung an \glsit{glo:arm}
    Prozessorarchitekturen bei Embedded-Systemen, sehr geläufig.

    Die von den jeweiligen Hardware\hyp{}Herstellern bereitgestellten Hardware\hyp{}Abstraktionen können vom Entwickler für eine
    Applikation verwendet werden. Ein fehlender Standard bei diesen Abstraktionen macht eine Applikation in vielen
    Fällen schwer portierbar.

    Alternativ dazu würde auf selber Ebene ein Betriebssystem liegen. Auch das Betriebssystem stellt eine
    Hardware-Abstraktion dar und ist von der verwendeten Prozessorarchitektur abhängig.
    \\

    Man betrachtet den Entwicklungszyklus einer Hardware-Applikation und der in Abschnitt \ref{sec:layers} erwähnten
    unterschiedlichen Verwendung für die Schichten und stellt folgende Problematiken fest:

    Das erste Problem ist ein fehlender Standard der vom Hersteller implementierten Hardware-Abstraktionen.
    Erneut bahnt sich hier das Problem an, an die Software eines bestimmten Herstellers und damit auch potenziell
    an qualitativ geringere Software gebunden zu sein. Ein fehlender Standard des \glsit{glo:hal}s verschlechtert die Portierbarkeit
    einer Code-Basis und der Entwickler wird bei einem Wechsel der Hardware mit redundanter Arbeit belastet.
    Abschnitt \ref{sec:overallsolution} beschäftigt sich im ersten Teil mit einem Lösungsansatz für dieses Problem.

    Die zweite Problematik ist der eigentliche Entwicklungsprozess einer Hardware-Applikation, welche das Aufspielen
    und Debuggen von Applikationen, basierend auf den \acrshortit{glo:jtag}-/\acrshortit{glo:swd}-Protokollen, beinhaltet. Ein Lösungsansatz für dieses
    Problem wird im zweiten Teil des Abschnitts \ref{sec:overallsolution} vorgestellt und dokumentiert.

	\section{Debugging}
\label{sec:debugging}
In diesem Kapitel sollen zunächst die Grundprinzipien der Fehlersuche in Firmware-Programmen erläutert werden. Im
Vordergrund stehen hier die in der \glsit{glo:arm}-Cortex-M-Familie unterstützten Debugging-Features, das durch
\acrshortit{glo:jtag} abgelöste \glsit{glo:ice}, sowie ein Einblick in die Debugging-Technologien auf Host-Seite. Es soll
weiterhin auf den für die Fehlersuche in Programmen für Embedded-Systeme relevanten Unterschiede zwischen Soft- und
Hardware Breakpoints eingegangen werden.
Anschließend folgt eine Ausführung der Funktionsweise von vorhandenen Host-seitigen Debugging-Lösungen.
	\subsection{Debugging-Features}
	\label{sec:debugfeatures}
	Debugging ist für die meisten Software-Entwickler ein fester Bestandteil des Entwicklungsprozesses. Jede von
	Menschen geschriebene Software enthält vorhersehbare und nicht vorhersehbare Fehlerfälle, welche idealerweise
	im Vorhinein gefunden und korrigiert werden können bzw. müssen. Debugger sind ein mächtiges Werkzeug, das dem
	Entwickler bei der Fehlersuche behilflich ist.
	Verschiedene Debugger stellen verschiedene Funktionalitäten und Methoden zur Fehlerfindung bereit. Die klassischen
    Debugging-Methoden, welche von \glsit{glo:arm}-Cortex-M-Prozessoren unterstützt werden, sind im Folgenden aufgelistet
    \begin{description}
        \item[Breakpoint] Breakpoints sind prinzipiell Punkte innerhalb eines Programms, an denen die Ausführung
        des Programms angehalten wird. Dabei lassen sich wichtige Unterscheidungen treffen:
        \begin{description}
            \item[Software Breakpoints] Die für den Software\hyp{}Entwickler bekannte Art von Breakpoints sind
            Software\hyp{}Breakpoints. Diese können bei einer gewünschten Anweisung im Programm\hyp{}Code platziert werden.
            Der Debugger ersetzt die Anweisung (oder je nach Architektur nur einen Teil der Anweisung) mit einer
            speziellen Breakpoint-Anweisung. Bei der Cortex-M-Prozessorfamilie werden Software Breakpoints mit der
            Anweisung \texttt{BKPT} gesetzt. Die Anzahl der Software Breakpoints ist in der Theorie unbegrenzt (logische
            Begrenzung durch den Speicher). Wird die Anweisung \texttt{BKPT} ausgeführt, generiert der Prozessor
            ein Debug-Event, auf welches den Umständen entsprechend reagiert werden kann
            \citep[vgl. Kap. 14.5]{Yiu2013}.
            Üblicherweise hält der Prozessor die Ausführung des Programms an und der Entwickler kann mit Hilfe des
            Debuggers den Speicherinhalt auslesen und ggf. verändern.
            \item[Hardware Breakpoints] Hardware\hyp{}Breakpoints sind für viele Software\hyp{}Entwickler zunächst unintuitiv.
            Ein Hardware Breakpoint enthält eine Zuordnungslogik für Adressen. Sie beobachten einen internen
            Bus oder Befehlszähler und generieren bei einer Übereinstimmung ein Debug-Event\\\citep[]{Styger}.

            Bei der Cortex-M-Prozessorfamilie steuert die \glsit{glo:fpb} die
            Zuordnungslogik, die selbst bei temporär unveränderbarem Speicher (z.B. Flash-Speicher) Debug-Events
            generieren kann\citep[vgl.~Kap.~14.5]{Yiu2013}. Hardware Breakpoints sind allerdings limitiert. Bei
            Cortex-M-Prozessoren können bis zu acht Breakpoints gesetzt werden, von denen sechs für Befehlsadressen sind.
        \end{description}
        \item[Single Stepping]
        \textit{Single Stepping} bezeichnet die Ausführung eines einzelnen Schrittes des Programms. Ein \textit{Schritt}
        bedeutet in diesem Fall die Ausführung einer Code-Zeile oder eines Maschinenbefehls
        \footnote{\href{https://sourceware.org/gdb/onlinedocs/gdb/Continuing-and-Stepping.html}
        {GDB Dokumentation}, Zugriff: 19.07.2017}.
        Die Ausführung eines Programms kann durch ein \textit{Continue} auch normal weiter geführt werden.
        Üblicherweise ist in Debugging-Software auch ein \textit{Step-Over}-Befehl verfügbar, der es ermöglicht
        verschachtelte Aufrufe als einen einzigen Schritt auszuführen.
        \textit{Single Stepping} wird bei vielen Architekturen durch temporäre Software Breakpoints realisiert.
        \item[Watchpoint] \textit{Watchpoints} spielen eine wichtige Rolle in der Fehlerfindung von
        \\Hardware-Applikationen. Sie ermöglichen das Anhalten des Programms, sobald sich der Wert eines Ausdrucks ändert.
        Als Ausdrücke werden jegliche semantische Konstrukte definiert, die einen Wert liefern.
        Watchpoints werden auch \textit{Data Breakpoints} genannt.

        Prozessoren der Cortex\hyp{}M\hyp{}Familie enthalten die \glsit{glo:dwt}, welche für die
        Verwaltung von Watchpoints zuständig ist. ``Es können Breakpoints funktional identisch zum Verhalten der
        \texttt{bkpt} Anweisung [...] ausgelöst werden, wenn eine definierbare Operation [...] auf einem definierbaren Speicherbereich auf dem Adressbus erkannt
        wird''\citep[vgl. Kap. 10.1.3]{Asche2017}.

        Außerdem können bestimmte Restriktionen auf den Watchpoint gesetzt werden, so dass dieser nur bei bestimmten
        Werten oder anderen definierten Umständen ein Debug-Ereignis auslöst.
        \item[On the fly memory access] \acrshortit{glo:jtag} ermöglicht es dem Debugger während der Ausführung eines Programms
        Speicherinhalte zu verändern ohne das Programm zu stopppen.
    \end{description}
	\subsection{GDB}
	\label{sec:gdb}
	Eine der gängigsten Debugging\hyp{}Softwares ist der frei verfügbare \glsit{glo:gdb}. Die Debugging\hyp{}Software
	wird für die Fehlersuche in Programmen vieler Programmiersprachen eingesetzt und wird von vielen
	Entwicklungsumgebungen als Debugger-Logik verwendet. Dabei wird dem Nutzer oft eine graphische Bedienung durch eine
	\glsit{glo:ide} zur Verfügung gestellt.\\

	Der \glsit{glo:gdb} unterstützt mit den folgenden vier Grundfunktionalitäten den Anwendungsentwickler bei der Fehlersuche in
	seinem Programm
	\footnote{\href{https://sourceware.org/gdb/}{GDB Produkt Seite}}:
	\begin{itemize}
	    \item Starten des Programms unter verschiedenen Umständen und Konfigurationen
	    \item Ausführung des Programms bei definierten Bedingungen stoppen
	    \item Zustand des Programms bei gestoppten Zustand analysieren
	    \item Speicherinhalte verändern, um die Ausführung des Programms zu beeinflussen
	\end{itemize}

	Außerdem definiert \glsit{glo:gdb} ein \glsit{glo:rsp}, welches die Fehlersuche auf einem externen Gerät
	ermöglicht. Dabei sendet eine auf dem Host-System laufende Software (zum Beispiel eine \glsit{glo:ide}) die vom Nutzer
	eingegebenen Befehle über einen Kommunikationsweg (zum Beispiel \glsit{glo:tcp}) an ein Target-System. Dort werden die
	Befehle von einer Debugging-Software empfangen und ausgeführt.

	\subsubsection{GDB-Remote-Server-Protocol-Beispiel}
	\textbf{Beispiel:} Der Nutzer setzt einen Breakpoint im Programm-Code. Übermittelt wird die Information über das
	definierte Protokoll als Nachricht in der Form:\\
	\begin{Verbatim}[frame=single]
Z<TYPE><ADDR><LEN>
    \end{Verbatim}

    \texttt{Z} entspricht der in dem \glsit{glo:rsp} definierten Anweisung zum Setzen von Breakpoints.
    Zum Löschen eines Breakpoints lautet der Anfang der Nachricht \texttt{z}. In das Feld \texttt{TYPE} wird ein Wert
    $0$ oder $1$ geschrieben, wobei $0$ einen Software Breakpoint und $1$ einen Hardware Breakpoint setzt. Das darauf folgende
    Feld \texttt{ADDR} gibt die Adresse an der der Breakpoint gesetzt werden soll an. Schließlich wird in das Feld
    \texttt{LEN} bei Software Breakpoints die Länge der zu ersetzenden Anweisung (also die Anweisung hinter dem
    Breakpoint) und bei Hardware Breakpoints die zu betrachtende Speicherstelle geschrieben
    \footnote{\href{http://davis.lbl.gov/Manuals/GDB/gdb_31.html}{Dokumentation des GDB RSP}}.

    \begin{Verbatim}[frame=single]
Z011504
    \end{Verbatim}

    Das oben aufgeführte Beispiel zeigt den Inhalt einer Nachricht, in der ein Software\hyp{}Breakpoint (\texttt{Z0}) an der
    Adresse 0x1150 (\texttt{1150}) mit der Länge der zu ersetzenden Anweisung (\texttt{4}) gesetzt wird.

    Der \glsit{glo:gdb}-Server muss auf die Anweisung entweder mit einem \texttt{Enn}, wobei \texttt{nn} für eine Error-Nummer
    steht, oder einem \texttt{Ok} antworten.

    \subsection{In-Circuit-Emulation}
    \label{sec:ice}
        \glsit{glo:ice} ermöglicht einem Entwickler Firmware durch die Emulation von Hardware auf Fehler zu untersuchen. Dabei
        handelt es sich um \textit{Non Intrusive Debugging}. Als  nicht intrusiv werden alle Debugging-Techniken
        bezeichnet, die weder die Größe des Firmware-Codes noch die Schnelligkeit der Ausführung dieses Codes beeinflussen.\\
        \textbf{Beispiel:} Das klassische \texttt{printf()}-Debugging beeinflusst beide genannten Aspekte und ändert
        durch die zusätzlichen Anweisungen außerdem den Programmverlauf. Somit ist dieses Verfahren intrusiv/störend.
        \\

        Da Speicher in einem Embedded\hyp{}System oft ein kritischer Punkt ist, kann ein durch Debug\hyp{}Anweisungen vergrößerter
        Speicherbedarf ein Problem darstellen.
        Ein weiterer kritischer Punkt sind die häufig an ein Embedded-System gestellten Echtzeitanforderungen, die
        durch eine erhöhte Laufzeit bedroht werden.

        Ein In-Circuit-Emulator ist eine auf Debugging spezialisierte Hardware, welche den Zielprozessor ersetzt.
        Der Emulator wird dann an den Host-PC angeschlossen und kann über Debugging-Software benutzt werden. Dies
        ermöglicht einen ungestörten Einblick in den Programmablauf
        \citep[vgl.~Kap.~1.1]{Rath2005}.

        Moderne \glsplit{glo:ice} nutzen keinen eigenen verbauten Prozessor um die eigentliche Hardware zu emulieren, sondern greifen
        direkt auf den Zielprozessor mit Hilfe des \acrshortit{glo:jtag}-Standards zu. Hierbei spricht man auch von
        \textit{On Chip Debugging}.

	\subsection{Tracing}
	\label{sec:trace}
	Wie bereits in Abschnitt \ref{sec:requirements} angesprochen, bezeichnet Tracing eine passive Analyse des Programms.
	Mit Hilfe von Werkzeugen können ausgegebene Tracing\hyp{}Informationen, ``also Programmzähler und sowie Inhalte der durch
	sie manipulierten Speicherstellen''\citep[vgl.~Kap.~10.1.6]{Asche2017}, analysiert werden. Hierbei werden lediglich
	Sprungbefehle, also nicht sequentiell aufeinanderfolgende Befehle, aufgeführt.

	Tracing birgt einige Vorteile gegenüber anderen Debugging-Techniken.

	\begin{itemize}
	    \item Es hat keinen relevanten Einfluss auf die tatsächliche Laufzeit eines Programms.
	    \item Inkonsistentes Verhalten eines Programms kann nach auftreten direkt analysiert werden. Ein aufgetretener
	    Fehler, welcher sich nicht einfach reproduzieren lässt, kann durch eine post-portem-Analyse gefunden werden.
	    \item Zeitinformationen (wie lange die Ausführung eines Programms in einem bestimmten Programmabschnitt verweilt)
	    werden in Tracing-Informationen mitgeliefert. Somit kann eine umfassendere Leistungsanalyse durchgeführt werden.
	\end{itemize}

	Der große Nachteil bei Tracing liegt bei der Hardware-Anforderung. Tracing-Hardware benötigt zusätzlichen Platz auf dem
	\glsit{glo:soc}. Es bewirkt außerdem einen höheren Stromverbrauch\citep[vgl.~S.~3]{Campbell2014}.
	``bis zu sieben zusätzliche Pins müssen dafür beschaltet werden. Diese Pins sind normalerweise mit anderen
	Funktionalitäten gemultiplext, woraus folgt, dass diese anderen Funktionalitäten während des Tracens nicht zur
	Verfügung stehen''\citep[vgl.~Kap.~10.1.6]{Asche2017}.

	Tracing wird oft in Kombination mit einer externen Hardware verwendet, die entweder speziell auf Tracing ausgelegt
	ist oder einer normalen \acrshortit{glo:jtag}-Debug-Probe.

	\subsection{OpenOCD/PyOCD}
	\label{sec:openOCD}
	\glsit{glo:openocd} \footnote{\href{http://openocd.org/}{OpenOCD Webseite \texttt{openocd.org}}}
	und PyOCD \footnote{\href{https://github.com/mbedmicro/pyOCD}{PyOCD auf github \texttt{github.com/mbedmicro/pyOCD}}}
	sind zwei Software-Lösungen zum Debuggen von Firmware\hyp{}Programmen auf \glsit{glo:arm}-Cortex-Prozessoren.
	Das primäre Ziel beider Programme ist es, mit Hilfe einer
	verbauten Debug-Einheit und dem \glsit{glo:cmsisdap}-Protokoll, ein Firmware-Programm auf den Speicher spielen zu können und
	entsprechend auf Fehler zu untersuchen. Da in beiden Fällen das \glsit{glo:rsp} verwendet wird, können auf Host-Seite
	\glsit{glo:gdb}-Clients benutzt werden um das Programm auf Fehler zu untersuchen.

	\begin{figure}
        \centering
        \caption{OpenOCD-Module}
        \label{fig:openocdmodules}
        \source{\citep[S.~40~Kap.~6.1]{Rath2005}}
        \includegraphics[scale=0.25]{../Graphiken/openocd_modules}
    \end{figure}

	Abbildung \ref{fig:openocdmodules} zeigt eine abstrakte modulare Aufteilung des \glsit{glo:openocd}-Programms. Es geht daraus
	hervor, dass das Kernstück des Programms der \texttt{Daemon} (Hintergrunddienst) ist. Dieser dient als Vermittler zwischen Host- und
	Target-Seite. Die Module \texttt{Target} und \texttt{Flash} sind Abstraktionen, welche für die Kommunikation bzw.
	Datenübertragung genutzt werden und prozessorspezifische Informationen und Konfigurationen enthalten.

	Auf Hostseite ist ein \texttt{\glsit{glo:gdb}}-Modul vorhanden, welches vom \glsit{glo:cli}
	zum Aufruf von Debugging-Befehlen verwendet werden kann, aber ebenso auch ein direktes Interface zu einem
	\glsit{glo:gdb}-Frontend, also zum Beispiel eine \glsit{glo:ide}, bereitstellt.
	Über das \glsit{glo:cli} kann das Firmware-Programm durch einen
	von den \texttt{Flash}- oder \texttt{Target}-Modulen zur Verfügung gestellten Befehle auf den Prozessor gespielt werden.

	Auf Target-Seite (abstrahiert im Programm) ist ein \texttt{JTAG}-Modul vorhanden, welches für die Umsetzung der
	\acrshortit{glo:jtag}-Befehle zuständig ist. Alle vom Daemon empfangenen Befehle werden in das \acrshortit{glo:jtag}-spezifische Protokoll übertragen
	und an die Ziel-Hardware gesendet. Das \texttt{JTAG}-Modul enthält außerdem die Umsetzung des
	\glsit{glo:cmsisdap}-Protokolls für die Übertragung über \glsit{glo:usb}.

	Grundsätzlich ist eine Entwicklungsumgebung mit \glsit{glo:openocd} folgendermaßen aufgestellt:

	\begin{itemize}
	    \item \glsit{glo:openocd} Daemon wird gestartet
	    \begin{itemize}
	        \item ein lokaler \glsit{glo:gdb}-Server wird gestartet
	        \item ein lokaler \glsit{glo:telnet}-Server wird gestartet
	        \item ein lokaler \glsit{glo:tcl}-Server wird gestartet
	    \end{itemize}
	    \item Der Entwickler kann sein kompiliertes Firmware-Programm über das \glsit{glo:cli} auf den Chip herunterladen
	    \item Der Entwickler kann Breakpoints oder andere Debugging-Anweisungen über einen \glsit{glo:gdb}-Client
	    (z.B. \glsit{glo:ide} oder \glsit{glo:cli}), welcher mit dem \glsit{glo:gdb}-Server kommunizieren muss, an den Prozessor senden
	    \item Der Daemon empfängt alle Befehle und ``verpackt'' sie in die entsprechenden Protokolle
	    (\acrshortit{glo:jtag}/\acrshortit{glo:swd} und anschließend \glsit{glo:cmsisdap})
	\end{itemize}

	\glsit{glo:openocd} ist neben PyOCD eine der wenigen frei verfügbaren Softwares für On-Chip-Debugging und ist
	deshalb sehr weit verbreitet.

	\section{Gesamtlösung}
\label{sec:overallsolution}
    Dieser Teil der Arbeit beschäftigt sich mit einem konkreten Lösungsansatz zur Vereinfachung der Entwicklung von
    Embedded-Systemen. Dabei skizziert der erste Teil Grundprinzipien und das Gesamtkonzept der Software-Architektur,
    dessen Implementierung nicht im Rahmen dieser Arbeit liegt. Der zweite Teil stellt das Software-Projekt für ein
    \glsit{glo:ide}-Plugin vor, welches im Rahmen der Arbeit implementiert wird, und dokumentiert den Entwicklungsprozess.
    \subsection{Gesamtlösung}

    Um den Entwicklungsprozess einer Hardware-Applikation zu vereinfachen, sind die in Abschnitt
    \ref{sec:abstractionoverview} aufgeführten Problematiken bezüglich der Hardware-Abstraktionen und der daraus
    entstehende umständliche Entwicklungsprozess zu beachten.

    Zugrunde liegt ein Konzept für eine Plattform zur Hardware-Entwicklung, welches aus zwei grundlegenden
    Komponenten besteht.

    Die erste Komponente ist eine modulare Hardware basierend auf dem Cortex-M7-Mikrocontroller. Diese wird in der
    Ausarbeitung nicht weiter beschrieben, da ein Prototyp für die Hardware noch nicht existiert und die Beschreibung den
    Rahmen der Arbeit sprengen würde.

    Die zweite Komponente ist eine Software-seitige Lösung zur Programmierung der Hardware-Applikation. Sie besteht aus
    mehreren Subkomponenten. Abbildung \ref{fig:softwarecomp} zeigt das Gesamtbild des Architekturkonzepts.
    Das primäre Ziel der Software-Komponenten ist es, eine einheitliche und einfach zu erweiternde Sammlung von
    Schnittstellen für Peripherie auf dem \glsit{glo:pcb} zu schaffen.

    Darauf aufbauend soll ein \glsit{glo:ui} (in weiteren Ausführungen \textit{Designer\hyp{}UI} genannt) geschaffen werden,
    mit der die Peripherie graphisch konfiguriert werden kann.
    Anschließend soll es möglich sein, die entwickelte Firmware (ob nun mit Designer-\glsit{glo:ui} oder selber
    programmiert) auf einem Host-System unabhängig von Plattform und Entwicklungsumgebung auf den Mikrocontroller zu
    spielen und auf Fehler zu untersuchen.

    \begin{figure}[h]
        \centering
        \caption{Konzipiertes Gesamtbild der Systemarchitektur}
        \source{eigene Zeichnung}
        \label{fig:softwarecomp}
        \includegraphics[scale=1.3]{../Graphiken/arch}
    \end{figure}

    \subsection{Anforderungen}
        \label{sec:demands}
        Die Anforderungen an eine Lösung zur Vereinfachung der Entwicklung bei Embedded\hyp{}Systemen ergeben sich aus den
        in Abschnitt \ref{sec:probleme} aufgeführten Probleme im Entwicklungsprozess.
        Die zu erfüllenden Anforderungen würden somit lauten:
        \begin{itemize}
            \item Ein Entwickler sollte nicht mehr an die vom Hersteller zur Verfügung gestellten Treiber und
            Entwicklungswerkzeuge gebunden sein.
            \item Ein unerfahrener Hardware-Entwickler sollte ohne großen Aufwand in der Lage sein, mit Hilfe der Software
            eine Applikation zu entwickeln und die auf dem \glsit{glo:pcb} vorhandene Peripherie miteinbeziehen.
            Dies heißt nicht, dass sich ein Entwickler nicht mit der Thematik auseinander setzen muss. Lediglich der
            durch Hardware-Hersteller erzeugte Überfluss an Schnittstellen, Konfigurationsmöglichkeiten und Treiberimplementierungen
            soll durch eine klar definierte, einheitliche, modulare und überschaubare Software mit integriertem
            \glsit{glo:api} ersetzt werden.
            \item Durch eine Vereinheitlichung soll die steile Lernkurve bei Embedded-Systemen
            abgeflacht werden und der Fokus des Entwicklers auf die eigentliche Entwicklung der Applikation gelenkt werden.
            \item Der Auswahlprozess soll durch die Bereitstellung eines einheitlichen Werkzeuges vereinfacht werden.
            Eine frei verfügbare Plattform, welche den Entwickler so weit wie möglich uneingeschränkt lässt, soll dem
            Entwickler wie im vorherigen Punkt umständliche Konfigurations- und Initialisierungsarbeit abnehmen.
            \item Alle vorherigen Punkte sollen ohne das Einbußen der von der Hardware erbrachten Leistung realisiert
            werden, um die Entwicklung von komplexen Applikationen für Embedded-Systeme zu ermöglichen.
        \end{itemize}

    \subsection{Herstellerunabhängigkeit}
    Abbildung \ref{fig:softwarecomp} zeigt das in Abschnitt \ref{sec:layers} und \ref{sec:abstractionoverview}
    aufgeführte Schichtenmodell der Entwicklung von Embedded-Systemen mit einer zusätzlichen
    \textit{Vendor-Independent-\glsit{glo:api}}-Schicht. Diese soll eine Schnittstelle zu den herstellerspezifischen
    \glsplit{glo:hal} darstellen und einen einheitlichen Zugriff auf \glsit{glo:rtos}-Betriebssysteme ermöglichen
    (hierfür kann auch \glsit{glo:cmsis}-\glsit{glo:rtos} in betracht gezogen werden).

    \glsit{glo:cmsis}-Driver befindet sich in der Grafik auf der selben Ebene wie die herstellerspezifischen \glsplit{glo:hal}. Grund dafür ist,
    dass nicht alle Chip-Hersteller das \glsit{glo:cmsis}-Driver-Interface unterstützen. Unterstützt ein Hersteller das
    \glsit{glo:cmsis}-Driver-Interface, so ist das Interface im Software-Framework, zusammen mit den herstellerspezifischen
    Treibern und \glsplit{glo:hal}, mitintegriert.

    Eine Software, welche von der \glsit{glo:hal} und den entsprechenden Treibern Gebrauch macht,
    muss also in mehreren Szenarien, in denen verschiedene Abstraktionsschichten verfügbar sind und gebraucht werden,
    funktionieren.


    \begin{figure}
        \centering
        \caption{Verwendete CMSIS-Abstraktionsschichten in einer Beispiel\hyp{}Applikation}
        \source{\href{http://www.keil.com/appnotes/files/apnt_268.pdf}{Skript zum Vortrag von Reinhard Keil}, Zugriff: 03.08.17}
        \label{fig:armcmsisexampleapp}
        \includegraphics[scale=0.5]{../Graphiken/STM_DEV_LAYERS_ARM}
    \end{figure}

    Abbildung \ref{fig:armcmsisexampleapp} zeigt ein von \glsit{glo:arm} vorgestelltes Beispiel für eine Applikation, in
    dem die Verwendung der \glsit{glo:cmsis}-Abstraktionsschichten verdeutlicht wird.
    Um vollständige Herstellerunabhängigkeit zu erlangen, muss alles unter dem \textit{Board-Support-Layer} abstrahiert werden.

    Um Implementierungsarbeit zu sparen, sollte, wenn möglich, das \glsit{glo:cmsis}-Driver-Interface genutzt werden. Da wie eben
    schon erwähnt nicht alle Hersteller \glsit{glo:cmsis}-Driver unterstützen und das Interface an sich nicht umfangreich genug für
    manche Applikationen ist, muss die Abstraktion auch direkt mit dem Hersteller-HAL funktionieren.

    Die Software soll noch einen Schritt weiter gehen und dem Entwickler die in Abschnitt \ref{sec:abstractionoverview}
    angesprochene Designer-\glsit{glo:ui} zur Verfügung stellen.

    \subsection{Ansatz der modellbasierten Entwicklung}
    \label{sec:modelbaseddev}
    Das Designer-\glsit{glo:ui} soll es Entwicklern ermöglichen ein funktionales Embedded-System zu entwerfen.
    Es muss ein Überblick über die Hardware geschaffen werden, wobei sämtliche Peripherie konfigurierbar und
    intrinsische so wie externe Kommunikation steuerbar sein muss. Es soll also möglich sein, die Hardware
    über graphische Werkzeuge zu programmieren.

    Folgender Abschnitt gibt einen kurzen Überblick über den Ansatz der modellbasierten Entwicklung und leitet daraus
    Anforderungen an das Designer-\glsit{glo:ui} ab.

    Modellbasierte Entwicklung bezeichnet einen Prozess der Software-Entwicklung. Dieser besteht aus der Erzeugung einer
    lauffähigen Software aus modellierten Komponenten.
    ``Dabei wird die Software nicht mehr textuell mit Programmiersprachen, wie beispielsweise C, programmiert, sondern
    mit graphischen Modellen auf einer höheren Abstraktionsebene spezifiziert und schrittweise über verschiedene
    Entwicklungsphasen von den Anforderungen über den Entwurf bis zur Implementierung verfeinert.''\citep[vgl.~S.~165]{Berns2010}

    In der Praxis setzt sich die modellbasierte Entwicklung aus \acrshortit{glo:uml}-Diagrammen in
    Kombination mit der Verwendung von \acrshortit{glo:case}-Werkzeugen zusammen.
    \acrshortit{glo:case}-Werkzeuge generieren ein Code-Gerüst aus \acrshortit{glo:uml}-Diagrammen, welches vom Entwickler für die weiterführende
    Entwicklung genutzt werden kann.

    Das Ziel des Designer-\glsit{glo:ui} ist jedoch nicht die reine Modellierung durch \acrshortit{glo:uml}-Diagramme, sondern viel mehr einen Weg
    zu finden, bestehende Funktionalitäten und Peripherie miteinander zu vernetzen. Dies soll möglichst übersichtlich
    und intuitiv für Entwickler geschehen.

    Eine abstrakte Definition von Prozess- und Produktmodellierung beschreibt viel mehr ein Paradigma der
    modellbasierten Entwicklung:
    \begin{description}
        \item[Prozess] Ein Prozess erlaubt die Beschreibung von \textit{Aktivitäten}. Aktivitäten sind
        \textit{wiederholbar}, \textit{umkehrbar} und \textit{nachverfolgbar}. Sie schließen unter anderem Aufgaben
        auf unterster Ebene wie zum Beispiel \textit{Refactoring} und \textit{Umbenennung}, so wie auch Aufgaben auf höheren
        Ebenen, wie zum Beispiel den Einsatz von abstrakten Controller-Funktionalitäten der Zielplattform, ein.
        Eine Aktivität ist prinzipiell eine Aufgabe, welche im Rahmen eines Entwicklungsprozesses geschieht, wie zum Beispiel
        die \textit{Definition eines System-Interfaces} oder die \textit{Generierung von Test-Szenarien}
        \citep[vgl.~Kap.~1~und~2]{Schaetz2002}.
        \item[Produkt] Ein Produkt modelliert Artefakte und deren Relationen zueinander. Das heißt für den Bereich der
        Embedded\hyp{}Systeme, dass alle für die Produktion relevanten \textit{Komponenten}, \textit{Zustände},
        \textit{Nachrichten} und deren Beziehungen zueinander in die Produktmodellierung einfließen
        \citep[vgl.~Kap.~1~und~2]{Schaetz2002}.
    \end{description}

    Betrachtet man die Programmierung, in Bezug auf formale Sprachen, für ein Embedded-System schichtenweise, kommt man
    auf drei grobe Bereiche:
    \begin{enumerate}
        \item Programmierung der Hardware mit Assembler-Code
        \item Erste Abstraktion durch Programmiersprachen wie zum Beispiel C
        \item modellbasierte Entwicklung auf graphischer Ebene
    \end{enumerate}

    Es stellt sich die Frage, welche Komponenten auf welche Weise modelliert werden sollen. Dabei muss die Software die
    in Abschnitt \ref{sec:demands} gestellten Anforderungen reflektieren.
    ``Allerdings deckt keines der heute verfügbaren Werkzeuge den gesamten Entwicklungslebenszyklus adäquat ab. Zur
    Modellierung einer Architektur ist es beispielsweise notwendig mehrere Sichten zu definieren, die jeweils
    unterschiedliche Aspekte des Systems repräsentieren''\citep[]{Trapp11}.

    Letztendlich sind folgende Aspekte bei einer graphischen Modellierung zu berücksichtigen:
    \begin{itemize}
        \item Hardware-Abstraktion - Definition der Hardware und deren Attribute so wie zur Verfügung stehende
        Funktionalitäten - \textbf{Schnittstellen/Treiber}\\
        \textbf{Beispiel:} LED kann an- und ausgeschaltet werden
        \item Verteilung von Aufgaben auf verschiedene Hardware-Komponenten\\
        \textbf{Beispiel:} Schalte LED an
        \item Kommunikation zwischen Hardware-Komponenten\\
        \textbf{Beispiel:} Sensor übermittelt Daten an ein Ethernet-Modul
    \end{itemize}

    \subsection{Konzept für eine Integration von OpenOCD in eine IDE}
    Der Schwerpunkt dieser Arbeit liegt auf der für die Hardware-Entwicklung genutzten Software. Im Vordergrund stehen
    dabei das Aufspielen von Firmware auf die Zielhardware und die Fehlersuche bei bereits aufgespielter Firmware über
    ein Host\hyp{}System.

    Der Entwickler muss sich also seine eigene \glsit{glo:toolchain} zusammenstellen, was wie in Abschnitt
    \ref{sec:demands} erläutert durch die Vielzahl an herstellerspezifischen Tools und Komplexität der einzelnen
    Werkzeuge oft ein Problem für Neueinsteiger darstellt.

    \glsplit{glo:ide} fassen die benötigten Werkzeuge zusammen und ermöglichen den Zugriff auf
    diese in einer gemeinsamen Umgebung (ohne sogenannte Medienbrüche).

    Bisher vorhandene Lösungen, wie zum Beispiel \textit{Atmel Studio} oder \textit{Keil IDE} binden den Entwickler an
    eine Plattform und lassen wenig Erweiterbarkeit zu oder sind mit hohen Kosten verbunden. Hinzu kommt, dass sie oft
    herstellerspezifisch (Atmel, STM, etc.) sind.

    Es wird also ein Werkzeug benötigt, welches es einem Hardware-Entwickler ermöglicht innerhalb einer integrierten
    Entwicklungsumgebung eine Applikation für Embedded-Systeme zu entwickeln, ohne sich Gedanken über die benötigten
    Werkzeuge für die Hardware machen zu müssen.

    Das Plugin soll für die \glsplit{glo:ide} der IntelliJ-Plattform
    \footnote{\href{https://www.jetbrains.com/}{Jetbrains-Seite}} entwickelt werden. Die Entscheidungsgründe für die
    Auswahl lauten:
    \begin{itemize}
        \item Plattformunabhängig
        \item Frei verfügbar als \textit{Community}-Version
        \item Verschiedene \glsplit{glo:ide} für verschiedene Sprachen verfügbar. Das Plugin funktioniert
        (außer sprachspezifische Funktionalitäten) für jede \glsit{glo:ide} der IntelliJ-Plattform
        \item Unterstützt Remote-Debugging (\glsit{glo:rsp})
    \end{itemize}

    Da C die am weitesten verbreitete Sprache in der Entwicklung von Embedded-Systemen ist, liegt der Fokus des Plugins
    jedoch auf der Entwicklungsumgebung CLion.
    In zukünftiger Entwicklung des Plugins wird ein Werkzeug konzipiert, mit dem es möglich ist, entwickelte Firmware auf
    den Flash-Speicher eines Prozessors herunterzuladen und auf Fehler zu untersuchen. Da dies mit einem größeren
    Zeitaufwand verbunden ist, wird zunächst das bereits angesprochene \glsit{glo:openocd}, welches für Windows, Linux
    und OSX verfügbar ist, genutzt um ein \glsit{glo:proofofconcept} zu erstellen.
        \subsubsection{Anforderungen}
        \label{sec:pluginrequirements}
        Das Plugin verfolgt das Ziel, dem Entwickler eine Entwicklungsplattform zu liefern mit der Firmware-Programme
        geschrieben, kompiliert, aufgespielt und eine Fehleranalyse durchgeführt werden kann. Es soll also der komplette
        Entwicklungszyklus für die Programmierung von Embedded-Systemen in einer geschlossenen Umgebung abgedeckt
        werden.

        Ein fortgeschrittener Editor zum Schreiben von Firmware wird dabei von der CLion-Entwicklungsumgebung
        bereitgestellt.

        Der Compiler kann in der Entwicklungsumgebung ausgewählt und konfiguriert werden. Hierzu ist ein Build-System
        namens \glsit{glo:cmake} in die \glsit{glo:ide} integriert, mit dem der zu verwendende Compiler bestimmt werden kann. Compiler können
        nach persönlichen Präferenzen und bedingt durch das Host-Betriebssystem ausgewählt werden. Hier ist zu beachten,
        dass der Compiler die richtige Zielarchitektur unterstützen muss.

        \glsit{glo:openocd} stellt eine Funktion zum Aufpielen von Firmware-Programmen auf einen Prozessor bereit. Hierzu müssen
        bestimmte Angaben, wie zum Beispiel die Startadresse im Flash-Speicher an die das Programm-Image geschrieben
        werden soll, gemacht werden. Diese müssen über ein \glsit{glo:ui} in der \glsit{glo:ide} konfigurierbar sein.

        Der letzte Punkt des Entwicklungszyklus, das Debugging, wird von einem lokalen \glsit{glo:gdb}-Server ermöglicht, welcher
        vom \glsit{glo:openocd} gestartet wird. Es muss dem Entwickler ermöglicht werden den lokalen Port für den \glsit{glo:gdb}-Server
        anzugeben. Da CLion Remote-Debugging unterstützt, muss anschließend eine \textit{Debugging-Configuration}
        mit der Adresse, unter dem der \glsit{glo:gdb}-Server erreichbar ist, erstellt werden.

        Schließlich sollte ein Nutzer außerdem über ein \glsit{glo:ui} in der Lage sein das Verhalten des Plugins zu
        konfigurieren. Hierzu gehört die Angabe des Ports, unter dem der \glsit{glo:gdb}-Server erreichbar ist, die für das
        \glsit{glo:openocd} relevante Angabe des Namens für die entsprechende Board-Konfiguration und die Startadresse für das
        Ablegen des Firmware-Programms.

        \subsubsection{Das IntelliJ-SDK}
        \label{sec:sdk}
        Um ein Plugin für eine \glsit{glo:ide} der IntelliJ-Familie zu entwickeln, muss das in Java geschriebene
        \glsit{glo:sdk} von IntelliJ genutzt werden.

        Aktionen, wie zum Beispiel das Öffnen einer Datei oder das Kompilieren eines Programms, werden durch die
        Klasse \texttt{AnAction} modelliert. Beim Auswählen eines Menüpunktes oder Toolbar-Buttons werden sie
        ausgeführt. Aktionen können durch das Ableiten der \texttt{AnAction}-Klasse erstellt
        werden und zu Gruppen hinzugefügt werden. Aus diesen Gruppen lassen sich eigene (Sub-)Menüs und Werkzeugleisten
        für die hinzugefügten Aktionen erstellen.

        Konfigurationen zum Ausführen und Debuggen eines Programms werden in IntelliJ mit \textit{Run Configurations}
        modelliert. Da CLion Remote-Debugging über \glsit{glo:gdb} unterstützt, enthält es dafür auch eine vorgefertigte
        \textit{Run Configuration}. Hier können Adresse und Port, unter denen der \glsit{glo:gdb}-Server erreichbar ist, angegeben
        werden.

        In Anbetracht dieser beiden Modellierungen ergeben sich folgende Ideen für eine Programmstruktur.

        Es lassen sich zunächst Aktionen definieren:
        \begin{itemize}
            \item Der \glsit{glo:openocd}-Daemon muss gestartet und gestoppt werden können
            \item Ein Programm muss für die korrekte Chip-Architektur kompiliert werden
            \item Die kompilierte Programmdatei muss auf den Chip-Speicher geschrieben werden
        \end{itemize}

        Für das Remote-Debugging kann eine \textit{Run Configuration} erstellt werden. Diese muss die korrekten
        Parameter für den \glsit{glo:gdb}-Server enthalten.

        \subsubsection{Architektur}

        Aus den in Abschnitt \ref{sec:pluginrequirements} und \ref{sec:sdk} erarbeiteten Anforderungen und Ansätze
        ergibt sich ein Konzept für die benötigten Komponenten und eine grobe Software-Architektur des Plugins.

        \begin{figure}[h]
            \centering
            \caption{Komponenten und Kommunikationswege}
            \label{fig:commplugin}
            \includegraphics[scale=0.5]{../Graphiken/plugin_arch_v5}
        \end{figure}

        Abbildung \ref{fig:commplugin} zeigt den allgemeinen Entwicklungs- und Kommunikationsprozess der beteiligten
        Instanzen.

        Für die Entwicklung von Embedded-Systemen muss die \glsit{glo:ide} zwei Arten von Frontends besitzen, also (in diesem
        Fall graphische) User-Interfaces, mit denen der Nutzer Funktionalitäten der dahinterliegenden Software benutzen
        kann.

        CLion unterstützt bestimmte Debug-Funktionalitäten, wie zum Beispiel das Setzen von Breakpoints oder
        schrittweise Ausführen von Anweisungen, die mit Hilfe des \glsit{glo:gdb}s realisiert werden. Zusätzlich sind
        hierfür graphische \glsplit{glo:ui} in die \glsit{glo:ide} integriert. Da Remote-Debugging mit dem \glsit{glo:gdb}
        \glsit{glo:rsp} von der Entwicklungsumgebung unterstützt wird, ist das benötigte Frontend für den \glsit{glo:gdb}-Server bereits
        vorhanden. Die \glsit{glo:ide} sendet also vom Nutzer initiierte Debug-Befehle, wandelt sie in das entsprechende
        \glsit{glo:rsp}-Paket um und sendet diese an einen \glsit{glo:gdb}-Server weiter.

        Das zweite Frontend stellt eine direkte Integration des \glsit{glo:openocd}-Werkzeuges dar. Der Nutzer bzw. das Plugin
        kann Funktionalitäten der lokalen Installation von \glsit{glo:openocd} nutzen ohne die \glsit{glo:ide} zu verlassen.

        Das \glsit{glo:openocd}-Programm besteht aus einem \glsit{glo:cli}, welches die Befehle weiter an den \glsit{glo:openocd}-Daemon
        sendet. Beim Start des \glsit{glo:openocd}-Daemons, wird gleichzeitig der lokale \glsit{glo:gdb}-Server gestartet.

        Der \glsit{glo:gdb}-Server empfängt vom \glsit{glo:gdb}\hyp{}Frontend der \glsit{glo:ide} gesendete Debugging\hyp{}Befehle und gibt diese weiter an den
        \glsit{glo:openocd}\hyp{}Daemon. Dieser wandelt die \glsit{glo:gdb}-Befehle (\glsit{glo:rsp}) in die entsprechenden \acrshortit{glo:jtag}-Befehle um und überträgt
        diese über \glsit{glo:usb} mit Hilfe des \glsit{glo:cmsisdap}-Protokolls an die Debug-Einheit der Target\hyp{}Hardware.

        Das \glsit{glo:openocd}-Frontend muss also drei grundlegende Aufgaben erfüllen
        \begin{enumerate}
            \item Der \glsit{glo:openocd}-Daemon muss gestartet werden
            \item Der \glsit{glo:openocd}-Daemon muss mit den gewünschten Einstellungen konfiguriert werden
            \item Das Programm-Image muss über \glsit{glo:openocd} und somit \glsit{glo:cmsisdap} auf das Board heruntergeladen werden
        \end{enumerate}

        Um den kompletten Entwicklungszyklus abzudecken fehlt die Kompilierung des Programms. Das von CLion
        unterstützte Build-System \glsit{glo:cmake} muss so konfiguriert werden, dass es
        \begin{enumerate}
            \item den richtigen Compiler zum Kompilieren für \glsit{glo:arm} Architekturen nutzt
            \item das Image an eine vorgesehene Stelle schreibt, auf die das \glsit{glo:openocd}-Frontend Zugriff hat
        \end{enumerate}

        Schließlich sind Besonderheiten des Host-Systems zu beachten. Es muss sichergestellt werden, dass die
        benötigte Software installiert ist (Cross-Compiler für \glsit{glo:arm}-Architekturen und \glsit{glo:openocd}).
        Da das Plugin so ausgelegt ist, dass es native Aufrufe des \glsit{glo:openocd} Programms ausführt, muss für
        Windows-basierte Betriebssysteme eine andere Logik entwickelt werden als für Unix-basierte Betriebssysteme.

        Es ergibt sich zunächst eine konzeptionelle Einteilung in folgende Software-Pakete für das Plugin:
        \begin{description}
            \item[\texttt{actions}] Das \texttt{actions}-Paket enthält alle Aktionen des \glsit{glo:openocd}-Frontends.
            Konzipierte Aktionen sind das Starten des \glsit{glo:openocd}-Daemons mit den richtigen Einstellungen und das
            Aufspielen des Programm-Images.
            \item[\texttt{extensions}] Als \texttt{extensions} werden alle Programmteile gesehen, die eine aktive
            Erweiterung der bereits vorhandenen \glsit{glo:ide}-Elemente sind. Ein Beispiel hierfür ist eine Initialisierungslogik,
            die beim Öffnen eines Projektes ausgeführt wird. Die Formulierung ist hierbei wie im vorherigen Punkt von
            dem IntelliJ-SDK übernommen.
            \item[\texttt{ui}] Alle graphischen Komponenten befinden sich im \texttt{ui}-Paket. Hierzu gehört im
            wesentlichen eine Toolbar, welche alle Einstellungsmöglichkeiten für das \glsit{glo:openocd} enthalten soll.
            \item[\texttt{util}] Das \texttt{util}-Paket enthält alle sonstigen Hilfsklassen, wie zum Beispiel eine
            Klasse mit Methoden zum Ausführen nativer Aufrufe.
        \end{description}

        Als Letztes stellt sich noch die Frage, welcher Compiler zum Kompilieren des Firmware-Programms genutzt
        wird. Verwendet werden soll der \glsit{glo:gcc}, da dieser die \glsit{glo:arm}-Architektur als Zielsystem unterstützt und
        Installationsdateien für OSX, Linux und Windows enthält.

        Das benötigte Paket trägt den Namen \texttt{arm-none-eabi-gcc} und enthält ein \glsit{glo:gcc}-Compiler für die \glsit{glo:arm}-Architektur.

    \subsection{Entwicklung}
        Die Entwicklung lässt sich in die drei Bereiche \textit{Vorbereitung und Einrichtung der Toolchain},
        \textit{Integration von \glsit{glo:openocd} in die \glsit{glo:ide}} und \textit{Erstellen eines User-Interfaces} unterteilen.
        \subsubsection{Compiler und CMake}
        Das Einrichten eines Cross-Compilers mit \glsit{glo:cmake} in CLion ist eine nicht-triviale Aufgabe. Im Folgenden sind
        die Schritte aufgelistet, die notwendig sind um ein C-Programm über CLion für die \glsit{glo:arm}-Architektur zu
        kompilieren.
        \begin{enumerate}
            \item Die voreingestellte \glsit{glo:toolchain} muss durch eine neue auf \glsit{glo:arm}-Architekturen ausgelegte \glsit{glo:toolchain}
            ersetzt werden. Hierfür wird eine neue \glsit{glo:cmake}-Datei mit einem beliebigen Namen (in dem Testdurchlauf
            \texttt{cc\_test.cmake} genannt) erstellt, dessen Inhalt in Listing \ref{cccmakeconf} angegeben ist.
            \begin{lstlisting}[
                language=CMake,
                caption=Verkürzte Version der CMake-Konfigurationsdatei für die Cross-Compiler-Toolchain,
                label=cccmakeconf
            ]
include(CMakeForceCompiler)
set(CMAKE_SYSTEM_NAME Generic)
set(CMAKE_SYSTEM_PROCESSOR cortex-m7)

find_program(ARM_CC arm-none-eabi-gcc
        ${TOOLCHAIN_DIR}/bin
        )
find_program(ARM_CXX arm-none-eabi-g++
        ${TOOLCHAIN_DIR}/bin
        )
find_program(ARM_OBJCOPY arm-none-eabi-objcopy
        ${TOOLCHAIN_DIR}/bin
        )
find_program(ARM_SIZE_TOOL arm-none-eabi-size
        ${TOOLCHAIN_DIR}/bin)

# specify the cross compiler
CMAKE_FORCE_C_COMPILER(${ARM_CC} GNU)
CMAKE_FORCE_CXX_COMPILER(${ARM_CXX} GNU)
            \end{lstlisting}
            Die erste Anweisung schließt das CMakeForceCompiler-Modul von \glsit{glo:cmake} ein, welches bestimmte Makros für die
            Benutzung von Cross-Compiling-\glsplit{glo:toolchain} definiert\citep[vgl.~module/CMakeForceCompiler]{CMAKEDOC}.

            Mit \texttt{find\_program()} wird nach den für die \glsit{glo:toolchain} relevanten Programmen gesucht und
            in die Variablen geschrieben, somit ist eine Überprüfung mitintegriert. Die Compiler können mit
            \texttt{CMAKE\_FORCE\_C\_COMPILER} und \texttt{CMAKE\_FORCE\_CXX\_COMPILER} gesetzt werden. Die Variable
            \texttt{\$\{TOOLCHAIN\_DIR\}} zeigt auf den Pfad zum Ordner, welcher die benötigten Compiler beinhaltet.
            \item Als nächstes müssen die \glsit{glo:toolchain}-Einstellungen von \glsit{glo:cmake} übernommen werden. Dafür müssen unter
            \texttt{Settings} $\rightarrow$ \texttt{Build, Execution, Deployment} $\rightarrow$ \texttt{CMake}
             $\rightarrow$ \texttt{CMake options} die beiden \glsit{glo:cmake}-Optionen
            \begin{Verbatim}[frame=single]
-DTOOLCHAIN_DIR:PATH=/usr/bin/
-DCMAKE_TOOLCHAIN_FILE=cc_test.cmake
            \end{Verbatim}
            gesetzt werden.
            \item Schließlich muss der \glsit{glo:cmake}-Cache in der \glsit{glo:ide} gelöscht und die \glsit{glo:cmake}-Dateien neu geladen werden.
        \end{enumerate}

        Um diesen Prozess im Plugin weitesgehend zu automatisieren, muss die \glsit{glo:cmake}-Datei aus Listing \ref{cccmakeconf}
        automatisch in das oberste Verzeichnis des Projekts geschrieben werden. Der Pfad zur \glsit{glo:toolchain} ist
        dabei in Form einer Benutzereinstellung zu setzen.

        \subsubsection{OpenOCD-Frontend}
        Bei der Implementierung des \glsit{glo:openocd}-Frontends sind einige Sachen zu beachten.

        Auf Unix-ähnlichen
        \footnote{Mit ``Unix-ähnlichen Systemen'' sind hauptsächlich die Betriebssysteme Linux und OS X gemeint}
        Systemen benötigt das \glsit{glo:openocd} \glsit{glo:root}-Rechte, um ein Programm-Image auf den Flash-Speicher der Zielhardware
        herunterzuladen. Dazu wird der \texttt{sudo} Befehl genutzt, welcher bestimmten Benutzern die Möglichkeit gibt
        einen Prozess, der \glsit{glo:root}-Rechte benötigt, auszuführen ohne das \glsit{glo:root}-Passwort kennen zu müssen.
        Vorraussetzung ist, dass \texttt{sudo} für den aktuellen Benutzer konfiguriert ist. Ist dies nicht der Fall,
        muss der Benutzer trotzdem ein Passwort eingeben.

        Das Plugin ruft einen nativen \glsit{glo:openocd}-Prozess auf, indem die Java-\glsplit{glo:api}\\\texttt{Process} und
        \texttt{ProcessBuilder} verwendet werden.

        Die benötigten \glsit{glo:openocd}-Kommandos lauten
        \begin{Verbatim}[frame=single]
$ openocd -f interface/cmsis-dap.cfg -f board/<BOARD CFG>
        \end{Verbatim}
        und
        \begin{Verbatim}[frame=single]
$ openocd -f board/<BOARD CFG>
    -c "program <IMAGE> exit <EXIT ADDRESS>"
        \end{Verbatim}

        Der erste Befehl dient dazu, dass das Programm \glsit{glo:openocd} mit dem \glsit{glo:cmsisdap}-Protokoll
        konfiguriert und der \glsit{glo:openocd}-Daemon gestartet wird. Bei jedem \glsit{glo:openocd}-Befehl, bei dem
        der Daemon gestartet wird, muss die Board-Konfiguration angegeben werden.

        Der zweite Befehl dient dazu, den \glsit{glo:openocd}-Daemon zu starten und dabei ein Programm-Image auf den Flash-Speicher
        des Chips herunterzuladen.

        Beide Befehle werden in Java-Code eingebettet und mit den entsprechenden Parametern versehen, die vom Nutzer
        in der \glsit{glo:ide} einstellbar sind.

        Außerdem können beim Starten des Daemons noch die gewünschten Port\hyp{}Nummern gesetzt werden. Diese sind
        standardmäßig so eingestellt, dass der \glsit{glo:gdb}-Server auf Port \texttt{3333}, der \glsit{glo:tcl}-Server auf Port
        \texttt{6666} und der \glsit{glo:telnet}-Server auf Port \texttt{4444} hört. Da weder der \glsit{glo:tcl}- noch der \glsit{glo:telnet}-Server
        in diesem Kontext benötigt wird, wird nicht weiter darauf eingegangen.

        \subsubsection{User-Interface}
        Das User-Interface des Plugins findet sich im Hauptmenü der \glsit{glo:ide} unter dem Menüpunkt
        \texttt{\underline{E}mbedded Systems} und in der Toolbar am rechten Bildschirmrand wieder.
        In der Toolbar können bestimmte Konfigurationen, wie zum Beispiel der Pfad zum Programm-Image, vorgenommen werden.

        Die graphischen Komponenten wurden nach Vorgaben des IntelliJ-\glsit{glo:sdk}s mit der Java-\texttt{Swing}-Bibliothek
        erstellt.

        Beim Registrieren der Aktionen (Starte Daemon, Lade Programm-Image und Generiere \glsit{glo:cmake}-Konfigurationsdatei) in der
        \texttt{Plugin.xml}-Datei, lassen sich Gruppen und sogenannte \textit{anchor-points} angeben. Diese bestimmen,
        an welche Stelle im \glsit{glo:ui} die Aktionen als Menüeinträge erscheinen. In diesem Fall wurde eine neue Gruppe
        erstellt und der Gruppe \texttt{MainMenu} als letztes angehängt.

        Bei jeder Auswahl des Aktionspunktes im Menü wird die in der Aktionsklasse überschriebene Methode
        \texttt{actionPerformed(ActionEvent e) \{...\}} ausgeführt.

        \subsubsection{Aktueller Stand und Ausblick}
        Folgende Funktionalitäten wurden im Plugin umgesetzt:
        \begin{itemize}
            \item Es wird für den Nutzer eine \glsit{glo:cmake}-Konfigurationsdatei generiert, in der die Verwendung der benötigten
            \glsit{glo:toolchain} für das Kompilieren eines Firmware-Programms auf \glsit{glo:arm}-Architekturen eingestellt wird.
            Dabei wird auf die Verfügbarkeit der \glsit{glo:toolchain}-Programme geprüft und CLion versucht bei jedem Laden der
            neuen \glsit{glo:cmake}-Konfiguration ein Testprogramm zu kompilieren und auszuführen um die Compiler auf ihre
            Funktionstüchtigkeit zu prüfen.
            \item Ein kompiliertes Programm kann durch den gestarteten \glsit{glo:openocd}-Daemon per \glsit{glo:usb} auf den Flash-Speicher der
            Zielhardware heruntergeladen werden. Dabei werden die von \glsit{glo:openocd} bereitgestellten Flash-Algorithmen
            genutzt.
            \item Der \glsit{glo:openocd}-Daemon kann, ohne dass die Firmware auf das Board heruntergeladen wird, gestartet werden.
            Dabei kann der Port für den lokalen \glsit{glo:gdb}-Server gesetzt werden. Über die \glsit{glo:ide} kann eine neue
            \textit{Remote-Debugging-Configuration} angelegt werden und anschließend das auf der Zielhardware laufende Programm
            auf Fehler untersucht werden.
        \end{itemize}

        Da es sich bei dem Plugin lediglich um ein \glsit{glo:proofofconcept} handelt, fehlen einige Aspekte und
        Komponenten, die zu der Robustheit und \textit{User-Experience} beitragen.
        \begin{itemize}
            \item Es muss ein Überblick geschaffen werden, welche Hardware von \glsit{glo:openocd} unterstützt wird. Eine Liste
            aller verfügbaren Konfigurationsdateien sollte für den Entwickler ohne umfangreiche Suche einsehbar sein.
            \item Das \glsit{glo:ui} bedarf an vielen Stellen einer Überarbeitung. Beispielsweise sollte dem Nutzer idealerweise
            ein Status des \glsit{glo:openocd}-Daemons, sowie Informationen zum \glsit{glo:gdb}-Server und der Zielhardware angezeigt werden.
            \item Das Konfigurieren der \glsit{glo:toolchain} soll einfacher und mit noch wenigeren Umständen geschehen.
            Zum Beispiel soll der Entwickler nicht mehr den Pfad der \glsit{glo:cmake}-Konfigurationsdatei in CLion
            als CMake-Option eintragen müssen.
            \item Fehlende benötigte Software-Komponenten, wie zum Beispiel das \glsit{glo:openocd} oder die
            \glsit{glo:arm}-Compiler-\glsit{glo:toolchain}, sollen vom Plugin erfasst und automatisch heruntergeladen
            werden.
            \item Überwachung (Monitoring) der Zielhardware und die Möglichkeit verschiedene Laufzeitparameter zu
            ändern, sollten dem Entwickler zur Verfügung stehen.
            \item Die Konfigurationen für das \glsit{glo:openocd}-Plugin müssen persistent gemacht werden, indem sie in eine
            Konfigurationsdatei geschrieben und bei jedem Start wieder ausgelesen werden.
            \item Beim Testen des Plugins wurden trotz \glsit{glo:jvm} unterschiedliche Verhaltensweisen beim Ausführen nativer
            Prozesse festgestellt. Beispiele hierfür sind die verschiedenen Rechte-Verwaltungen oder die Angabe von
            Dateipfaden (Leerzeichen, Anführungszeichen, etc.).
        \end{itemize}

        Letztendlich stellt das Plugin einen ersten Ansatz dar, um eine frei verfügbare Alternative für die Integrierung
        von Entwicklungswerkzeugen für Embedded-Systeme in eine Entwicklungsumgebung zu schaffen.

        Das \glsit{glo:openocd} steht dabei für ein Entwicklungswerkzeug, welches durch eine eigens implementierte
        Lösung ersetzt werden soll (siehe Abschnitt \ref{sec:prospect}).

	\section{Fazit}

    \subsection{Zusammenfassung}
    Die Entwicklung von Embedded-Systemen mit Mikrocontrollern gewinnt an Popularität. Ein Hauptgrund dafür ist der
    Trend des \glsit{glo:iot}. Durch Plattformen wie Arduino und Raspberry Pi wird diese Entwicklung
    herangetrieben und zugänglicher gemacht.

    Die am weitesten verbreitete Chip-Architektur kommt von \glsit{glo:arm}, welche die zum heutigen Zeitpunkt relevanten
    Prozessorfamilien Cortex-A, -R und -M entwickeln. Diese Architekturen werden an verschiedene Hardware-Hersteller
    verkauft, wie zum Beispiel \textit{Texas Instruments}, \textit{Atmel} oder \textit{STMicroelectronics}.

    Zwischen der Hardware und einer Hardware-Applikation liegen mehrere Abstraktionsschichten.

    Dabei nutzt der Entwickler bzw. die Entwicklungswerkzeuge die Schichten mit den Technologien \acrshortit{glo:jtag}/\acrshortit{glo:swd} und \glsit{glo:cmsisdap}
    (falls unterstützt).

    Eine auf \glsit{glo:arm}-Cortex-basierende Applikation nutzt in der Regel ein Echtzeitbetriebssystem (oft bei Applikationen
    mitintegriert), vom Hersteller zur Verfügung gestellte Treiber und einen \glsit{glo:hal}.
    Diese nutzen die von \glsit{glo:arm} entwickelte unterste Abstraktionsschicht \glsit{glo:cmsis}-Core.

    Der Entwicklungsprozess wird durch folgende Punkte gestört:
    \begin{itemize}
        \item Herstellerspezifische Hardware-Abstraktionen (Applikation schlechter portierbar)
        \item Plattformbindende Entwicklungswerkzeuge
        \item Teure Lizenzen
        \item Herstellerspezifische Entwicklungswerkzeuge
        \item Steile Lernkurve für Neueinsteiger (größtenteils durch die vorherigen Punkte veranlasst)
    \end{itemize}

    Die Analyse eines Programms auf Fehler und Effizienz spielt eine große Rolle im Entwicklungsprozess von
    Applikationen. Dafür gibt es mehrere Ansätze, welche sich in dem Bereich der Embedded-Systeme etabliert haben,
    jedoch meistens die Anschaffung von externer Hardware voraussetzen.

    Der \acrshortit{glo:jtag}-Standard stellt eine einheitliche Schnittstelle zur Logik des Prozessors (oder jedem
    \glsit{glo:ic}, der \acrshortit{glo:jtag} implementiert) dar, mit dem Debugging-Anweisungen, auch zur Laufzeit, an den
    Prozessor zur Verarbeitung gesendet werden können.

    Eine Alternative zur Fehleranalyse mit externer Hardware stellt \glsit{glo:cmsisdap} dar, welches auf einer Debug-Einheit in Form
    eines \glsit{glo:ic} auf der Zielhardware verbaut ist und in den meisten Fällen einen \glsit{glo:usb}-Anschluss besitzt. Somit können
    Debugging-Anweisungen mit Hilfe des \acrshortit{glo:jtag}-Standards vom Entwicklungssystem direkt an die Zielhardware übertragen
    werden.

    Zum Herunterladen des entwickelten Firmware-Programms und anschließend zur Fehlersuche auf der Zielhardware kann
    neben externer Debugging-Hardware auch das Programm \glsit{glo:openocd} verwendet werden, welches durch den \acrshortit{glo:jtag}-Standard ein
    Protokoll und die benötigte Hardware für \textit{On-Chip Debugging} zur Verfügung gestellt bekommt.

    Software-seitige Ansätze, so wie \glsit{glo:openocd}, nutzen diese Technologie, um den Debugger auf dem Entwicklungssystem mit
    der Zielhardware kommunizieren zu lassen. Diese Arbeit thematisiert unter anderem auch die Integration einer
    solchen Software in eine \glsit{glo:ide}.

    Um den genannten Problemen und Hindernissen des Entwicklungsprozesses entgegenzuwirken, wird ein Konzept für eine
    Entwicklungsplattform mit folgenden Komponenten aufgestellt:
    \begin{itemize}
        \item Einfach zu nutzende und herstellerunabhängige Abstraktionsschicht für Funktionalitäten der Hardware, um
        Applikations-Code portabler zu machen und bei der trotzdem die maximale Leistung der verwendeten Hardware
        genutzt werden kann
        \item Designer-\glsit{glo:ui}, welches einem Entwickler eine einfache Alternative bietet Applikationen für
         beliebige Hardware zu entwickeln
        \item Erweiterbare Hardware, welche aber nicht zwangsläufig verwendet werden muss
    \end{itemize}

    \subsection{Ausblick}
    \label{sec:prospect}
    Der nächste Schritt, um den Entwicklungsprozess von Embedded-Systemen zugänglicher zu machen, ist aus dem Konzept
    für die Hardware einen Prototypen zu erstellen.

    An dem Hardware-Prototypen kann anschließend eine Schnittstelle für den herstellerunabhängigen
    \glsit{glo:hal} entwickelt werden. Die Implementierung der Interfaces erfolgt zunächst für die Hardware
    des Prototypen, anschließend soll die Implementierung von so vielen verschiedenen Hardware-Ausführungen wie möglich
    erfolgen.

    Mit der Definierung der Schnittstelle kann das \textit{Designer-}\glsit{glo:ui} entwickelt werden. Unter den in
    Abschnitt \ref{sec:modelbaseddev} aufgeführtem Paradigma zur modellbasierten Entwicklung wird eine graphische
    Entwicklungs- und Konfigurierungsmöglichkeit erstellt.

    Im Hinblick auf das \textit{Designer-}\glsit{glo:ui} soll ein abstrakter Weg gefunden werden, die Kommunikation
    unterschiedlicher Services (Funktionalitäten von Peripherie) mit verschiedenen Protokollen zu regeln.
    In anderen Worten: Wie lassen sich in einem Datenfluss zwischen zwei abstrahierten Hardware-Instanzen beliebige
    Protokolle aneinanderketten, um die Kommunikation zwischen diesen zu ermöglichen?\\
    \textbf{Beispiel:} Ein Sensor übermittelt gemessene Daten an einen Ethernet-Adapter. Zwischen den beiden
    Peripheriegeräten wären die in Abbildung \ref{fig:protocolabstraction} dargestellten Protokollschichten denkbar.
    \begin{figure}
        \centering
        \caption{Protokollschichten}
        \label{fig:protocolabstraction}
        \source{eigene Zeichnung}
        \includegraphics[scale=0.6]{../Graphiken/protocol_konsch}
    \end{figure}
    \texttt{MyProtocol} bezeichnet dabei ein vom Entwickler definiertes Format zur Datenübertragung.

    Im Verlauf des Prozesses ist nach neuen Technologien für potenzielle Alternativen zu den bis jetzt eingesetzten
    Technologien Ausschau zu halten. Eine Vielzahl an verschiedenen Hardware-Herstellern bedeutet gleichzeitig auch
    einen Wettlauf um die Entwicklung neuer Technologien zur Vereinfachung der Hardware-Entwicklung.
    Beispielsweise stellt der Hersteller NXP eine Flash-\glsit{glo:api} zur Verfügung, um ein Firmware-Programm vom
    Dateiverwaltungssystem eines Host-Systems per \glsit{glo:usb} direkt auf die Zielhardware herunterzuladen (``flashen''). Unter
    Windows kann ein Entwickler beispielsweise die kompilierte Datei per \textit{Drag-and-Drop} auf den Flash-Speicher
    der Zielhardware spielen.

    Auch Veränderungen im Bereich der Entwicklungsumgebungen sind zu beachten. In regelmäßigen Abständen werden neue
    Funktionalitäten hinzugefügt, von denen auch ein Entwickler für Embedded-Systeme profitieren könnte.

    Der \glsit{glo:openocd} ist durch eine ähnliche aber selbst entwickelte Software zu ersetzen. Diese soll eine deutlich
    reduzierte Version darstellen und auf die Verwendung als \glsit{glo:ide}-Plugin optimiert werden.

    Die zu entwickelnden Komponenten verfolgen das Ziel, den Entwickler in allen Aspekten des Entwicklungsprozesses für
    Embedded-Software zu unterstützen und ergeben zusammen eine neue Plattform für die Entwicklung von Embedded-Systemen.
	
	\appendix
\section{Abgabe}
    Teil der Abgabe ist ein Datenträger mit dem Quelltext, welcher im Rahmen dieser Arbeit erstellt worden ist.
    \dirtree{%
    .1 /.
    .2 \texttt{embedded\_plugin.jar} \DTcomment{Das Plugin als \texttt{.jar}-Datei um es zu installieren}.
    .2 EmbeddedPlugin/ \DTcomment{Plugin}.
    .3 resources/ \DTcomment{Ressourcen}.
    .4 icon/.
    .4 META-INF/ \DTcomment{Enthält die Konfigurationsdatei für das Plugin}.
    .3 src/ \DTcomment{Quellcode}.
    .2 Bachelorarbeit\_Calvert\_4012564.pdf \DTcomment{Diese Arbeit, in digitaler Form}.
    }
\section{Installationsanleitung}
    \subsection{Software-Abhängigkeiten}
    \begin{itemize}
        \item \href{https://www.jetbrains.com/clion/}{CLion Version 2017.2}\\
        \texttt{Build \#CL-172.3317.49, built on July 11, 2017}
        \item \href{http://openocd.org/}{Open On-Chip Debugger 0.10.0} Lizensiert unter der GNU GPL v2
        \item \href{https://developer.arm.com/open-source/gnu-toolchain/gnu-rm}{GNU ARM Embedded Toolchain}
        \item \href{https://www.jetbrains.com/idea/}{IntelliJ} Community oder Ultimate Edition (optional, um das
        Projekt zu bauen)
    \end{itemize}

    \subsection{CLion Plugin installieren}
    Plugins werden in CLion unter dem Menüpunkt \texttt{Settings} $\rightarrow$ \texttt{Plugins}
    verwaltet. Hier kann \texttt{Install plugin from disk} ausgewählt werden, um ein Plugin aus einer \texttt{.jar}-
    oder \texttt{.zip}-Datei zu installieren.

    Alternativ hierzu kann die Archiv-Datei, welche das Plugin enthält, in das Plugin-Verzeichnis der entsprechenden
    CLion-Installation kopiert werden und anschließend in der Plugin-Liste ausgewählt werden. Hier wird auf Abbildung
    \ref{fig:pluginlistinstall} verwiesen.

    \begin{figure}[h]
        \centering
        \caption{Plugin Liste mit hinzugefügtem Plugin}
        \label{fig:pluginlistinstall}
        \includegraphics[scale=0.5]{../Graphiken/plugininstall}
    \end{figure}

    Um die Funktionalität des Plugins zu aktivieren ist ein Neustart der IDE nötig.

    \subsection{Source Code}
    Der mitgelieferte Source-Code kann bei Bedarf selbst kompiliert und ausgeführt werden.
    Da IntelliJ keine Modulerkennung für Plugins unterstützt kann externer Plugin-Source-Code nicht ohne Weiteres
    importiert und ausgeführt werden.
    Um den Source-Code trotzdem selbst ausführen zu können muss über
    \href{https://www.jetbrains.com/idea/}{IntelliJ} zunächst ein \underline{neues} Plugin-Projekt angelegt werden.
    Ist beim Project-Wizard keine Plugin-Option verfügbar, muss zunächst die \textit{Plugin-Development}-Erweiterung
    von Jetbrains heruntergeladen und aktiviert werden (\texttt{Settings} $\rightarrow$ \texttt{Plugins}).

    Wurde ein neues Plugin-Projekt angelegt, muss das korrekte SDK ausgewählt werden. Die Einstellung erfolgt unter
    dem Menüpunkt \texttt{File} $\rightarrow$ \texttt{Project Structure} $\rightarrow$ \texttt{Project}.
    Danach \texttt{New...} und anschließend \texttt{IntelliJ Plattform Plugin SDK} auswählen und auf den CLion
    Installationsordner zeigen.

    Zuletzt muss der gesamte Source-Code (\texttt{./src}) in den Source-Ordner und alle Ressourcen (./resources) in
    den Ressourcenordner des neu angelegten Projekts kopiert werden. Die \texttt{plugin.xml}-Datei soll dabei
    überschrieben werden.

    Nach diesen Schritten ist das Plugin-Projekt bereit zur Ausführung.
	
	\section{Verwendung des Plugins}
Das Plugin definiert drei Aktionen: CMake Konfiguration generieren, OpenOCD Daemon starten und Programm-Image auf die
Ziel-Hardware spielen.

Alle drei Aktionen finden sich zum Einen im Hauptmenü unter dem Menüpunkt \texttt{\underline{E}mbedded Systems} und zum
Anderen in der Seitenleiste unter dem Toolwindow \texttt{Embedded Dev} wieder.

    Um die Toolchain einzurichten, müssen bestimmte CMake-Konfigurationen vorgenommen werden. Hierzu generiert das
    Plugin eine Datei, mit der die Toolchain ausgewählt und geprüft wird.

    Die Konfigurationsdatei muss als Toolchain-Datei für CMake definiert werden. Da das Plugin (zu jetzigem Stand)
    keinen Weg bereitstellt CMake-Optionen zu setzen, muss der Benutzer selber die CMake-Option \\
    \texttt{-DCMAKE\_TOOLCHAIN\_FILE=arm\_toolchain.cmake} unter dem Menüpunkt \\
    \texttt{Settings} $\rightarrow$
    \texttt{Build, Execution, Deployment} $\rightarrow$ \texttt{CMake} $\rightarrow$ \texttt{CMake Options} eintragen.

    Um die Konfigurationen zu laden, muss der CMake-Cache der IDE gelöscht und neu geladen werden. Dies kann im
    Hauptmenü unter dem Menüpunkt \texttt{Tools} $\rightarrow$ \texttt{CMake} $\rightarrow$
    \texttt{Reset Cache and Reload Project} getan werden.

    Nun kann der Source-Code über \texttt{Build} kompiliert werden. Ein Output-Ordner kann über das Toolwindow an der
    rechten Seite angegeben werden.\\
    \textbf{Hinweis:} Sollte das Programm nicht kompilieren und die Fehlermeldung
    \begin{Verbatim}
/usr/lib/gcc/arm-none-eabi/7.1.0/../../../../arm-none-eabi/lib/libg.a
    (lib_a-exit.o): In function `exit':
exit.c:(.text.exit+0x2c): undefined reference to `_exit'
        ...
    \end{Verbatim}
    erscheinen, so ist die in der CMake-Konfigurationsdatei \texttt{arm\_toolchain.cmake} auskommentierte Zeile
    \begin{Verbatim}
set(CMAKE_C_FLAGS "${CMAKE_C_FLAGS} --specs=nosys.specs")
    \end{Verbatim}
    zu setzen. Nun sollte das Programm kompilieren
    \footnote{Schuld daran sind undefinierte Funktionen, die in der ARM libc aufgerufen werden.
    Eine vorübergehende Lösung ist das setzen des \texttt{specs}-Parameters.
    \href{https://stackoverflow.com/a/23922211/6755839}{Siehe Frage bei Stackoverflow}}.

    Ein Programm wird mit Hilfe der \texttt{Flash Binary}-Aktion auf die Zielhardware heruntergeladen.
    Dafür müssen im \textit{OpenOCD-Panel} jeweils der Pfad zur kompilierten Programmdatei und die Exit-Adresse
    angegeben werden. Außerdem ist es wichtig, dass der Pfad zur OpenOCD installation angegeben ist.

    Zum Starten des Daemons und damit auch des GDB-Servers kann die Aktion \texttt{Start OpenOCD Daemon}
    ausgewählt werden. Die Fehlersuche kann über eine \textit{GDB Remote Debug}-Konfiguration mit CLion ausgeführt
    werden. Hierzu muss unter der zu erreichenden Adresse der Port des GDB-Servers mitangegeben werden. Standardmäßig
    ist dieser bei \texttt{3333}, kann jedoch im OpenOCD-Panel verändert werden.
    \\

    Die Hardware muss JTAG/SWD und CMSIS-DAP unterstützen. OpenOCD funktioniert mit Prozessoren basierend auf den
    ARM7/9/11,-Cortex und MIPS-Architekturen.

    \textbf{Hinweis:} Es können nur die vom Prozessor unterstützten Funktionalitäten für das Debugging genutzt werden.

	
	\newpage
	
	\printglossaries

	\newpage
	
	\bibliographystyle{dinat}
	\bibliography{literature}
	
	\newpage
	\pagestyle{empty}
	Zunächst möchte ich mich an dieser Stelle bei all denjenigen bedanken, die mich während der Anfertigung dieser
Bachelorarbeit unterstützt und motiviert haben.\\

Ich danke Herrn Prof. Dr. Volker Sander, für die Betreuung dieser Arbeit und die ausführlichen Hinweise und Anregungen
während der Bearbeitung.\\

Ich danke außerdem meinem Zweitprüfer Lukas Knuth für seine konstruktive Kritik und Verbesserungsvorschläge.
Ja Lukas, hast mich schon verstanden.\\

Insbesondere gilt mein Dank meinem Chef und Ausbilder Sebastian Floss, der mir das Thema Embedded-Systeme näher brachte
und stets hilfsbereit war.
Weiterhin ermöglichte er mir, die Arbeit im Unternehmen ImagineOn durchzuführen.\\

Ich danke vor Allen Sebastian Vogt und Sebastian Scheliga für das Korrigieren der Arbeit, so wie für das Aushalten meiner
grausamen Rechtschreibung und Artikulation, die vielen Nebensätze, und für die Erkenntnis, dass echt viele Menschen
Sebastian heißen, und die vielen doppelten Nebensätze.\\

Mein besonderer Dank gilt meinen Eltern, Freunden und Verwandten, die mich während des Studiums und insbesondere der
Bearbeitungszeit dieser Arbeit unterstützt haben.

\end{document}
