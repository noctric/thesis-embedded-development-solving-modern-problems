\appendix
\section{Abgabe}
    Teil der Abgabe ist ein Datenträger mit dem Quelltext, welcher im Rahmen dieser Arbeit erstellt worden ist.
    \dirtree{%
    .1 /.
    .2 \texttt{embedded\_plugin.jar} \DTcomment{Das Plugin als \texttt{.jar}-Datei um es zu installieren}.
    .2 EmbeddedPlugin/ \DTcomment{Plugin}.
    .3 resources/ \DTcomment{Ressourcen}.
    .4 icon/.
    .4 META-INF/ \DTcomment{Enthält die Konfigurationsdatei für das Plugin}.
    .3 src/ \DTcomment{Quellcode}.
    .2 Bachelorarbeit\_Calvert\_4012564.pdf \DTcomment{Diese Arbeit, in digitaler Form}.
    }
\section{Installationsanleitung}
    \subsection{Software-Abhängigkeiten}
    \begin{itemize}
        \item \href{https://www.jetbrains.com/clion/}{CLion Version 2017.2}\\
        \texttt{Build \#CL-172.3317.49, built on July 11, 2017}
        \item \href{http://openocd.org/}{Open On-Chip Debugger 0.10.0} Lizensiert unter der GNU GPL v2
        \item \href{https://developer.arm.com/open-source/gnu-toolchain/gnu-rm}{GNU ARM Embedded Toolchain}
        \item \href{https://www.jetbrains.com/idea/}{IntelliJ} Community oder Ultimate Edition (optional, um das
        Projekt zu bauen)
    \end{itemize}

    \subsection{CLion Plugin installieren}
    Plugins werden in CLion unter dem Menüpunkt \texttt{Settings} $\rightarrow$ \texttt{Plugins}
    verwaltet. Hier kann \texttt{Install plugin from disk} ausgewählt werden, um ein Plugin aus einer \texttt{.jar}-
    oder \texttt{.zip}-Datei zu installieren.

    Alternativ hierzu kann die Archiv-Datei, welche das Plugin enthält, in das Plugin-Verzeichnis der entsprechenden
    CLion-Installation kopiert werden und anschließend in der Plugin-Liste ausgewählt werden. Hier wird auf Abbildung
    \ref{fig:pluginlistinstall} verwiesen.

    \begin{figure}[h]
        \centering
        \caption{Plugin Liste mit hinzugefügtem Plugin}
        \label{fig:pluginlistinstall}
        \includegraphics[scale=0.5]{../Graphiken/plugininstall}
    \end{figure}

    Um die Funktionalität des Plugins zu aktivieren ist ein Neustart der IDE nötig.

    \subsection{Source Code}
    Der mitgelieferte Source-Code kann bei Bedarf selbst kompiliert und ausgeführt werden.
    Da IntelliJ keine Modulerkennung für Plugins unterstützt kann externer Plugin-Source-Code nicht ohne Weiteres
    importiert und ausgeführt werden.
    Um den Source-Code trotzdem selbst ausführen zu können muss über
    \href{https://www.jetbrains.com/idea/}{IntelliJ} zunächst ein \underline{neues} Plugin-Projekt angelegt werden.
    Ist beim Project-Wizard keine Plugin-Option verfügbar, muss zunächst die \textit{Plugin-Development}-Erweiterung
    von Jetbrains heruntergeladen und aktiviert werden (\texttt{Settings} $\rightarrow$ \texttt{Plugins}).

    Wurde ein neues Plugin-Projekt angelegt, muss das korrekte SDK ausgewählt werden. Die Einstellung erfolgt unter
    dem Menüpunkt \texttt{File} $\rightarrow$ \texttt{Project Structure} $\rightarrow$ \texttt{Project}.
    Danach \texttt{New...} und anschließend \texttt{IntelliJ Plattform Plugin SDK} auswählen und auf den CLion
    Installationsordner zeigen.

    Zuletzt muss der gesamte Source-Code (\texttt{./src}) in den Source-Ordner und alle Ressourcen (./resources) in
    den Ressourcenordner des neu angelegten Projekts kopiert werden. Die \texttt{plugin.xml}-Datei soll dabei
    überschrieben werden.

    Nach diesen Schritten ist das Plugin-Projekt bereit zur Ausführung.