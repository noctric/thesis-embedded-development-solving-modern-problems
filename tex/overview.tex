\section{Einführung in Embedded-Systeme}
		\subsection{Arduino}
		\label{sec:arduino}
		Da Arduino mitverantwortlich für den Anstieg des Interesses an Microcontroller-Programmierung in der
		Entwicklerszene ist, wird im Folgenden eine kurze Übersicht über die angebotene Technologie geliefert.

		Wie oben schon aufgeführt, ist Arduino eine Plattform für die Entwicklung mit Mikrocontrollern. Dabei werden
		sowohl Software- als auch Hardware-Komponenten angeboten, um Embedded-Programmierung zu betreiben.

		Es ist ein kostenfreies \glsit{glo:ide} verfügbar, über die sogenannte \glsplit{glo:sketch} entwickelt
		werden können. Über \glsit{glo:usb} werden diese Programme auf das Board heruntergeladen.
	    Außerdem stehen viele Ressourcen und Programmbeispiele online sowie in zahlreichen Lehrbüchern zu Verfügung.
	    Programmiert wird mit einer reduzierten Version der Sprache C. Um einen einfachen \glsit{glo:sketch} zu schreiben,
	    müssen lediglich die beiden Funktionen \texttt{setup()} und \texttt{loop()} implementiert werden.

        Die meisten Prozessoren der Arduino-Hardware stammen aus der Atmel-AVR Familie.
        \\

	    Während Arduino die allgemeine Entwicklung mit Mikrocontrollern sehr stark vereinfacht, sind diese
	    Vereinfachungen oft mit Kosten verbunden.
	    \\

	    Die vereinfachte Sprache und das Framework insgesamt ermutigt den Nutzer den gesamten Code in einen \glsit{glo:sketch}, also
	    eine Datei zu schreiben. Dies ist für viele Programmierer das genaue Gegenteil von dem, was in der Theorie
	    gelehrt wird: Logische Aufteilung von Programmcode in Module, sprich Modularisierung.
	    Möchte man trotzdem ein modularisiertes Programm entwerfen, wird der Nutzer gezwungen auf andere Sprachen und
	    Werkzeuge zurückzugreifen (zum Beispiel \textit{Atmel Studio}).

		Weiterhin ist das Debuggen von Firmware nur mit entsprechender Hardware möglich (siehe \ref{sec:jtag}). Wichtige
		Funktionalitäten, wie zum Beispiel Breakpoints, werden dem Arduino-Nutzer (angenommen man verwendet das
		Arduino-Software-Framework) vorenthalten. Hier wird in der Regel über serielle Ausgaben, ähnlich wie \texttt{printf()},
		ein Fehler gesucht.

		Arduino vereinfacht also den Einstieg in die Entwicklung von Mikrocontrollern enorm, aber macht dafür eine
		etwas anspruchsvollere Entwicklung um einiges schwieriger.

		\subsection{Raspberry Pi}
        Raspberry Pi ist im Vergleich zu Arduino keine gesamte Entwicklungsplattform, sondern bezeichnet lediglich
        einen \textit{Single Board Computer}, also einen kleinen (minimalen) Heim-Computer. Dieser verfolgt das Ziel eine
        einfache Anlaufstelle für das Experimentieren mit Computer-Systemen, vor allem für Schüler und
        Studenten, zu schaffen.

        Die Prozessoren variieren je nach Modell, gehören aber meistens der \glsit{glo:arm}-Cortex-A-Familie an.

        Da der Raspberry Pi einen Heimcomputer darstellt, wird in der Regel ein vollständiges und meistens angepasstes
        Betriebssystem (beispielsweise \textit{Raspbian}, ein auf Debian basierendes Betriebssystem) installiert.

        Mit Peripheriegeräten, welche sich nicht direkt auf dem Board befinden, kann über die \glsplit{glo:gpio}
        kommuniziert werden.

        Die Entwicklung mit dem Raspberry Pi ist durch die Abstraktion mit Betriebssystem nicht sehr Hardware-nah.
        Außerdem bietet der Raspberry Pi im Vergleich zu Personal Computers einen deutlichen Preisvorteil, ist jedoch
        kostspieliger als andere Mikrocontroller.

        Für die meisten Anwendungsbereiche innerhalb der Entwicklung für Embedded-Systeme ist ein Raspberry Pi
        letztendlich ein Übermaß, da die meisten Projekte einen sehr spezifischen Fall oder eine sehr spezifische
        Funktion erfüllen müssen und das Raspberry Pi aber darauf ausgelegt ist mit einem Betriebssystem zu
        laufen und somit die Komplexität deutlich erhöht wird. In Abschnitt \ref{sec:os} wird genauer auf die Thematik
        von Betriebssystemen bei Embedded-Systemen eingegangen.

	\subsection{Die ARM-Cortex-Prozessorfamilien}
	\label{sec:cortex_fam}
	In der \glsit{glo:arm}-Cortex-Prozessorfamilie gibt es 3 grundlegende Subklassen.
	\begin{description}
	    \item[Cortex-A] Das A steht für \textit{Application} und beschreibt die auf Hochleistung fixierte Prozessorfamilie.
	    Sie werden meistens in mobilen Geräten (Smartphones, Tablets, etc...) eingesetzt und können mit einer
	    relativ hohen Taktung betrieben werden.
	    ``While the Cortex-A processors have high performance, the processor is not designed to
          provide rapid response time to hardware events (i.e., real-time requirements)''\citep[Kap.~0]{Yiu2015}
	    \item[Cortex-R] Prozessoren der Cortex-R-Familie sind ebenfalls als Hochleistungsprozessoren gedacht und
	    vor Allen für Echtzeitsysteme entwickelt. Sie garantieren ein deterministisches Verhalten und werden in
	    Systemen verwendet, welche zuverlässige Antworten in Echtzeit benötigen
	    \footnote{siehe \href{https://www.arm.com/products/processors/cortex-r}{ARM Produkt-Seite}}.
	    Praktische Verwendung finden die Prozessoren beispielsweise in der Automobilbranche, im industriellen Bereich
	    und in der Medizintechnik.
	    Die Prozessoren gelten aufgrund ihrer komplexen Architekturen als nicht sehr energieeffizient
	    \citep[vgl.~Kap.~0]{Yiu2015}.
	    \item[Cortex-M] Cortex-M-Prozessoren besetzen den \glsit{glo:mainstream}-Markt für Mikrocontroller und gelten als
	    energieeffizient. Sie besitzen eine kürzere Pipeline zum Abarbeiten von Befehlen und laufen mit einer geringeren
	    Taktung. Sie zeichnen sich außerdem hauptsächlich durch ihre Benutzerfreundlichkeit, kurze Antwortzeit und ihren
	    geringen Platzaufwand aus.\\
	    Der eingebaute \glsit{glo:nvic} bietet ein mächtiges und doch einfach zu nutzendes
	    Interrupt Management\citep[vgl. Kap. 0]{Yiu2015}.
	    Cortex-M-Prozessoren werden am häufigsten in Embedded-Systemen eingesetzt und besitzen eine Variation von
	    verschiedenen Modellen für Anwendungsgebiete deren Anforderungen von Energieeffizienz und minimalen Kosten bis
	    zu Hochleistung reichen.
	    Abbildung \ref{fig:arm_cortex_m_rel_performance} zeigt die Leistungsunterschiede der Cortex-M-Modelle relativ
	    zum Cortex-M0-Prozessor-Typ.

%TODO CORMARK TEST
	    \begin{figure}[h]
	        \caption{Relative Leistung der verschiedenen Cortex-M Modelle}
	        \source{
	            \href{
	                https://www.arm.com/-/media/arm-com/products/processors/Cortex-M-series-performance-graph.jpg?la=en
	            }{
	                ARM Produkt Seite
	            }, Zugriff: 17.08.2017
	        }
	        \label{fig:arm_cortex_m_rel_performance}
            \centering
            \includegraphics[scale=0.3]{../Graphiken/Cortex-M-series-performance-graph.jpg}
	    \end{figure}

	\end{description}

	\subsection{Erläuterung allgemeiner Begriffe}
	Im folgenden Abschnitt werden häufig verwendete Begriffe kurz erläutert, um Missverständnisse bei
	Formulierungen vorzubeugen und den späteren Kontext besser verstehen zu können.
	    \begin{description}
	        \item[Prozessor] Der Prozessor ist ein programmierbarer digitaler integrierter Schaltkreis und gilt als
	        innerstes Modul eines Systems. Er übernimmt die Ausführung von Maschinenbefehlen und steuert andere
	        elektrische Schaltungen\citep[vgl.~Kap.~1]{Asche2017}.
	        \item[Mikrocontroller] Ein Mikrocontroller ist ein zusammenfassender Begriff für Prozessoren, Speicher und
	        Ein- und Ausgabeperipherie.
	        \item[SoC] \glsit{glo:soc} bezeichnet die Zusammenfassung oder Integration von
	        mehreren Funktionen eines programmierbaren Systems auf einem integrierten Schaltkreis. Auch Mikrocontroller
	        lassen sich als \glsit{glo:soc} bezeichnen.
	        \item[RAM] Bei \glsit{glo:ram} kann auf jede Speicherzelle über die Speicheradresse direkt
	        zugegriffen werden. Dabei ist das als Arbeitsspeicher verwendete \glsit{glo:dram} ein
	        flüchtiger Speicher. Im Gegensatz dazu ist \glsit{glo:sram} nicht flüchtig und wird in Caches, also Zwischenspeicher,
	        genutzt. Der Zugriff auf diesen Speicher erfolgt im Gegensatz zu Massenspeichern schnell.
	        \item[ROM] \glsit{glo:rom} bezeichnet einen permanenten Speicher. Der Zugriff auf diesen
	        Speicher erfolgt im Vergleich zum \glsit{glo:ram} langsam.
	        \item[Instruction Set] Als \textit{Instruction Set} wird die Menge der vom Prozessor verstandenen Anweisungen
	        bezeichnet. Anweisungen können den Kontrollfluss verändern, Daten aus dem Speicher lesen, schreiben oder
	        manipulieren\citep[]{Asche2017}.
	        \item[Firmware] Als Firmware wird das vom Prozessor auszuführende Programm bezeichnet. Dabei kann es, falls
	        vorhanden, vom Arbeitsspeicher oder vom Flash-Speicher ausgeführt werden. Das Programm wird dabei als
	        \textit{Image} in den Speicher geladen und ausgeführt.
	        \item[Peripherie] Alle externen Geräte und Anschlüsse, die sich auf dem Board bzw. innerhalb des
	        eingebetteten Systems befinden und mit dem Prozessor kommunizieren (siehe auch \textit{Pins}).
	        \item[Pin] Pins sind Kontakte, die zum Beispiel vom Prozessor für die Kommunikation mit der ``Außenwelt''
	        genutzt werden. Die Pins eines Mikroprozessors können in der Regel mehrere Funktionen übernehmen. Beim
	        Hochfahren muss dem Prozessor mitgeteilt werden, wie er bestückt ist und wie er deswegen die Pins
	        behandeln muss\citep[vgl.~Kap.~1.3]{Asche2017}.
	        \item[Serielle Schnittstelle] Ein serieller Datenfluss bezeichnet das Versenden von Daten in Form von
	        aufeinanderfolgenden Bits. Das Gegenstück zu seriell ist die parallele Übertragung von Daten, in der über
	        mehrere Datenströme parallel gesendet wird.
	        \item[Flash-Speicher] Ein nicht-flüchtiger Speicher, welcher elektronisch gelöscht und programmiert werden
	        kann.
	        \item[RTOS] \glsit{glo:rtos} bezeichnet ein Betriebssystem mit
	        Echtzeitanforderungen. Das heißt, es muss innerhalb eines bestimmten Zeitraums antworten und Daten ohne
	        Pufferverzögerungen bearbeiten.
	        Im Embedded\hyp{}Bereich kommen Echtzeitbetriebssysteme oft zum Einsatz, da sie ``im Gegensatz zu anderen
	        Betriebssystemen dem Entwickler einen hohen Grad der Kontrolle darüber, welche Ereignisse mit welcher
	        Priorität behandelt werden können[, geben]''\citep[Kap.~3.1]{Asche2017}.
	        Genaueres wird in Abschnitt \ref{sec:os} erläutert.
	        \item[Interrupt] Ein Interrupt bezeichnet ein unterbrechendes Ereignis.
	        Diese spielen vor allem bei Echtzeitsystemen eine große Rolle. Sie signalisieren der Hardware sich sofort
	        auf ein anderes Ereignis zu konzentrieren. Der Prozessor unterbricht daraufhin alle Aktivitäten und
	        wendet sich durch einen \textit{Interrupt Handler} dem besagtem Ereignis zu.
	        ``In vielen Prozessoren genießen Interrupts einen privilegierten Status, indem sie in einer
              höheren Sicherheitsstufe und möglicherweise in einem anderen Kontext als Applikations-Code ausgeführt
              werden.''\citep[Kap.~1.5]{Asche2017}
	        \item[Host] Als Host wird die Instanz bezeichnet, von der das Programm auf die Zielhardware gespielt wird.
	        Hierbei handelt es sich meistens um den Entwicklungsrechner.
	        \item[Target] Die Zielhardware wird als Target (also ``Ziel'') bezeichnet.
	        \item[PCB] \glsit{glo:pcb} ist englisch für Leiterplatte. Ein \glsit{glo:pcb} enthält
	        mehrere \glsplit{glo:ic} oder andere elektronische Bauteile, die miteinander Verbunden sind.
	    \end{description}
	\subsection{Betriebssysteme}
	\label{sec:os}
	Wie in Abschnitt \ref{sec:cortex_fam} erwähnt, gibt es für verschiedene \glsit{glo:arm}-Cortex-Prozessorserien unterschiedliche
	Anwendungsgebiete. Damit stellt sich auch die Frage, welches Betriebssystem am besten geeignet ist und ob
	überhaupt ein solches benötigt wird.

	Die Aufgabe eines Betriebssystems ist es die Hard- und Software eines Systems zu verwalten. Es stellt eine
	Abstraktionsebene zwischen System-Hardware und Applikationen dar und ist für die Bereitstellung von verschiedenen
	Diensten zuständig. Dabei arbeitet das Betriebssystem logisch unterteilte Aufgaben ab. Diese können unabhängig
	voneinander ausgeführt und, falls nötig, bei Abschluss wieder miteinander synchronisiert werden.

	Bei Embedded-Systemen ist die Frage, ob und welches Betriebssystem benötigt wird, oft mit mehreren Faktoren
	verbunden. Ein großer Entscheidungspunkt ist die zur Verfügung stehende Hardware. Oft wird bei der Entwicklung
	von Embedded-Systemen versucht so viel wie möglich an Kosten zu sparen, d.h. die Hardware wird oft so minimal wie
	möglich gestaltet. Da ein Betriebssystem durch seine Abstraktion und Bereitstellung von verschiedenen Diensten und
	Programmen großen \glsit{glo:overhead} erzeugt, gestaltet sich die Benutzung eines Betriebssystems bei
	Embedded-Systemen oft als problematisch.
	``Wenn also ein Abstraktionslayer zwischen der konkreten Implementation und dem sie benutzenden Code liegt, bedeutet
	das oft größeren Stack- und CPU-Zeitbedarf allein durch den Aufruf der Abstraktionsfunktion, der bei knapp
	geschneiderten Plattformen durchaus schmerzhaft sein kann!''\citep[Kap.~2.2]{Asche2017}.

	Prinzipiell lässt sich also die Regel aufstellen, dass eine ``einfache'' und nicht parallel laufende Anwendung auf
	einem Embedded-System kein Betriebssystem benötigt, da die meisten Dienste des Betriebssystems nicht gebraucht werden
	und somit unnötigen \glsit{glo:overhead} erzeugen.

	Auch komplexere Anwendungen können vollständig ohne Betriebssystem laufen. Die normalerweise von einem
	Betriebssystem logisch unterteilten und abgearbeiteten Teilaufgaben werden von der eigens entwickelten Firmware
	übernommen,
	``in der Praxis [wird] in diesen Fällen normalerweise die Gliederung in Teilaufgaben von im Hause entwickelten
	Komponenten wie message pumps [...] übernommen [...]. Diese Komponenten lassen sich bereits als kleine
	Betriebssysteme bezeichnen''\citep[Kap.~3.2]{Asche2017}.

	Anders ist die Lage bei mobilen Geräten, wie zum Beispiel Smartphones und Tablets. Dieser Anwendungsbereich
	erfordert viele Funktionen eines Betriebssystems, wie zum Beispiel ein File-System, Speicherverwaltung und Weiteres.
	Die hier verwendeten leistungsfähigen Prozessoren der Cortex-A-Serie bieten sich deshalb für
	die Verwendung eines Betriebssystems, beispielsweise Linux, iOS, Android, etc an.

	Für Echtzeitsysteme existieren die oben schon erwähnten \glsplit{glo:rtos}. Diese sind in frei
	verfügbaren Ausführungen, wie z.B. FreeRTOS, vorhanden oder auch kostenpflichtig verfügbar.

	Ein Echtzeitsystem ist dadurch definiert, dass es genau festgelegte Zeitschranken für Prozesse erfüllen muss, um
	ernsthafte Konsequenzen (wie zum Beispiel ein Systemausfall) oder Misserfolge zu vermeiden. Als Misserfolg wird
	jedes Systemverhalten definiert, welches die formale Systemspezifikation nicht erfüllt
	\citep[vgl.~Kap.~2.2.5]{Brause2017}.

    Zwei Verfahren, die beim dynamischen Prozess-Scheduling für Echtzeitsysteme verwendet werden, sind Polling und
    Interrupts\footnote{\citep[vgl.~Kap.~2.2.5]{Brause2017}}:
    \begin{description}
        \item[Polling] Der Prozessor fragt alle verbundenen Geräte in einer Schleife nacheinander ab, ob ein Ereignis
        oder neue Daten vorhanden sind. Tritt ein dringendes Ereignis während der Bearbeitung eines anderen Ereignisses
        auf, scheitert diese Strategie.
        \item[Interrupt] Der Prozessor befindet sich in einer \textit{Idle Loop} und kann von
        \glsplit{glo:irq} unterbrochen werden. Dabei arbeitet die \glsit{glo:isr} das
        unterbrechende Ereignis ab, wobei der entsprechenden \glsit{glo:isr} auch eine Priorität zugewiesen werden kann. In diesem
        Fall findet zusätzlich ein Prioritätsscheduling statt. Hierbei liegt die Problematik bei einer Anhäufung von
        Interrupts mit hoher Priorität, was dazu führen kann, dass Interrupts mit geringer Priorität ignoriert oder
        überschrieben werden.
    \end{description}

    Für den restlichen Verlauf der Arbeit wird angenommen, dass kein Betriebssystem benötigt wird.

    \subsection{Interrupthandling bei ARM-Cortex-Prozessoren}
    Interrupts stellen ein Kernkonzept der Firmware-Entwicklung dar. Durch sie kann der geregelte Programmablauf
    beeinflusst werden. Ein Interrupt kann genutzt werden, um einen Dienst vom Prozessor anzufordern (zum Beispiel von
    einem Peripheriegerät).
    Folgender Ablauf ergibt sich bei dem Senden eines Interrupts\footnote{\citep[vgl.~Kap.~7]{Yiu2013}}:
    \begin{enumerate}
        \item Es wird ein \glsit{glo:irq} an der Prozessor geschickt
        \item Der Prozessor suspendiert den aktuellen Task (bei geringerer Priorität)
        \item Der Prozessor arbeitet den mit dem Interrupt gekoppelten \textit{Interrupt Handler} (also eine Funktion) als
        \glsit{glo:isr} ab
        \item Der Prozessor führt nach Abschluss den ursprünglichen Task aus
    \end{enumerate}

    Verantwortlich für das Verarbeiten der Interrupts ist der \glsit{glo:nvic}, der für die
    Abarbeitung von Interrupts und Exceptions zuständig ist, wobei in den Architekturen der \glsit{glo:arm}-Cortex-Serien Interrupts
    als Exceptions gelten. Die verschiedenen \textit{Interrupt Handler} sind in der \glsit{glo:ivt} gelistet,
    welche am Anfang eines jeden Programms initialisiert und an der korrekten Stelle im Speicher hinterlegt werden muss.
    Bei den meisten Chip-Architekturen der \glsit{glo:arm}-Cortex-M-Serie bietet diese \glsit{glo:ivt} Platz für 256 Interrupt-Vektoren.

    \begin{table}[h]
        \centering
        \caption{Beispiel für eine \glsit{glo:ivt}}
        \label{tab:itvexample}
        \begin{tabular}{|c|c|}
            \hline
            Interrupt-Nummer    & Speicheradresse des \glsit{glo:isr}       \\ \hline
            1                   & Interrupt-Service-Routine 1   \\ \hline
            2                   & Interrupt-Service-Routine 2   \\ \hline
            3                   & Interrupt-Service-Routine 3   \\ \hline
            $\vdots$            & $\vdots$                      \\ \hline
        \end{tabular}
    \end{table}

    Tabelle \ref{tab:itvexample} zeigt eine beispielhafte \glsit{glo:ivt}. Jeder Eintrag besteht aus einer Interrupt Nummer und
    einem Funktions-Pointer, der auf den \textit{Interrupt Handler} zeigt.

    Jedem Interrupt kann eine Priorität zugewiesen werden. Dabei können diese zwischen 0 und 255 (8 Bit) liegen.
    Hierbei ist Priorität 0 die höchste und 255 die geringste. Wie viele programmierbare Prioritäten letztendlich
    möglich sind, entscheidet der Chip-Hersteller. Eine hohe Anzahl an programmierbaren Prioritäten kann die Komplexität
    eines \glsit{glo:nvic}s, sowie auch den Stromverbrauch erhöhen und gleichzeitig die Leistung bzw. die Geschwindigkeit der
    Architektur verringern\citep[vgl.~Kap.~7.4]{Yiu2013}.

    Die ersten 16 Einträge der \glsit{glo:ivt} sind für den Prozessorkern reserviert. Sie besitzen feste negative Prioritäten.
    Ein Beispiel ist der Vektor 12 bei Adresse \texttt{0x0000\_0030}, der den \textit{Debug Monitor Interrupt Handler} enthält.
    Dieser wird von Debuggern zum Verwalten von Breakpoints benutzt, kann aber auch durch die Maschinenanweisung
    \texttt{bkpt} erzwungen werden\citep[vgl.~Kap.~2.8]{Asche2017}.
    Die restlichen Einträge werden vom Prozessorhersteller definiert.

    Die Stelle, an der sich der gesuchte Interrupt-Vektor im Speicher befindet, lässt sich mit Hilfe der
    Interrupt-Nummer berechnen. Standardmäßig fängt die \glsit{glo:ivt} bei Adresse 0 (also \texttt{0x0} in hexadezimal) an und die
    Adresse eines Vektors lässt sich durch $x \cdot 4$ errechnen, wobei $x$ die Interrupt-Nummer ist
    \citep[vgl.~Kap.~7.5]{Yiu2013}.\\
    \textbf{Beispiel:} Der erste nicht reservierte Interrupt besitzt die Nummer 16. Somit liegt der Funktions-Pointer
    für den Interrupt Handler bei der Adresse $(16 \cdot 4)_2 = (64)_2 = (40)_{16}$, also \texttt{0x40}. Je nach
    Architektur kann diese Rechnung variieren.

	\subsection{Allgemeine Voraussetzungen zur Entwicklung von Embedded\hyp{}Systemen}
	\label{sec:requirements}
	Ohne Betriebssystem tun sich allerdings im Hinblick auf die Entwicklung einige Schwierigkeiten auf.

    Abgesehen davon, dass ein Entwickler die Prozessorarchitektur genau kennen muss, stellt sich die Frage, wie denn eine
    entwickelte Firmware aufgespielt werden und eine Fehlersuche stattfinden kann.

    Das Programm-Image muss an die richtige Startadresse des Speichers geschrieben werden, damit
    der Prozessor alle im Programm vorkommenden Befehle in korrekter Reihenfolge abarbeiten kann.
    \\\textbf{Beispiel:} Beim \texttt{ATSAM E70 Q21} lautet diese Adresse \texttt{0x00400000}.

    Besagte Startadresse geht aus der \glsit{glo:ivt} hervor. Das Herunterladen der
    Firmware in den Speicher kann von Hand implementiert werden, es bieten sich allerdings eine Reihe an Werkzeugen an,
    welche diese Aufgabe übernehmen.

    Außerdem erfordert die Abwesenheit eines Betriebssystems eine Schnittstelle um die installierte Firmware
    auf Fehler untersuchen zu können. Hier gibt es verschiedene Möglichkeiten:
    \begin{itemize}
        \item Target-seitig ist ein Debug-Interface vorhanden und kann über \glsit{glo:usb} angesprochen werden. Genaueres wird in
        Abschnitt \ref{sec:cmsisdap} erklärt.
        \item Externe Debugging-Hardware. Diese Hardware konvertiert Befehle, die über \glsit{glo:usb} gesendet werden,
        in \acrshortit{glo:jtag}- und \acrshortit{glo:swd}-Befehle. Genaueres hierzu in Abschnitt \ref{sec:jtag} und \ref{sec:swd}.
        \item Eine weitere Debugging-Technik ist \glsit{glo:rtt}. Hierbei nimmt ein \glsit{glo:rtt}-Modul alle Aktivitäten
        des Prozessors zur Laufzeit der Programms auf, zum Beispiel Lese- und Schreibzugriffe auf den Speicher. Mit den richtigen
        Werkzeugen kann dann post mortem eine Analyse des Programms stattfinden. Dies ist besonders bei
        Echtzeitsystemen von Nutzen\citep[vgl.~S.~3]{Campbell2014}. Genaueres hierzu in Abschnitt \ref{sec:trace}.
    \end{itemize}

    Die Fehleranalyse bei Embedded-Systemen unterscheidet sich stark von dem für einen Software-Entwickler gewohnten
    Debug-Prozess. Beispielsweise muss bei Breakpoints zwischen Hard- und Software Breakpoints unterschieden werden und
    darauf geachtet werden, wie viele Hardware Breakpoints vom Chip unterstützt werden. Wie diese Breakpoints
    funktionieren und wann man welche Art verwenden kann wird in Abschnitt \ref{sec:debugfeatures} erklärt.

    Doch bevor es zur Fehleranalyse kommt, muss die Firmware erst einmal entwickelt werden. Dazu benötigt man vorerst einen
    Compiler. Das Programm kann in jeder Programmiersprache geschrieben werden, die einen Compiler für die richtige
    Chip-Architektur besitzt und Binärdateien erzeugt. Üblicherweise wird hier C bzw. C++ genutzt, andere Sprachen werden
    wegen ihrer zu hohen Abstraktion oft nicht in Erwägung gezogen. Natürlich besteht auch die Möglichkeit, direkt
    in Assembler zu programmieren.

    ``Im Vergleich zu C und C++ „höhere“ Programmiersprachen wie Java spielen sich auf einer höheren Abstraktionsebene
    ab, das heißt sie verstecken bewusst Details wie Speicherverwaltung, Datenorganisation oder Hardware-Ansteuerung vor
    dem Entwickler. Diese erst einmal nicht schlechte Eigenschaft steht aber im direkten Gegensatz zu den Anforderungen,
    die an eingebettete Systeme gestellt werden.''\citep[Kap.~1.1]{Asche2017}

    Ein weiterer Grund für die Verwendung von C als Programmiersprache für die Entwicklung eingebetteter Systeme ist die
    Bereitstellung von C-Header-Dateien und Treiberbibliotheken durch einen Großteil an Chip-Hersteller.

    Für den weiteren Verlauf dieser Arbeit wird angenommen, dass die Firmware in der Sprache C entwickelt wird, da dies
    zum heutigen Zeitpunkt die relevanteste Sprache im Embedded-Bereich ist.

    Betrachtet man nun die bisher erarbeiteten Voraussetzung zur Entwicklung eines Embedded-Systems, kommt man zu
    folgender Liste:
    \begin{itemize}
        \item Host-PC zum Entwickeln der Software
        \begin{itemize}
            \item Cross-Compiler
            \item \acrshortit{glo:usb}-Verbindung zur Zielhardware
            \item Software zum Aufspielen der Firmware
            \item Software zur Fehleranalyse (genaueres hierzu in Abschnitt \ref{sec:gdb})
        \end{itemize}
        \item Debugging-Hardware
        \begin{itemize}
            \item Debug-Chip auf dem Board mit \acrshort{glo:cmsisdap}-Interface (siehe Abschnitt \ref{sec:cmsisdap}) \textbf{oder}
            \item Debug-Probe zum umwandeln von \acrshortit{glo:jtag}\hyp{}/\acrshortit{glo:swd}\hyp{}Befehlen (siehe\\Abschnitt \ref{sec:jtag} und \ref{sec:swd})
        \end{itemize}
        \item Zielhardware
        \begin{itemize}
            \item Das \glsit{glo:datasheet} liefert einen umfangreichen Überblick über den Aufbau und die Funktionalitäten des
            Prozessors
        \end{itemize}
    \end{itemize}
		\subsubsection{Einrichtung der Entwicklungsumgebung}
        Die Auswahl der richtigen Werkzeuge für die Entwicklung von Embedded-Systemen ist oft eine herausfordernde
        Aufgabe. Die Anzahl an verfügbaren Werkzeugen wirkt überwältigend und unübersichtlich.
        Zunächst ist die Frage zu klären, mit welcher Hardware man arbeiten wird. Viele Chip-Hersteller bieten online
        einen Assistenten an, der anhand von ausgewählten Kriterien die große Auswahl an verschiedenen Prozessoren auf
        die für den Nutzer relevanten beschränkt.

        Es besteht auch die Möglichkeit ganze Development Boards zu kaufen, welche zusätzliche Peripherie und
        zum Beispiel eine Debug-Einheit enthalten.

        Wofür man sich letztlich entscheidet, hängt von den Anforderungen des Projekts ab.

        In der Regel ist es üblich, sich erst einmal für einen Chip-Hersteller zu entscheiden (am besten für ein
        konkretes Modell) und anschließend die vom Hersteller bereitgestellten Werkzeuge zu nutzen.

        Wie in Abschnitt \ref{sec:requirements} bereits erwähnt benötigt man für die Firmware-Entwicklung auf Seiten des Hosts
        folgende Komponenten:
        \begin{description}
            \item[Cross-Compiler] Der geschriebene Source\hyp{}Code muss spezifisch für die Chip\hyp{}Architektur kompiliert werden.
            \item[Programmiergerät] Der kompilierte Source-Code in Form einer Binärdatei muss auf den Speicher des
            Prozessors heruntergeladen werden.
            \item[Debugger] Bei selbst geschriebener Firmware ist die Fehlersuche im Code so gut wie unausweichlich.
            Hierbei helfen einem Debug-Funktionalitäten wie Breakpoints, um Fehler im Code zu finden.
        \end{description}

        Verwendet man unterschiedliche Programme für die oben genannten Prozesse, bezeichnet man die Sammlung dieser
        Programme als \glsit{glo:toolchain}. Die Komplexität besteht nun darin, herauszufinden welche Werkzeuge mit der
        Zielhardware kompatibel sind und welche Werkzeuge am besten miteinander funktionieren.

        Atmel bietet beispielsweise ein Visual-Studio-Plugin mit dem Namen \textit{Atmel Studio} an. Genanntes Plugin
        funktioniert mit vielen von Atmel hergestellten Chips und unterstützt diverse Debugging-Features, wie zum Beispiel
        \textit{Trace} oder \glsit{glo:cmsisdap}. Auch das Aufspielen von Firmware funktioniert problemlos.
        Das Plugin erfüllt also alle oben genannten Voraussetzungen für die Entwicklung eines Embedded-Systems auf
        Seiten des Hosts, sofern das Board entsprechend mit debug-fähiger Hardware ausgestattet ist. Wie diese Hardware
        aussieht und funktioniert wird im Abschnitt \ref{sec:protocols} erarbeitet.

        Da Atmel Studio ein Plugin der von Microsoft entwickelten IDE \textit{Visual Studio} ist, bindet es den
        Entwickler an die Windows-Plattform. Für Entwickler auf anderen Systemen, wie zum Beispiel Linux oder Mac ist
        dies ohne Virtualisierung keine Option. Des weiteren kommt mit der benötigten Installation von Visual Studio ein
        großer \glsit{glo:overhead}, da dieses Programm sehr umfassend ist und somit die Entwicklungsumgebung deutlich verlangsamt.

        Eine Open-Source-Lösung zum Programmieren und Debuggen von Firmware bietet das Programm \glsit{glo:openocd}
        \footnote{Das Projekt wurde 2005 im Rahmen einer Masterarbeit von Dominic Rath
        angestoßen und seitdem von diversen Mitwirkenden weiterentwickelt}.
        Es unterstützt sowohl Windows als auch Unix-ähnliche
        Systeme. Lediglich der richtige Cross-Compiler fehlt, und muss separat vom Entwickler betrieben werden.

        Für die Produktion bietet sich das kostenpflichtige Toolset von Keil an, welches eine IDE mit
        Debugging-Unterstützung enthält. Keil ist eine der wenigen Lösungen, die eine sehr große Menge an
        Chip-Architekturen von verschiedenen Herstellern unterstützt.

	\subsection{Negative Aspekte des Enticklungsprozesses von Embedded\hyp{}Systemen}
	\label{sec:probleme}
	Durch eine genauere Betrachtung des Entwicklungsprozesses zeichnen sich mehrere Hürden und Probleme ab, die es zu
	bewältigen gilt:
	\begin{itemize}
	    \item Für Neueinsteiger in den Bereich der Embedded-Systeme, seien es bereits ausgebildete Software-Entwickler
	    oder Anfänger, ist die Entwicklung mit einer steilen Lernkurve verbunden. Teilweise verantwortlich
	    dafür ist die große Anzahl an unterschiedlicher Hardware und verschiedene Entwicklungsumstände (zum Beispiel
	    auf welchem Host-System entwickelt wird).
	    \item Fehlende oder nicht eingehaltene Standards zwischen Chip-Herstellern sind dafür verantwortlich, dass
	    Applikations-Code schwieriger zu portieren ist. Das heißt auch, dass bei einem Wechsel der Hardware ein neues
	    Interface studiert und in den bestehenden Applikations-Code integriert werden muss.
	    \item Der vorherige Punkt hat zu Folge, dass Entwickler oft an einen Hersteller gebunden sind, da die
	    \glsit{glo:hal} und somit auch die Treiber herstellerspezifisch sind. Ein Wechsel der
	    Hardware ist also mit größeren Umständen verbunden.
	    \item Auch besitzt fast jeder Hersteller seine eigene Sammlung an Enwicklungswerkzeugen, zusätzlich zu anderer
	    Software, die von Drittanbietern angeboten wird. Bestimmte Entwicklungswerkzeuge binden den Entwickler an
	    Plattformen, wohingegen frei verfügbare Software oft unübersichtlich gestaltet ist und somit vor allem für
	    Anfänger eine große Hürde darstellt.
	    \item Bestehende Abstraktionen auf einer höheren Ebene, wie zum Beispiel das Wiring-Projekt (unterstützt nur
	    AVR Prozessoren) \footnote{\href{http://www.wiring.org.co/}{Offizielle Seite \texttt{wiring.org.co}}}, binden den Entwickler an
	    bestimmte Hardware und schränken die Möglichkeiten der Erweiterung und Leistung von Hardware ein. Bei der
	    Wiring-Plattform ist dies dem Ziel geschuldet, möglichst einfaches und sicheres Hardware-Prototyping für unerfahrene
	    Entwickler anzubieten. Die Menge an Funktionalitäten um potenzielle Fehlerquellen abzufangen resultiert in
	    starken Leistungseinbußen.\\
	    \textbf{Beispiel:} Arduino-Shields sind Hardware-Erweiterungen, die auf ein Arduino-\glsit{glo:pcb} gebaut werden können.
	    Dabei müssen die Shields über unterschiedliche Pins angesteuert werden. Dies beschränkt die möglichen
	    Erweiterungen.
	    \item Es herrscht ein Mangel an effizienten und modernen Entwicklungsumgebungen, die ein Software-Enwickler
	    wahrscheinlich gewohnt ist, mit Unterstützung für Embedded-Systeme. Bereits erwähnte \glsplit{glo:ide}, wie zum Beispiel
	    Keil oder Atmel Studio sind entweder an eine Plattform und einen Hersteller gebunden oder nicht frei verfügbar.
	\end{itemize}

    Die aufgeführten Punkte stellen gleichzeitig die Anforderungen an eine Software-Plattform zur Vereinfachung der
    Entwicklung im Embedded-Bereich und werden in Abschnitt \ref{sec:demands} formuliert.
