\section{Verwendung des Plugins}
Das Plugin definiert drei Aktionen: CMake Konfiguration generieren, OpenOCD Daemon starten und Programm-Image auf die
Ziel-Hardware spielen.

Alle drei Aktionen finden sich zum Einen im Hauptmenü unter dem Menüpunkt \texttt{\underline{E}mbedded Systems} und zum
Anderen in der Seitenleiste unter dem Toolwindow \texttt{Embedded Dev} wieder.

    Um die Toolchain einzurichten, müssen bestimmte CMake-Konfigurationen vorgenommen werden. Hierzu generiert das
    Plugin eine Datei, mit der die Toolchain ausgewählt und geprüft wird.

    Die Konfigurationsdatei muss als Toolchain-Datei für CMake definiert werden. Da das Plugin (zu jetzigem Stand)
    keinen Weg bereitstellt CMake-Optionen zu setzen, muss der Benutzer selber die CMake-Option \\
    \texttt{-DCMAKE\_TOOLCHAIN\_FILE=arm\_toolchain.cmake} unter dem Menüpunkt \\
    \texttt{Settings} $\rightarrow$
    \texttt{Build, Execution, Deployment} $\rightarrow$ \texttt{CMake} $\rightarrow$ \texttt{CMake Options} eintragen.

    Um die Konfigurationen zu laden, muss der CMake-Cache der IDE gelöscht und neu geladen werden. Dies kann im
    Hauptmenü unter dem Menüpunkt \texttt{Tools} $\rightarrow$ \texttt{CMake} $\rightarrow$
    \texttt{Reset Cache and Reload Project} getan werden.

    Nun kann der Source-Code über \texttt{Build} kompiliert werden. Ein Output-Ordner kann über das Toolwindow an der
    rechten Seite angegeben werden.\\
    \textbf{Hinweis:} Sollte das Programm nicht kompilieren und die Fehlermeldung
    \begin{Verbatim}
/usr/lib/gcc/arm-none-eabi/7.1.0/../../../../arm-none-eabi/lib/libg.a
    (lib_a-exit.o): In function `exit':
exit.c:(.text.exit+0x2c): undefined reference to `_exit'
        ...
    \end{Verbatim}
    erscheinen, so ist die in der CMake-Konfigurationsdatei \texttt{arm\_toolchain.cmake} auskommentierte Zeile
    \begin{Verbatim}
set(CMAKE_C_FLAGS "${CMAKE_C_FLAGS} --specs=nosys.specs")
    \end{Verbatim}
    zu setzen. Nun sollte das Programm kompilieren
    \footnote{Schuld daran sind undefinierte Funktionen, die in der ARM libc aufgerufen werden.
    Eine vorübergehende Lösung ist das setzen des \texttt{specs}-Parameters.
    \href{https://stackoverflow.com/a/23922211/6755839}{Siehe Frage bei Stackoverflow}}.

    Ein Programm wird mit Hilfe der \texttt{Flash Binary}-Aktion auf die Zielhardware heruntergeladen.
    Dafür müssen im \textit{OpenOCD-Panel} jeweils der Pfad zur kompilierten Programmdatei und die Exit-Adresse
    angegeben werden. Außerdem ist es wichtig, dass der Pfad zur OpenOCD installation angegeben ist.

    Zum Starten des Daemons und damit auch des GDB-Servers kann die Aktion \texttt{Start OpenOCD Daemon}
    ausgewählt werden. Die Fehlersuche kann über eine \textit{GDB Remote Debug}-Konfiguration mit CLion ausgeführt
    werden. Hierzu muss unter der zu erreichenden Adresse der Port des GDB-Servers mitangegeben werden. Standardmäßig
    ist dieser bei \texttt{3333}, kann jedoch im OpenOCD-Panel verändert werden.
    \\

    Die Hardware muss JTAG/SWD und CMSIS-DAP unterstützen. OpenOCD funktioniert mit Prozessoren basierend auf den
    ARM7/9/11,-Cortex und MIPS-Architekturen.

    \textbf{Hinweis:} Es können nur die vom Prozessor unterstützten Funktionalitäten für das Debugging genutzt werden.
