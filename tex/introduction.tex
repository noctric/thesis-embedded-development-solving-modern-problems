\section{Einleitung}
	\subsection{Steigende Popularität von Embedded-Entwicklung}
	In den letzten Jahren hat der Bereich der Embedded-Systeme deutlich an Popularität gewonnen.
	Embedded-Systeme bezeichnen prozessorbasierte Hardware, die in ein bestehendes System \textit{eingebettet} sind
	und eine dedizierte Aufgabe übernehmen.
	Vorangetrieben durch Trends, wie \glsit{glo:iot}, nehmen immer mehr IT-Unternehmen Embedded-Systeme
	in ihr Repertoire auf.
	``These embedded systems are increasingly being joined together into an ``Internet of things.'' In the future,
	machines that are linked together in networks could revolutionize manufacturing, energy distribution, and
	transportation systems.''\citep[]{Siemens2014}

    Mitverantwortlich für den Aufschwung der Embedded-Entwicklung ist die Arduino\hyp{}Plattform, welche bereits seit 2005
    den Fokus auf einfach zu verwendende Software und Hardware setzt. Ziel der Plattform ist es, vor allem Studenten die
    Möglichkeiten zu geben, schnelles Hardware-Prototyping ohne große Vorkenntnisse im Hardware- und Programmierbereich
    durchzuführen
    \footnote{Laut der \href{https://www.arduino.cc/en/Guide/Introduction}{Hersteller-Seite} Zugriff: 5. Juli 2017}.
    Während die Einfachheit und Schnelligkeit ein großer Verkaufspunkt der Arduino\hyp{}Plattform ist, birgt sie auch große
    Nachteile, die es Entwicklern schwer macht damit zu arbeiten. Genauer wird dies in \ref{sec:arduino} analysiert.

	``Das Gesamtmarktvolumen für Embedded-Systeme für Deutschland wird auf über 18,7 Mrd. im vergangenen Jahr geschätzt.
      Die Anbieter tragen in 2007 dazu 3,7 Mrd. Euro bei und die Wertschöpfung in den Anwendersektoren umfasst mehr als
      15 Mrd. Euro.''\citep[vgl. S. 1]{Bitkom2009}

	Die wachsende Relevanz der Mikrocontroller-Programmierung resultiert darin, dass sich mehr Entwickler mit der
	Thematik beschäftigen und einen Einstieg finden wollen. Vor allem im Bereich der Programmierung ist der Bedarf an
	Entwicklern groß.
	``Von den vier Bereichen Hardware Design, Produktion, Programmierung und Integration von Embedded-Systemen
    ist die Softwareentwicklung und -programmierung der mitarbeiterintensivste Bereich.''\citep[vgl. S. 24]{Bitkom2009}

    In der Regel bilden E-Techniker und Studierte der Technischen Informatik den Großteil der Entwickler für den
    Embedded-Bereich, jedoch finden sich immer mehr Quereinsteiger (unter denen sich viele Software-Entwickler befinden)
    in die Programmierung für Mikrocontroller ein.

    Zusätzlich zu der wachsenden Anzahl an Entwicklern steigt auch die Anzahl an Entwicklungswerkzeugen und es stehen
    immer mehr davon frei zur Verfügung. Die Nähe zur Hardware bringt allerdings das Problem mit sich,
    dass diese Entwicklungswerkzeuge meistens auf ein spezifisches Modell und somit auch auf einen spezifischen
    Hersteller eines Prozessors zugeschnitten sind.

    Zudem mangelt es an kostenfreien Möglichkeiten die Firmware eines Prozessors auf Fehler zu untersuchen.

    Das Open-Source-Programm
    \glsit{glo:openocd} verfolgt das Ziel die Firmware eines Prozessor über \glsit{glo:gdb} debuggen zu können.
    ``The GNU Debugger
      (gdb) offers excellent debugging support, but covers only some areas of embedded systems debugging.
      Low level tasks require additional hard- and software, and existing open source solutions for these tasks
      are incomplete or at least partially deficient.''\citep[vgl. Introduction]{Rath2005}

    Diese Arbeit beschäftigt sich mit dem Entwicklungsprozess für Embedded\hyp{}Systeme, speziell für die von \glsit{glo:arm}
    entwickelten Cortex\hyp{}M\hyp{}Serien.
    Dabei stehen die Aspekte der Chip-Programmierung und der Fehleranalyse im Vordergrund. Das Ziel dieser Arbeit
    ist es, die Probleme und Herausforderungen, vor allem für Neueinsteiger, der Entwicklung für Embedded-Systeme herauszufinden und
    einen allgemeinen Lösungsansatz dieser Problematiken sowohl für Neueinsteiger, als auch für fortgeschrittene
    Entwickler im Embedded-Bereich zu erarbeiten.

    \subsection{Gliederung}
    Um bestimmte Design-Entscheidungen besser verstehen zu können, folgen zunächst theoretische Hintergründe zu
    \glsit{glo:arm}-Cortex-M-Microcontrollern. Die Betrachtungsweise ähnelt dabei der eines Software-Entwicklers.

    Im zweiten Abschnitt wird zunächst die \glsit{glo:arm}-Cortex-M-Familie vorgestellt und ein allgemeiner Entwicklungsprozess
    beschrieben. Hierbei wird auf Nachteile und übliche Probleme dieses Prozesses eingegangen.

    Der dritte Abschnitt befasst sich mit vorhandenen Abstraktionen (nicht Betriebssystemen) und Protokollen, welche
    dem Entwickler einen einfacheren Zugriff auf die Hardware ermöglichen sollen oder bestimmte Funktionalitäten, wie
    zum Beispiel Debugging, erfüllen.

    Der vierte Teil geht auf die Software-seitige Fehleranalyse mit \glsit{glo:gdb} und vorhandene Software für On-Chip-Debugging
    ein. Unterschiede zwischen Software- und Hardware-Debugging sollen hier dargelegt werden.

    Als fünfter Teil folgt ein eigener Lösungsvorschlag zur Vereinfachung der Entwicklung von Embedded-Systemen.
    Ziel ist es, auf Software- wie auf Hardware-Seite, eine Lösung zu finden, die den Entwicklungsprozess für
    Embedded-Systeme vereinfacht. Der praktische Teil dieser Arbeit ist es, den ersten Schritt auf der Software-Seite
    zu implementieren. Hierzu wird mit Hilfe von \glsit{glo:openocd} ein \acrshort{glo:ide}-Plugin speziell für die Embedded-Entwicklung
    implementiert. Es folgt außerdem eine Einordnung der Rolle dieses Arbeitsschrittes in Hinblick auf die Gesamtlösung.

    Anschließend wird der Inhalt der Arbeit zusammengefasst und ein Ausblick über die Weiterentwicklung der zu
    implementierenden Plattform für Embedded-Systeme gegeben.
    \\

    Als zu testende Hardware, wird ein von Atmel hergestelltes Board mit einem Cortex-M7 Prozessor (\texttt{AT SAM E70 Q21})
    verwendet. Das Board enthält außerdem eine Debug-Einheit mit \acrshortit{glo:cmsisdap}-Unterstützung.