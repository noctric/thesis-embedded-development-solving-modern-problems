\section{Hardware-Debugging, Protokolle und Abstraktionen}
\label{sec:protocols}
Bestimmte Protokolle und Abstraktionen bieten dem Entwickler einen einheitlichen Weg auf bestimmte Funktionen eines
Prozessors zuzugreifen. Es bleibt die Frage, wie ein Entwickler diese Technologien für den Entwicklungsprozess nutzen
kann.

In diesem Kapitel werden alle für den Entwicklungsprozess wichtigen Target-seitigen Technologien, vor allem in Hinblick
auf Debugging, erläutert.

    \subsection{Ein Schichtenmodell für Embedded-Systeme}
    \label{sec:layers}
    Die betrachteten Technologien funktionieren auf verschiedenen Abstraktionsebenen und stellen im Entwicklungskontext
    ein Schichtenmodell zwischen der tatsächlichen Zielhardware, der vom Nutzer entwickelten Applikation und den
    Entwicklungswerkzeugen dar.

    %TODO
    \begin{figure}[h]
        \centering
        \caption{Entwicklungsbasiertes Schichtenmodell der Hardware-Abstraktionen}
        \label{fig:layerhwab}
        \source{eigene Zeichnung}
        \includegraphics[scale=0.5]{../Graphiken/hwschichten_v2}
    \end{figure}

    Abbildung \ref{fig:layerhwab} zeigt den groben Aufbau des Schichtenmodells. \acrshortit{glo:jtag} und \acrshortit{glo:swd} (siehe Abschnitt
    \ref{sec:jtag} und \ref {sec:swd}) sind dabei direkt an die Hardware gebunden. Grund dafür ist, dass für den
    \acrshortit{glo:jtag}-Standard sowohl zusätzliche Hardware als auch eine Abstraktion durch das Protokoll vorhanden ist. Das \glsit{glo:cmsisdap}
    liegt dabei genau über dem \acrshortit{glo:jtag} bzw. \acrshortit{glo:swd}, da es die \acrshortit{glo:jtag}-Funktionalitäten direkt benutzt
    (siehe Abschnitt \ref{sec:cmsisdap}). \glsit{glo:cmsisdap} gehört zum \glsit{glo:cmsis}, welches auch andere Hardware-Abstraktionen,
    auf verschiedenen Schichten, wie zum Beispiel \glsit{glo:cmsis}-Core oder \glsit{glo:cmsis}-Driver, enthält (siehe Abschnitt
    \ref{sec:cmsis}).

    \acrshortit{glo:jtag}, \acrshortit{glo:swd} (optional) und \glsit{glo:cmsisdap} werden primär für das Testen und Aufspielen, sowie die Fehlersuche in
    Applikationen verwendet.

    Die restlichen aufgeführten Schichten dienen als vom Entwickler in der Applikation zu verwendende Abstraktionen.

	\subsection{JTAG}
	\label{sec:jtag}
	\acrshortit{glo:jtag} steht für \textit{Joint Test Action Group} und bezeichnet einen Standard zum Testen von Hardware, welcher 1985
	in Zusammenarbeit von verschiedenen Hardware-Herstellern erarbeitet wurde und als Standard des \glsit{glo:ieee} festgehalten ist.
	Der Standard wurde ursprünglich entwickelt, um Komponentenbasiertes Testen, Vernetzung und Interaktionen von
	verschiedenen Komponenten auf einem \glsit{glo:ic} zu vereinfachen bzw. zu ermöglichen\citep[vgl.~Kap.~2]{Rath2005}.
	Das Testverfahren wird \textit{Boundary Scan} genannt.
	Am häufigsten benutzt wird \acrshortit{glo:jtag} zum Programmieren und Debuggen eines \glsit{glo:ic}.

    \acrshortit{glo:jtag} umfasst sowohl ein Protokoll als auch die entsprechende Hardware. Die Hardware sitzt auf dem \glsit{glo:ic} und ermöglicht
    zusammen mit dem \acrshortit{glo:jtag}-Protokoll direkten Zugriff auf bestimmte Funktionalitäten eines Prozessors.

    Ein \acrshortit{glo:jtag}-komformes Gerät enthält ein \glsit{glo:ir}, \glsit{glo:dr} und
    ein \glsit{glo:tap}. Die beiden Register sind als Schieberegister
    (engl. ``shift register'') implementiert, d.h. der Speicherinhalt von aneinandergeketteten \glsplit{glo:flipflop} wird
    pro Takt ins nächste \glsit{glo:flipflop} verschoben.
    Als \glsit{glo:dr} können je nach Befehl verschiedene Register eingesetzt werden.

    Jeder Pin auf dem \glsit{glo:ic} enthält eine \acrshortit{glo:jtag}- bzw. Boundary-Scan-Zelle, die sowohl Daten vom Pin ``aufnehmen'' als auch
    auf den Pin schreiben kann. Aufgenommene, so wie geschriebene Daten werden seriell über die \acrshortit{glo:jtag}-Zellen durch den
    \glsit{glo:tap} verschoben\citep[]{Corelis}.

    Der \glsit{glo:tap} stellt die \acrshortit{glo:jtag}-Hardware-Schnittstelle dar und besteht aus den folgenden Anschlüssen
    \begin{description}
        \item[TDI] Die Testdaten werden über \texttt{Test Data Input} in die \acrshortit{glo:jtag}-Zellen geschoben
        \item[TCK] Das Signal-Interface \texttt{Test Clock} taktet die \acrshortit{glo:jtag}-State-Machine
        \item[TMS] Kontrolliert wird die \acrshortit{glo:jtag}-State-Machine (\glsit{glo:tap}-Controller) von dem \texttt{Test Mode Select}
        \item[TDO] Die Testdaten werden über das \texttt{Test Data Output} zurück gelesen
        \item[TRST] Das \texttt{Test Reset} ist optional und ermöglicht asynchrones Zurücksetzen des \glsit{glo:tap}-Controllers
    \end{description}

    \texttt{TDI}, \texttt{TCK}, \texttt{TMS} und \texttt{TRST} sind für den Prozessor also Output-Signale und
    \texttt{TDO} ein Input-Signal.
    Diese Pins sind üblicherweise dediziert, d.h. sie werden nicht für andere Zwecke genutzt.

    Im \glsit{glo:datasheet} der Test-Hardware \texttt{AT SAM E70 Q21}, stehen die unterstützten \acrshortit{glo:jtag}-Befehle. Dem Standard nach
    sind drei Befehle verpflichtend, sie müssen also vom entsprechenden Chip-Hersteller implementiert werden\footnote{
    \citep[]{Corelis}}:
    \begin{description}
    \item[\texttt{BYPASS}] Beim BYPASS wird das \textit{Bypass-Register} zwischen \texttt{TDI} und \texttt{TDO}
    geschaltet. Das Bypass-Register hält ein einzelnes Bit (engl. ``single bit register'') und umgeht das bei EXTEST
    verwendete längere Boundary-Scan-Register. Dieser Befehl wird genutzt, um Geräte bei einer Verkettung von
    \acrshortit{glo:jtag}-Geräten zu umgehen. Dabei bleibt das betroffene Gerät funktionsfähig.
    \item[\texttt{EXTEST}] Der EXTEST-Befehl wird verwendet, um die zusammengeschalteten Verbindungen zwischen den
    Geräten durchzuführen. Das Boundary-Scan-Register (eins der \glsit{glo:dr}) wird zwischen \texttt{TDI} und
    \texttt{TDO} geschaltet.
    Dabei wird das Gerät in den \textit{externen} Modus versetzt, d.h. Testdaten werden über die
    \acrshortit{glo:jtag}-Zellen auf die Pins geschoben. Gleichzeitig nehmen die \acrshortit{glo:jtag}-Zellen auch die von den Pins geschriebenen Daten
    auf. EXTEST ist der Hauptbefehl um Boundary-Scans durchzuführen.
    \item[\texttt{SAMPLE/PRELOAD}] SAMPLE ähnelt dem EXTEST\hyp{}Befehl, indem das Boundary\hyp{}Scan\hyp{}Register zwischen
    \texttt{TDI} und \texttt{TDO} geschaltet wird. Der Unterschied liegt darin, dass das Boundary-Scan-Gerät trotzdem
    funktionsfähig bleibt. Das Boundary-Scan-Register ist also zugreifbar, ohne dass das Gerät seine
    Funktion unterbrechen muss.
    \end{description}

    Der in Abbildung \ref{fig:tapcontroller} skizzierte \glsit{glo:tap}-Controller bezeichnet eine \glsit{glo:statemachine}
    mit 16 verschiedenen Zuständen.
    Diese \glsit{glo:statemachine} enthält Zustände, welche sowohl das \glsit{glo:dr} als auch das \glsit{glo:ir}
    aufnehmen und Daten auf besagte Register schieben können.
    \begin{figure}[h]
        \centering
        \caption{Endlicher Automat des TAP-Controllers}
        \source{
            \href{
                https://commons.wikimedia.org/w/index.php?curid=45147866}{Von Rudolph H 17:30, 25. Jul. 2009 (CEST)
                - Eigene Zeichung, Copyrighted free use
                }
        }
        \label{fig:tapcontroller}
        \includegraphics[scale=0.3]{../Graphiken/JTAG_TAP_Controller_State_Diagram.png}
    \end{figure}
    \acrshortit{glo:jtag} stellt also eine einheitliche Schnittstelle zur internen Logik des \glsit{glo:ic} dar. Somit können vom Chip-Hersteller
    implementierte Funktionalitäten, wie zum Beispiel das Setzen von Hardware-Breakpoints durch eine externe Instanz
    (sei es zusätzliche Hardware), genutzt werden.

    Eine von \glsit{glo:arm} entwickelte Alternative zu \acrshortit{glo:jtag} ist das im nächsten Abschnitt \ref{sec:swd} behandelte
    \acrshortit{glo:swd}, welches den ``Vorteil hat, dass es pinkompatibel zu \acrshortit{glo:jtag} ist, es aber erlaubt, softwaregesteuert
    zwischen \acrshortit{glo:jtag} und \acrshortit{glo:swd} zu wechseln''\citep[]{Asche2017}.

    Ein \acrshortit{glo:jtag}-Adapter ist ein externer Mikrocontroller mit spezieller Firmware, der vom Host-System gesendete
    Debug-Befehle (z.B. über \glsit{glo:usb}) in \acrshortit{glo:jtag}-Befehle umsetzt und direkt an das \acrshortit{glo:jtag}-Interface weiterleitet.
    Eine Alternative zu diesen externen Hardware-Debuggern ist ein auf dem \glsit{glo:pcb} integrierter \glsit{glo:ic} mit einer
    \glsit{glo:cmsisdap}-fähigen Firmware. Weiteres dazu in Abschnitt \ref{sec:cmsisdap}.
	\subsection{SWD}
	\label{sec:swd}
	Im Gegensatz zu dem im vorherigen Abschnitt \ref{sec:jtag} behandelten \acrshortit{glo:jtag}-Standard, ist
	\acrfullit{glo:swd} eine von \glsit{glo:arm} implementierte proprietäre Technologie, die \acrshortit{glo:jtag}-Funktionalität zu
	effizienteren Umständen verspricht.

	\acrshortit{glo:swd} benötigt im Vergleich zu \acrshortit{glo:jtag} nur 2 Pins für die Signal-Interfaces.
	\begin{description}
	    \item[\texttt{SWDIO}] Bidirektionaler Daten-Pin zum Schreiben und Lesen von Daten. Ersetzt die in \acrshortit{glo:jtag}
	    verwendeten \texttt{TDI}- und \texttt{TDO}-Kontakte, wobei der in \acrshortit{glo:jtag} als \texttt{TMS} bezeichnete Pin bei
	    \acrshortit{glo:swd} in den \texttt{SWDIO} umgewandelt wird \citep[vgl.~S.~6]{Booth}
	    \item[\texttt{SWCLK}] Ähnlich wie bei \acrshortit{glo:jtag} ein Taktungs-Pin zum Takten des Test-Controllers. Dieser ersetzt
	    den \acrshortit{glo:jtag}-\texttt{TCK}-Pin \citep[vgl.~S.~6]{Booth}
	\end{description}

    \acrshortit{glo:swd} nutzt ein nach \glsit{glo:arm} standartisiertes bidirektionales Protokoll, welches im \glsit{glo:adi} v5
    definiert ist, um Daten vom Debugger zum Target-System zu übertragen
    \footnote{\href{https://www.arm.com/products/processors/serial-wire-debug.php}
    {ARM SWD Produkt Seite, Zugriff: 15.07.2017}}.
    Das \glsit{glo:adi} ist eine allgemeine Abstraktion und Definierung eines standartisierten
    Debug-Interfaces für \glsit{glo:arm} Prozessoren. Version 5 des \glsit{glo:adi} ist jedoch darauf ausgelegt, nicht
    \glsit{glo:arm}-prozessorspezifisch zu sein, sondern eine weitere Bandbreite an Geräten zu unterstützen
    \citep[vgl.~Kap.~1]{ArmAdi}.
    \acrshortit{glo:swd} ist kompatibel mit allen \glsit{glo:arm} Prozessoren und anderen Prozessoren, welche \acrshortit{glo:jtag} zur Fehlersuche nutzen. Es verschafft
    außerdem Zugang zu den Debug-Registern der oben genanntent \glsit{glo:arm}-Cortex-Prozessorfamilien
    \footnote{\href{https://www.arm.com/products/processors/serial-wire-debug.php}
    {ARM SWD Produkt Seite, Zugriff: 15.07.2017}}.

    \subsection{ARM CMSIS}
    \label{sec:cmsis}
    Die \glsit{glo:cmsis}-Bibliothek wird von \glsit{glo:arm} entwickelt und stellt eine
    herstellerunabhängige Hardware-Abstraktion für \glsit{glo:arm}-Cortex-Architekturen dar. Es ist der Versuch eine einheitliche
    und konsistente Schnittstelle für den Zugriff auf die Hardware zu schaffen. ``Creation of software is a major cost
    factor in the embedded industry. Standardizing the software interfaces across all Cortex-M silicon vendor products,
    especially when creating new projects or migrating existing software to a new device, means significant cost
    reductions.''\footnote{\href{https://developer.arm.com/embedded/cmsis}{ARM cmsis Produktbeschreibung}}

    Die \glsit{glo:cmsis}-Bibliothek stellt eine Vielzahl an Funktionen für den Zugriff auf die Hardware auf unterschiedlichen
    Ebenen bereit.

    ``Prozessorhersteller (oder ggf. Drittfirmen oder die Public Domain) werden dazu ermutigt, diese Bibliotheken für
    ihre Produkte zu realisieren; gleichzeitig sollen Middlewarehersteller und Entwickler gerätetreiberseitig ebenfalls
    diese eingekapselten Aufrufe benutzen, so dass theoretisch von \glsit{glo:arm} „angestupst“ eine sehr große Interoperabilität
    zwischen Soft- und Middleware und verschiedenen ACPs besteht.''
    \footnote{ACP steht für \textit{ARM Cortex based processor}}
    \citep[Kap.~2.1]{Asche2017}

    \glsit{glo:cmsis} besteht zu heutigem Zeitpunkt aus verschiedenen Paketen auf verschiedenen Abstraktionsstufen.
    Das \glsit{glo:cmsis}-Core ist der Grundbaustein der \glsit{glo:cmsis}-Bibliothek und stellt Kernfunktionen für den Zugriff auf die
    Peripherie dar. Außerdem enthält es bestimmte intrinsische Funktionen.

    \textbf{Beispiel:} Um einen Interrupt zu aktivieren, kann die Funktion
    \begin{lstlisting}
void NVIC_EnableIRQ(IRQn_Type IRQn)
    \end{lstlisting}
    mit \texttt{IRQn} als Interrupt-Nummer verwendet werden.

    Für die Fehleranalyse sind vor Allen die beiden Pakete \glsit{glo:cmsis}-SVD und \glsit{glo:cmsisdap} relevant.
    Das \glsit{glo:cmsis}-SVD soll es Debuggern ermöglichen, eine genaue Übersicht über Peripherie zu liefern, ohne auf
    die tatsächliche Implementierung dieser achten zu müssen.

    Auf \glsit{glo:cmsisdap} wird in Abschnitt \ref{sec:cmsisdap} genauer eingegangen.

    \textit{\glsit{glo:cmsis}\hyp{}\glsit{glo:rtos}} stellt ein generisches \glsit{glo:rtos}\hyp{}Interface für \glsit{glo:arm}-Cortex basierte Geräte bereit.
    Es ist eine Schnittstelle für Software-Komponenten, welche \glsit{glo:rtos}-Funktionalität benötigen und ermöglicht den
    einheitlichen Zugriff auf \glsit{glo:rtos}\hyp{}Funktionalitäten verschiedener Echtzeitbetriebssysteme.

    \begin{figure}[h]
    	\caption{Die CMSIS Softwarepakete}
    	\source{
    	    \href{
    	        https://developer.arm.com/embedded/cmsis
    	    }{
    	        ARM CMSIS Produkt Seite
    	    }, Zugriff: 17.08.2017
    	}
    	\label{fig:cmsis_packages}
        \centering
        \includegraphics[scale=0.3]{../Graphiken/CMSIS_block_diagram.png}
    \end{figure}

    Eine Bibliothek für das Verarbeiten von Digitalen Signalen ist unter dem Namen \glsit{glo:cmsis}-DSP vorhanden und
    enthält viele auf Cortex-M optimierte Verarbeitungs- und Berechnungsfunktionen.

    Als letztes ist ein Peripherie-Treiber-Interface \glsit{glo:cmsis}-Driver für die Verwendung in Middleware und
    Nutzerapplikationen verfügbar. Es ist ein von \glsit{glo:arm} definiertes Interface, welches von unabhängigen
    Hardware-Herstellern implementiert werden kann. Code-Basen, die mit dem \glsit{glo:cmsis}-Driver-Interface arbeiten sind in der
    Theorie portabler als diejenigen, welche direkt die herstellerspezifischen Treiber
    nutzen. Da es sich hierbei lediglich um einen vorgeschlagenen Standard handelt, sind die Hardware-Hersteller nicht
    dazu verpflichtet diesen Standard einzuhalten bzw. zu implementieren.

	\subsection{CMSIS-DAP}
	\label{sec:cmsisdap}
	\acrfullit{glo:cmsisdap} ist ein Produkt in der \glsit{glo:cmsis}-Reihe (siehe Abschnitt \ref{sec:cmsis}) und bezeichnet die Umsetzung eines
	von \glsit{glo:arm} definierten Protokolls für den Zugriff auf die Debug-Funktionalitäten eines Prozessors.

	Der \glsit{glo:dap} ist eine Implementierung des in Abschnitt \ref{sec:swd} erwähnten \glsit{glo:adi}.
	Der \glsit{glo:dap} wird in \glsplit{glo:dp} und \glsplit{glo:ap} aufgeteilt.
	Die \glsplit{glo:dp} können von externen Debuggern genutzt werden, um den \glsit{glo:dap} zu erreichen, über die \glsplit{glo:ap} kann
	auf On-Chip-Ressourcen zugegriffen werden. \acrshortit{glo:jtag} und \acrshortit{glo:swd} nutzen \glsit{glo:dap} für die Kommunikation mit der entsprechenden
	Hardware.

	\glsit{glo:cmsisdap} läuft als Firmware auf einer Debug-Einheit, ein zweiter \glsit{glo:ic}, welcher sich auf dem \glsit{glo:pcb} befindet.
	Dieser Chip wird üblicherweise direkt per \glsit{glo:usb} mit einem Host-System verbunden. Über die Debug-Einheit können
	mit Hilfe des \glsit{glo:cmsisdap}-Protokolls auf direktem Weg \acrshortit{glo:jtag}-Befehle an die Ziel-Hardware gesendet werden. Es stellt
	also ein unmittelbares On-Chip-Debug-Interface dar, welches die Verwendung von externen Hardware-Debuggern obsolet
	macht. Falls unterstützt, kann auch \acrshortit{glo:swd} genutzt werden, um mit dem Prozessor zu
	kommunizieren (siehe Abschnitt \ref{sec:swd}).

	\begin{figure}[h]
	    \centering
	    \caption{CMSIS-DAP Übersicht}
	    \source{\citep[Übersicht]{CMSISDAP}}
	    \label{fig:cmsisdapoverview}
	    \includegraphics[scale=0.7]{../Graphiken/cmsis_dap}
	\end{figure}

	Das \glsit{glo:cmsisdap}-Protokoll liefert Befehle zum Kommunizieren mit der Ziel-Hardware. Nebenbei können auch bestimmte
	Informationen über die Debug-Einheit abgerufen werden. Listing \ref{cmsisdapled} zeigt beispielhaft einen
	Kommunikationsaufruf mit einer auf dem \textit{Evaluation Board} von Atmel verbauten Debug-Einheit.

	Das \glsit{glo:cmsisdap}-Protokoll definiert Befehle, die über die entsprechende Befehls-ID ausgeführt werden.
	In Listing \ref{cmsisdapled} lautet die Befehls-ID für den LED-Befehl (\texttt{dap\_led}) \texttt{0x01} und wird in
	\texttt{buf[1]} geschrieben.
	Die Felder \texttt{buf[2]} und \texttt{buf[3]} enthalten die entsprechenden Parameter für den LED-Befehl.
	\texttt{buf[0]} ist die \texttt{report id} für das \acrshortit{glo:hid}-Protokoll und hat nichts mit \glsit{glo:cmsisdap} zu tun.

	\newpage

	\begin{lstlisting}[
	    language=Rust,
	    caption=CMSIS-DAP-Befehl um die LEDs auf der Debug-Einheit zu bedienen (Rust-Code),
	    label=cmsisdapled,
	    float,
	    floatplacement=H
	]
// HELPER Send LED Command to CMSIS_DAP Dev --> write
fn dap_led(dev: &hidapi::HidDevice, led: u8, output: u8) {
    let mut buf: [u8; 512 + 1] = [0xff; 512 + 1];
    buf[0] = 0x00;      // report id
    buf[1] = 0x01;      // dap_led
    buf[2] = led;       // LED: specifies the LED
                        // that is addressed: 0 = Connect LED;
                        // 1 = Running LED.
    buf[3] = output;    // Output: specifices the status of
                        // the LED after the command: 1 = LED on;
                        // 0 = LED off.

    println!("writing to device.");

    let res = dev.write(&buf).unwrap();

    println!("wrote {:?} Byte(s).", res);
}
	\end{lstlisting}

	Verdeutlicht werden kann die Verwendung von \glsit{glo:cmsisdap} in Kombination mit \acrshortit{glo:jtag} und \acrshortit{glo:swd} an einem Beispiel. Hierzu
	wird zunächst der Quelltext des \glsit{glo:openocd} genauer untersucht.
	Das folgende Beispiel zeigt eine Kette von Aufrufen der verschiedenen Programmmodule innerhalb des \glsit{glo:openocd}.

	Es wird angenommen, dass das Host-System mit einer Debugging-Software einen Breakpoint in der auf dem Prozessor
	laufenden Firmware setzen möchte (Genaueres zu Debugging in Abschnitt \ref{sec:debugging}).
	Der Fokus bei dem Beispiel liegt auf der letzten Instanz, in der der Breakpoint Befehl vom \glsit{glo:cmsisdap}-Programmmodul
	empfangen und schließlich per \glsit{glo:usb} an die Debug-Einheit gesendet wird.

    \begin{enumerate}
    \item \texttt{gdb\_server.c}
    \begin{itemize}
        \item Empfängt Breakpoint-Befehl von Debugging-Software (zum Beispiel eine \glsit{glo:ide} oder lokaler \glsit{glo:gdb})
        \item Übergibt den Breakpoint-Befehl weiter an ein allgemeines Hardware-Interface \texttt{target.c}
        \begin{lstlisting}
static int gdb_write_memory_binary_packet(struct connection *connection, char const *packet, int packet_size){...}
        \end{lstlisting}
    \end{itemize}
     \item \texttt{target.c}
     \begin{itemize}
        \item Interface\hyp{}ähnliche Programmlogik sucht die korrekte Prozessorfamilie heraus und leitet den Befehl weiter
        an die konkrete Hardware\hyp{}Abstraktion
        \begin{lstlisting}
static int target_write_buffer_default(struct target *target, uint32_t address, uint32_t count, const uint8_t *buffer){...}
        \end{lstlisting}
     \end{itemize}
     \item \texttt{cortex\_m.c}
     \begin{itemize}
        \item Berücksichtigt Besonderheiten, die bei der Cortex-M-Prozessorfamilie auftreten. Hier sind die
        Debug-spezifischen Methoden enthalten, die über den Befehl
        \begin{lstlisting}
static int cortex_m_write_memory(struct target *target, uint32_t address, uint32_t size, uint32_t count, const uint8_t *buffer) {...}
        \end{lstlisting}
        weiter an das Debug-Interface gereicht werden
     \end{itemize}
     \item \texttt{arm\_adi\_v5.h/.c}
     \begin{itemize}
        \item Enthält Methoden zum Ausführen der im \glsit{glo:adi} v5 (siehe Abschnitt \ref{sec:swd}) von \glsit{glo:arm}
        definierten Befehle
        \item In diesem Fall ist es der Schreibbefehl für den Speicherpuffer, welcher über die Methode
        \begin{lstlisting}
int mem_ap_sel_write_buf(struct adiv5_dap *swjdp, uint8_t ap, const uint8_t *buffer, uint32_t size, uint32_t count, uint32_t address) {...}
        \end{lstlisting}
        und anschließend
        \begin{lstlisting}
static inline int dap_queue_ap_write(struct adiv5_dap *dap, unsigned reg, uint32_t data){...}
        \end{lstlisting}
        den Breakpoint-Befehl schreibt
     \end{itemize}
     \item \texttt{adi\_v5\_swd.c}
        \begin{itemize}
        \item Der Breakpoint-Befehl wird in das \acrshortit{glo:swd}-Protokoll überführt
        \begin{lstlisting}
static int swd_queue_ap_write(struct adiv5_dap *dap, unsigned reg, uint32_t data){...}
        \end{lstlisting}
     \end{itemize}
     \item \texttt{cmsis\_dap\_usb.c}
     \begin{itemize}
        \item Auf Software-Ebene ist dies die letzte Schicht, die den Breakpoint-Befehl endgültig an die Ziel-Hardware
        sendet
        \item Hier wird zunächst das Kommando in das \glsit{glo:cmsisdap}-Protokoll überführt und in eine Warteschlange geschrieben
        \item Über das \glsit{glo:cmsisdap}-Kommando \texttt{CMSIS\_DAP\_TFER}, welches die definierte ID \texttt{0x05} besitzt,
        wird nun der Breakpoint-Befehl per \glsit{glo:usb} an den Prozessor gesendet
        \item Der Transfer-Befehl ähnelt dem oben gezeigten Code-Beispiel \ref{cmsisdapled} mit dem Unterschied, dass
        bei Listing \ref{cmsisqueue} eine gesamte Warteschlange an Befehlen hintereinander abgearbeitet wird.
        Dieser sieht (in gekürzter Fassung) wie folgt aus:
        \begin{lstlisting}[caption={OpenOCD, \texttt{cmsis\_dap\_usb.c}}, label=cmsisqueue]
static int cmsis_dap_swd_run_queue(struct adiv5_dap *dap)
{
    uint8_t *buffer = cmsis_dap_handle->packet_buffer;
    size_t idx = 0;
    buffer[idx++] = 0;	/* report number */
    buffer[idx++] = CMD_DAP_TFER;
    buffer[idx++] = 0x00;	/* DAP Index */
    buffer[idx++] = pending_transfer_count;

    for (int i = 0; i < pending_transfer_count; i++) {
        uint8_t cmd = pending_transfers[i].cmd;
        uint32_t data = pending_transfers[i].data;

        buffer[idx++] = (cmd >> 1) & 0x0f;
        if (!(cmd & SWD_CMD_RnW)) {
        	buffer[idx++] = (data) & 0xff;
        	buffer[idx++] = (data >> 8) & 0xff;
        	buffer[idx++] = (data >> 16) & 0xff;
        	buffer[idx++] = (data >> 24) & 0xff;
        }
    }

    queued_retval = cmsis_dap_usb_xfer(cmsis_dap_handle, idx);
}
    \end{lstlisting}
    \end{itemize}
    \end{enumerate}

    \subsection{Hardware Abstraction Layer}
    Der \glsit{glo:hal} spielt mit der wachsenden Anzahl an verschiedenen Variationen von Hardware
    eine immer größer werdende Rolle in der Entwicklung von Embedded-Systemen. Der \glsit{glo:hal} stellt als weitere
    Abstraktionsschicht ein Bindeglied zwischen Applikation und Hardware dar. Diese verwenden in der Regel das
    \glsit{glo:cmsis}-Core als unterste Abstraktionsschicht und bauen darauf auf.

    Chip-Hersteller bieten ihren eigenen \glsit{glo:hal} an, um einen einfacheren Zugriff auf die
    entsprechende Peripherie eines \glsit{glo:pcb}s zu ermöglichen. Beispielsweise stellt Atmel das
    \glsit{glo:asf} zur Verfügung.

    \subsection{Applikationsentwicklung mit Abstraktionsschichten}
    \label{sec:abstractionoverview}
    Die vorgestellten Technologien ergeben, wie das Code-Beispiel in Abschnitt \ref{sec:cmsisdap} verdeutlichen soll,
    in Kombination miteinander ein Schichtenmodell indem Hardware-Schnittstellen und Protokolle aufeinander aufbauen.

    Tabelle \ref{tab:hwabstractionlayer} zeigt eine vereinfachte Strukturierung der verschiedenen Schichten. Es ist
    anzumerken, dass \acrshortit{glo:swd} oder \glsit{glo:cmsis}-Driver nicht zwangsläufig verwendet werden. Die Tabelle dient lediglich der
    Verdeutlichung der Struktur der beschriebenen Technologien.

    Prinzipiell kann ein Entwickler seine Anwendung auch ohne jegliche Abstraktionen direkt auf der Hardware
    (``bare metal'') entwickeln. Dies hat aber zu Folge, dass der Aufwand um ein Vielfaches gesteigert wird und
    möglicherweise bei verschiedenen Applikationen viel redundante Arbeit entsteht. Einfache Zugriffe auf
    bestimmte Funktionalitäten müssen für jede neue Applikation neu implementiert werden.

    \begin{table}[h]
        \centering
        \caption{Abstraktionsschichten bei der Entwicklung von Embedded-Systemen}
        \label{tab:hwabstractionlayer}
        \begin{tabular}{| c | C{7cm} |}
        \hline
        \textbf{6.} & Applikation                                           \\ \hline
        \textbf{5.} & \glsit{glo:cmsis}-Driver / herstellerunabhängige Schnittstelle   \\ \hline
        \textbf{4.} & \glsit{glo:hal} / Treiber / Betriebssystem                        \\ \hline
        \textbf{3.} & \glsit{glo:cmsis}-Core und \glsit{glo:cmsisdap}                              \\ \hline
        \textbf{2.} & \acrshortit{glo:swd}                                                   \\ \hline
        \textbf{1.} & \acrshortit{glo:jtag}                                                  \\ \hline
        \end{tabular}
    \end{table}

    Da das \acrshortit{glo:jtag}-Protokoll der einzige Standard in dem Bereich ist, bauen viele Technologien und Abstraktionen darauf
    auf. Dies führt dazu, dass es auf vielen Prozessoren verbaut und unterstützt wird.

    \acrshortit{glo:swd} und \glsit{glo:cmsis} sind die von \glsit{glo:arm} entwickelten Technologien und sind deshalb, wegen der weiten Verbreitung an \glsit{glo:arm}
    Prozessorarchitekturen bei Embedded-Systemen, sehr geläufig.

    Die von den jeweiligen Hardware\hyp{}Herstellern bereitgestellten Hardware\hyp{}Abstraktionen können vom Entwickler für eine
    Applikation verwendet werden. Ein fehlender Standard bei diesen Abstraktionen macht eine Applikation in vielen
    Fällen schwer portierbar.

    Alternativ dazu würde auf selber Ebene ein Betriebssystem liegen. Auch das Betriebssystem stellt eine
    Hardware-Abstraktion dar und ist von der verwendeten Prozessorarchitektur abhängig.
    \\

    Man betrachtet den Entwicklungszyklus einer Hardware-Applikation und der in Abschnitt \ref{sec:layers} erwähnten
    unterschiedlichen Verwendung für die Schichten und stellt folgende Problematiken fest:

    Das erste Problem ist ein fehlender Standard der vom Hersteller implementierten Hardware-Abstraktionen.
    Erneut bahnt sich hier das Problem an, an die Software eines bestimmten Herstellers und damit auch potenziell
    an qualitativ geringere Software gebunden zu sein. Ein fehlender Standard des \glsit{glo:hal}s verschlechtert die Portierbarkeit
    einer Code-Basis und der Entwickler wird bei einem Wechsel der Hardware mit redundanter Arbeit belastet.
    Abschnitt \ref{sec:overallsolution} beschäftigt sich im ersten Teil mit einem Lösungsansatz für dieses Problem.

    Die zweite Problematik ist der eigentliche Entwicklungsprozess einer Hardware-Applikation, welche das Aufspielen
    und Debuggen von Applikationen, basierend auf den \acrshortit{glo:jtag}-/\acrshortit{glo:swd}-Protokollen, beinhaltet. Ein Lösungsansatz für dieses
    Problem wird im zweiten Teil des Abschnitts \ref{sec:overallsolution} vorgestellt und dokumentiert.