\section{Gesamtlösung}
\label{sec:overallsolution}
    Dieser Teil der Arbeit beschäftigt sich mit einem konkreten Lösungsansatz zur Vereinfachung der Entwicklung von
    Embedded-Systemen. Dabei skizziert der erste Teil Grundprinzipien und das Gesamtkonzept der Software-Architektur,
    dessen Implementierung nicht im Rahmen dieser Arbeit liegt. Der zweite Teil stellt das Software-Projekt für ein
    \glsit{glo:ide}-Plugin vor, welches im Rahmen der Arbeit implementiert wird, und dokumentiert den Entwicklungsprozess.
    \subsection{Gesamtlösung}

    Um den Entwicklungsprozess einer Hardware-Applikation zu vereinfachen, sind die in Abschnitt
    \ref{sec:abstractionoverview} aufgeführten Problematiken bezüglich der Hardware-Abstraktionen und der daraus
    entstehende umständliche Entwicklungsprozess zu beachten.

    Zugrunde liegt ein Konzept für eine Plattform zur Hardware-Entwicklung, welches aus zwei grundlegenden
    Komponenten besteht.

    Die erste Komponente ist eine modulare Hardware basierend auf dem Cortex-M7-Mikrocontroller. Diese wird in der
    Ausarbeitung nicht weiter beschrieben, da ein Prototyp für die Hardware noch nicht existiert und die Beschreibung den
    Rahmen der Arbeit sprengen würde.

    Die zweite Komponente ist eine Software-seitige Lösung zur Programmierung der Hardware-Applikation. Sie besteht aus
    mehreren Subkomponenten. Abbildung \ref{fig:softwarecomp} zeigt das Gesamtbild des Architekturkonzepts.
    Das primäre Ziel der Software-Komponenten ist es, eine einheitliche und einfach zu erweiternde Sammlung von
    Schnittstellen für Peripherie auf dem \glsit{glo:pcb} zu schaffen.

    Darauf aufbauend soll ein \glsit{glo:ui} (in weiteren Ausführungen \textit{Designer\hyp{}UI} genannt) geschaffen werden,
    mit der die Peripherie graphisch konfiguriert werden kann.
    Anschließend soll es möglich sein, die entwickelte Firmware (ob nun mit Designer-\glsit{glo:ui} oder selber
    programmiert) auf einem Host-System unabhängig von Plattform und Entwicklungsumgebung auf den Mikrocontroller zu
    spielen und auf Fehler zu untersuchen.

    \begin{figure}[h]
        \centering
        \caption{Konzipiertes Gesamtbild der Systemarchitektur}
        \source{eigene Zeichnung}
        \label{fig:softwarecomp}
        \includegraphics[scale=1.3]{../Graphiken/arch}
    \end{figure}

    \subsection{Anforderungen}
        \label{sec:demands}
        Die Anforderungen an eine Lösung zur Vereinfachung der Entwicklung bei Embedded\hyp{}Systemen ergeben sich aus den
        in Abschnitt \ref{sec:probleme} aufgeführten Probleme im Entwicklungsprozess.
        Die zu erfüllenden Anforderungen würden somit lauten:
        \begin{itemize}
            \item Ein Entwickler sollte nicht mehr an die vom Hersteller zur Verfügung gestellten Treiber und
            Entwicklungswerkzeuge gebunden sein.
            \item Ein unerfahrener Hardware-Entwickler sollte ohne großen Aufwand in der Lage sein, mit Hilfe der Software
            eine Applikation zu entwickeln und die auf dem \glsit{glo:pcb} vorhandene Peripherie miteinbeziehen.
            Dies heißt nicht, dass sich ein Entwickler nicht mit der Thematik auseinander setzen muss. Lediglich der
            durch Hardware-Hersteller erzeugte Überfluss an Schnittstellen, Konfigurationsmöglichkeiten und Treiberimplementierungen
            soll durch eine klar definierte, einheitliche, modulare und überschaubare Software mit integriertem
            \glsit{glo:api} ersetzt werden.
            \item Durch eine Vereinheitlichung soll die steile Lernkurve bei Embedded-Systemen
            abgeflacht werden und der Fokus des Entwicklers auf die eigentliche Entwicklung der Applikation gelenkt werden.
            \item Der Auswahlprozess soll durch die Bereitstellung eines einheitlichen Werkzeuges vereinfacht werden.
            Eine frei verfügbare Plattform, welche den Entwickler so weit wie möglich uneingeschränkt lässt, soll dem
            Entwickler wie im vorherigen Punkt umständliche Konfigurations- und Initialisierungsarbeit abnehmen.
            \item Alle vorherigen Punkte sollen ohne das Einbußen der von der Hardware erbrachten Leistung realisiert
            werden, um die Entwicklung von komplexen Applikationen für Embedded-Systeme zu ermöglichen.
        \end{itemize}

    \subsection{Herstellerunabhängigkeit}
    Abbildung \ref{fig:softwarecomp} zeigt das in Abschnitt \ref{sec:layers} und \ref{sec:abstractionoverview}
    aufgeführte Schichtenmodell der Entwicklung von Embedded-Systemen mit einer zusätzlichen
    \textit{Vendor-Independent-\glsit{glo:api}}-Schicht. Diese soll eine Schnittstelle zu den herstellerspezifischen
    \glsplit{glo:hal} darstellen und einen einheitlichen Zugriff auf \glsit{glo:rtos}-Betriebssysteme ermöglichen
    (hierfür kann auch \glsit{glo:cmsis}-\glsit{glo:rtos} in betracht gezogen werden).

    \glsit{glo:cmsis}-Driver befindet sich in der Grafik auf der selben Ebene wie die herstellerspezifischen \glsplit{glo:hal}. Grund dafür ist,
    dass nicht alle Chip-Hersteller das \glsit{glo:cmsis}-Driver-Interface unterstützen. Unterstützt ein Hersteller das
    \glsit{glo:cmsis}-Driver-Interface, so ist das Interface im Software-Framework, zusammen mit den herstellerspezifischen
    Treibern und \glsplit{glo:hal}, mitintegriert.

    Eine Software, welche von der \glsit{glo:hal} und den entsprechenden Treibern Gebrauch macht,
    muss also in mehreren Szenarien, in denen verschiedene Abstraktionsschichten verfügbar sind und gebraucht werden,
    funktionieren.


    \begin{figure}
        \centering
        \caption{Verwendete CMSIS-Abstraktionsschichten in einer Beispiel\hyp{}Applikation}
        \source{\href{http://www.keil.com/appnotes/files/apnt_268.pdf}{Skript zum Vortrag von Reinhard Keil}, Zugriff: 03.08.17}
        \label{fig:armcmsisexampleapp}
        \includegraphics[scale=0.5]{../Graphiken/STM_DEV_LAYERS_ARM}
    \end{figure}

    Abbildung \ref{fig:armcmsisexampleapp} zeigt ein von \glsit{glo:arm} vorgestelltes Beispiel für eine Applikation, in
    dem die Verwendung der \glsit{glo:cmsis}-Abstraktionsschichten verdeutlicht wird.
    Um vollständige Herstellerunabhängigkeit zu erlangen, muss alles unter dem \textit{Board-Support-Layer} abstrahiert werden.

    Um Implementierungsarbeit zu sparen, sollte, wenn möglich, das \glsit{glo:cmsis}-Driver-Interface genutzt werden. Da wie eben
    schon erwähnt nicht alle Hersteller \glsit{glo:cmsis}-Driver unterstützen und das Interface an sich nicht umfangreich genug für
    manche Applikationen ist, muss die Abstraktion auch direkt mit dem Hersteller-HAL funktionieren.

    Die Software soll noch einen Schritt weiter gehen und dem Entwickler die in Abschnitt \ref{sec:abstractionoverview}
    angesprochene Designer-\glsit{glo:ui} zur Verfügung stellen.

    \subsection{Ansatz der modellbasierten Entwicklung}
    \label{sec:modelbaseddev}
    Das Designer-\glsit{glo:ui} soll es Entwicklern ermöglichen ein funktionales Embedded-System zu entwerfen.
    Es muss ein Überblick über die Hardware geschaffen werden, wobei sämtliche Peripherie konfigurierbar und
    intrinsische so wie externe Kommunikation steuerbar sein muss. Es soll also möglich sein, die Hardware
    über graphische Werkzeuge zu programmieren.

    Folgender Abschnitt gibt einen kurzen Überblick über den Ansatz der modellbasierten Entwicklung und leitet daraus
    Anforderungen an das Designer-\glsit{glo:ui} ab.

    Modellbasierte Entwicklung bezeichnet einen Prozess der Software-Entwicklung. Dieser besteht aus der Erzeugung einer
    lauffähigen Software aus modellierten Komponenten.
    ``Dabei wird die Software nicht mehr textuell mit Programmiersprachen, wie beispielsweise C, programmiert, sondern
    mit graphischen Modellen auf einer höheren Abstraktionsebene spezifiziert und schrittweise über verschiedene
    Entwicklungsphasen von den Anforderungen über den Entwurf bis zur Implementierung verfeinert.''\citep[vgl.~S.~165]{Berns2010}

    In der Praxis setzt sich die modellbasierte Entwicklung aus \acrshortit{glo:uml}-Diagrammen in
    Kombination mit der Verwendung von \acrshortit{glo:case}-Werkzeugen zusammen.
    \acrshortit{glo:case}-Werkzeuge generieren ein Code-Gerüst aus \acrshortit{glo:uml}-Diagrammen, welches vom Entwickler für die weiterführende
    Entwicklung genutzt werden kann.

    Das Ziel des Designer-\glsit{glo:ui} ist jedoch nicht die reine Modellierung durch \acrshortit{glo:uml}-Diagramme, sondern viel mehr einen Weg
    zu finden, bestehende Funktionalitäten und Peripherie miteinander zu vernetzen. Dies soll möglichst übersichtlich
    und intuitiv für Entwickler geschehen.

    Eine abstrakte Definition von Prozess- und Produktmodellierung beschreibt viel mehr ein Paradigma der
    modellbasierten Entwicklung:
    \begin{description}
        \item[Prozess] Ein Prozess erlaubt die Beschreibung von \textit{Aktivitäten}. Aktivitäten sind
        \textit{wiederholbar}, \textit{umkehrbar} und \textit{nachverfolgbar}. Sie schließen unter anderem Aufgaben
        auf unterster Ebene wie zum Beispiel \textit{Refactoring} und \textit{Umbenennung}, so wie auch Aufgaben auf höheren
        Ebenen, wie zum Beispiel den Einsatz von abstrakten Controller-Funktionalitäten der Zielplattform, ein.
        Eine Aktivität ist prinzipiell eine Aufgabe, welche im Rahmen eines Entwicklungsprozesses geschieht, wie zum Beispiel
        die \textit{Definition eines System-Interfaces} oder die \textit{Generierung von Test-Szenarien}
        \citep[vgl.~Kap.~1~und~2]{Schaetz2002}.
        \item[Produkt] Ein Produkt modelliert Artefakte und deren Relationen zueinander. Das heißt für den Bereich der
        Embedded\hyp{}Systeme, dass alle für die Produktion relevanten \textit{Komponenten}, \textit{Zustände},
        \textit{Nachrichten} und deren Beziehungen zueinander in die Produktmodellierung einfließen
        \citep[vgl.~Kap.~1~und~2]{Schaetz2002}.
    \end{description}

    Betrachtet man die Programmierung, in Bezug auf formale Sprachen, für ein Embedded-System schichtenweise, kommt man
    auf drei grobe Bereiche:
    \begin{enumerate}
        \item Programmierung der Hardware mit Assembler-Code
        \item Erste Abstraktion durch Programmiersprachen wie zum Beispiel C
        \item modellbasierte Entwicklung auf graphischer Ebene
    \end{enumerate}

    Es stellt sich die Frage, welche Komponenten auf welche Weise modelliert werden sollen. Dabei muss die Software die
    in Abschnitt \ref{sec:demands} gestellten Anforderungen reflektieren.
    ``Allerdings deckt keines der heute verfügbaren Werkzeuge den gesamten Entwicklungslebenszyklus adäquat ab. Zur
    Modellierung einer Architektur ist es beispielsweise notwendig mehrere Sichten zu definieren, die jeweils
    unterschiedliche Aspekte des Systems repräsentieren''\citep[]{Trapp11}.

    Letztendlich sind folgende Aspekte bei einer graphischen Modellierung zu berücksichtigen:
    \begin{itemize}
        \item Hardware-Abstraktion - Definition der Hardware und deren Attribute so wie zur Verfügung stehende
        Funktionalitäten - \textbf{Schnittstellen/Treiber}\\
        \textbf{Beispiel:} LED kann an- und ausgeschaltet werden
        \item Verteilung von Aufgaben auf verschiedene Hardware-Komponenten\\
        \textbf{Beispiel:} Schalte LED an
        \item Kommunikation zwischen Hardware-Komponenten\\
        \textbf{Beispiel:} Sensor übermittelt Daten an ein Ethernet-Modul
    \end{itemize}

    \subsection{Konzept für eine Integration von OpenOCD in eine IDE}
    Der Schwerpunkt dieser Arbeit liegt auf der für die Hardware-Entwicklung genutzten Software. Im Vordergrund stehen
    dabei das Aufspielen von Firmware auf die Zielhardware und die Fehlersuche bei bereits aufgespielter Firmware über
    ein Host\hyp{}System.

    Der Entwickler muss sich also seine eigene \glsit{glo:toolchain} zusammenstellen, was wie in Abschnitt
    \ref{sec:demands} erläutert durch die Vielzahl an herstellerspezifischen Tools und Komplexität der einzelnen
    Werkzeuge oft ein Problem für Neueinsteiger darstellt.

    \glsplit{glo:ide} fassen die benötigten Werkzeuge zusammen und ermöglichen den Zugriff auf
    diese in einer gemeinsamen Umgebung (ohne sogenannte Medienbrüche).

    Bisher vorhandene Lösungen, wie zum Beispiel \textit{Atmel Studio} oder \textit{Keil IDE} binden den Entwickler an
    eine Plattform und lassen wenig Erweiterbarkeit zu oder sind mit hohen Kosten verbunden. Hinzu kommt, dass sie oft
    herstellerspezifisch (Atmel, STM, etc.) sind.

    Es wird also ein Werkzeug benötigt, welches es einem Hardware-Entwickler ermöglicht innerhalb einer integrierten
    Entwicklungsumgebung eine Applikation für Embedded-Systeme zu entwickeln, ohne sich Gedanken über die benötigten
    Werkzeuge für die Hardware machen zu müssen.

    Das Plugin soll für die \glsplit{glo:ide} der IntelliJ-Plattform
    \footnote{\href{https://www.jetbrains.com/}{Jetbrains-Seite}} entwickelt werden. Die Entscheidungsgründe für die
    Auswahl lauten:
    \begin{itemize}
        \item Plattformunabhängig
        \item Frei verfügbar als \textit{Community}-Version
        \item Verschiedene \glsplit{glo:ide} für verschiedene Sprachen verfügbar. Das Plugin funktioniert
        (außer sprachspezifische Funktionalitäten) für jede \glsit{glo:ide} der IntelliJ-Plattform
        \item Unterstützt Remote-Debugging (\glsit{glo:rsp})
    \end{itemize}

    Da C die am weitesten verbreitete Sprache in der Entwicklung von Embedded-Systemen ist, liegt der Fokus des Plugins
    jedoch auf der Entwicklungsumgebung CLion.
    In zukünftiger Entwicklung des Plugins wird ein Werkzeug konzipiert, mit dem es möglich ist, entwickelte Firmware auf
    den Flash-Speicher eines Prozessors herunterzuladen und auf Fehler zu untersuchen. Da dies mit einem größeren
    Zeitaufwand verbunden ist, wird zunächst das bereits angesprochene \glsit{glo:openocd}, welches für Windows, Linux
    und OSX verfügbar ist, genutzt um ein \glsit{glo:proofofconcept} zu erstellen.
        \subsubsection{Anforderungen}
        \label{sec:pluginrequirements}
        Das Plugin verfolgt das Ziel, dem Entwickler eine Entwicklungsplattform zu liefern mit der Firmware-Programme
        geschrieben, kompiliert, aufgespielt und eine Fehleranalyse durchgeführt werden kann. Es soll also der komplette
        Entwicklungszyklus für die Programmierung von Embedded-Systemen in einer geschlossenen Umgebung abgedeckt
        werden.

        Ein fortgeschrittener Editor zum Schreiben von Firmware wird dabei von der CLion-Entwicklungsumgebung
        bereitgestellt.

        Der Compiler kann in der Entwicklungsumgebung ausgewählt und konfiguriert werden. Hierzu ist ein Build-System
        namens \glsit{glo:cmake} in die \glsit{glo:ide} integriert, mit dem der zu verwendende Compiler bestimmt werden kann. Compiler können
        nach persönlichen Präferenzen und bedingt durch das Host-Betriebssystem ausgewählt werden. Hier ist zu beachten,
        dass der Compiler die richtige Zielarchitektur unterstützen muss.

        \glsit{glo:openocd} stellt eine Funktion zum Aufpielen von Firmware-Programmen auf einen Prozessor bereit. Hierzu müssen
        bestimmte Angaben, wie zum Beispiel die Startadresse im Flash-Speicher an die das Programm-Image geschrieben
        werden soll, gemacht werden. Diese müssen über ein \glsit{glo:ui} in der \glsit{glo:ide} konfigurierbar sein.

        Der letzte Punkt des Entwicklungszyklus, das Debugging, wird von einem lokalen \glsit{glo:gdb}-Server ermöglicht, welcher
        vom \glsit{glo:openocd} gestartet wird. Es muss dem Entwickler ermöglicht werden den lokalen Port für den \glsit{glo:gdb}-Server
        anzugeben. Da CLion Remote-Debugging unterstützt, muss anschließend eine \textit{Debugging-Configuration}
        mit der Adresse, unter dem der \glsit{glo:gdb}-Server erreichbar ist, erstellt werden.

        Schließlich sollte ein Nutzer außerdem über ein \glsit{glo:ui} in der Lage sein das Verhalten des Plugins zu
        konfigurieren. Hierzu gehört die Angabe des Ports, unter dem der \glsit{glo:gdb}-Server erreichbar ist, die für das
        \glsit{glo:openocd} relevante Angabe des Namens für die entsprechende Board-Konfiguration und die Startadresse für das
        Ablegen des Firmware-Programms.

        \subsubsection{Das IntelliJ-SDK}
        \label{sec:sdk}
        Um ein Plugin für eine \glsit{glo:ide} der IntelliJ-Familie zu entwickeln, muss das in Java geschriebene
        \glsit{glo:sdk} von IntelliJ genutzt werden.

        Aktionen, wie zum Beispiel das Öffnen einer Datei oder das Kompilieren eines Programms, werden durch die
        Klasse \texttt{AnAction} modelliert. Beim Auswählen eines Menüpunktes oder Toolbar-Buttons werden sie
        ausgeführt. Aktionen können durch das Ableiten der \texttt{AnAction}-Klasse erstellt
        werden und zu Gruppen hinzugefügt werden. Aus diesen Gruppen lassen sich eigene (Sub-)Menüs und Werkzeugleisten
        für die hinzugefügten Aktionen erstellen.

        Konfigurationen zum Ausführen und Debuggen eines Programms werden in IntelliJ mit \textit{Run Configurations}
        modelliert. Da CLion Remote-Debugging über \glsit{glo:gdb} unterstützt, enthält es dafür auch eine vorgefertigte
        \textit{Run Configuration}. Hier können Adresse und Port, unter denen der \glsit{glo:gdb}-Server erreichbar ist, angegeben
        werden.

        In Anbetracht dieser beiden Modellierungen ergeben sich folgende Ideen für eine Programmstruktur.

        Es lassen sich zunächst Aktionen definieren:
        \begin{itemize}
            \item Der \glsit{glo:openocd}-Daemon muss gestartet und gestoppt werden können
            \item Ein Programm muss für die korrekte Chip-Architektur kompiliert werden
            \item Die kompilierte Programmdatei muss auf den Chip-Speicher geschrieben werden
        \end{itemize}

        Für das Remote-Debugging kann eine \textit{Run Configuration} erstellt werden. Diese muss die korrekten
        Parameter für den \glsit{glo:gdb}-Server enthalten.

        \subsubsection{Architektur}

        Aus den in Abschnitt \ref{sec:pluginrequirements} und \ref{sec:sdk} erarbeiteten Anforderungen und Ansätze
        ergibt sich ein Konzept für die benötigten Komponenten und eine grobe Software-Architektur des Plugins.

        \begin{figure}[h]
            \centering
            \caption{Komponenten und Kommunikationswege}
            \label{fig:commplugin}
            \includegraphics[scale=0.5]{../Graphiken/plugin_arch_v5}
        \end{figure}

        Abbildung \ref{fig:commplugin} zeigt den allgemeinen Entwicklungs- und Kommunikationsprozess der beteiligten
        Instanzen.

        Für die Entwicklung von Embedded-Systemen muss die \glsit{glo:ide} zwei Arten von Frontends besitzen, also (in diesem
        Fall graphische) User-Interfaces, mit denen der Nutzer Funktionalitäten der dahinterliegenden Software benutzen
        kann.

        CLion unterstützt bestimmte Debug-Funktionalitäten, wie zum Beispiel das Setzen von Breakpoints oder
        schrittweise Ausführen von Anweisungen, die mit Hilfe des \glsit{glo:gdb}s realisiert werden. Zusätzlich sind
        hierfür graphische \glsplit{glo:ui} in die \glsit{glo:ide} integriert. Da Remote-Debugging mit dem \glsit{glo:gdb}
        \glsit{glo:rsp} von der Entwicklungsumgebung unterstützt wird, ist das benötigte Frontend für den \glsit{glo:gdb}-Server bereits
        vorhanden. Die \glsit{glo:ide} sendet also vom Nutzer initiierte Debug-Befehle, wandelt sie in das entsprechende
        \glsit{glo:rsp}-Paket um und sendet diese an einen \glsit{glo:gdb}-Server weiter.

        Das zweite Frontend stellt eine direkte Integration des \glsit{glo:openocd}-Werkzeuges dar. Der Nutzer bzw. das Plugin
        kann Funktionalitäten der lokalen Installation von \glsit{glo:openocd} nutzen ohne die \glsit{glo:ide} zu verlassen.

        Das \glsit{glo:openocd}-Programm besteht aus einem \glsit{glo:cli}, welches die Befehle weiter an den \glsit{glo:openocd}-Daemon
        sendet. Beim Start des \glsit{glo:openocd}-Daemons, wird gleichzeitig der lokale \glsit{glo:gdb}-Server gestartet.

        Der \glsit{glo:gdb}-Server empfängt vom \glsit{glo:gdb}\hyp{}Frontend der \glsit{glo:ide} gesendete Debugging\hyp{}Befehle und gibt diese weiter an den
        \glsit{glo:openocd}\hyp{}Daemon. Dieser wandelt die \glsit{glo:gdb}-Befehle (\glsit{glo:rsp}) in die entsprechenden \acrshortit{glo:jtag}-Befehle um und überträgt
        diese über \glsit{glo:usb} mit Hilfe des \glsit{glo:cmsisdap}-Protokolls an die Debug-Einheit der Target\hyp{}Hardware.

        Das \glsit{glo:openocd}-Frontend muss also drei grundlegende Aufgaben erfüllen
        \begin{enumerate}
            \item Der \glsit{glo:openocd}-Daemon muss gestartet werden
            \item Der \glsit{glo:openocd}-Daemon muss mit den gewünschten Einstellungen konfiguriert werden
            \item Das Programm-Image muss über \glsit{glo:openocd} und somit \glsit{glo:cmsisdap} auf das Board heruntergeladen werden
        \end{enumerate}

        Um den kompletten Entwicklungszyklus abzudecken fehlt die Kompilierung des Programms. Das von CLion
        unterstützte Build-System \glsit{glo:cmake} muss so konfiguriert werden, dass es
        \begin{enumerate}
            \item den richtigen Compiler zum Kompilieren für \glsit{glo:arm} Architekturen nutzt
            \item das Image an eine vorgesehene Stelle schreibt, auf die das \glsit{glo:openocd}-Frontend Zugriff hat
        \end{enumerate}

        Schließlich sind Besonderheiten des Host-Systems zu beachten. Es muss sichergestellt werden, dass die
        benötigte Software installiert ist (Cross-Compiler für \glsit{glo:arm}-Architekturen und \glsit{glo:openocd}).
        Da das Plugin so ausgelegt ist, dass es native Aufrufe des \glsit{glo:openocd} Programms ausführt, muss für
        Windows-basierte Betriebssysteme eine andere Logik entwickelt werden als für Unix-basierte Betriebssysteme.

        Es ergibt sich zunächst eine konzeptionelle Einteilung in folgende Software-Pakete für das Plugin:
        \begin{description}
            \item[\texttt{actions}] Das \texttt{actions}-Paket enthält alle Aktionen des \glsit{glo:openocd}-Frontends.
            Konzipierte Aktionen sind das Starten des \glsit{glo:openocd}-Daemons mit den richtigen Einstellungen und das
            Aufspielen des Programm-Images.
            \item[\texttt{extensions}] Als \texttt{extensions} werden alle Programmteile gesehen, die eine aktive
            Erweiterung der bereits vorhandenen \glsit{glo:ide}-Elemente sind. Ein Beispiel hierfür ist eine Initialisierungslogik,
            die beim Öffnen eines Projektes ausgeführt wird. Die Formulierung ist hierbei wie im vorherigen Punkt von
            dem IntelliJ-SDK übernommen.
            \item[\texttt{ui}] Alle graphischen Komponenten befinden sich im \texttt{ui}-Paket. Hierzu gehört im
            wesentlichen eine Toolbar, welche alle Einstellungsmöglichkeiten für das \glsit{glo:openocd} enthalten soll.
            \item[\texttt{util}] Das \texttt{util}-Paket enthält alle sonstigen Hilfsklassen, wie zum Beispiel eine
            Klasse mit Methoden zum Ausführen nativer Aufrufe.
        \end{description}

        Als Letztes stellt sich noch die Frage, welcher Compiler zum Kompilieren des Firmware-Programms genutzt
        wird. Verwendet werden soll der \glsit{glo:gcc}, da dieser die \glsit{glo:arm}-Architektur als Zielsystem unterstützt und
        Installationsdateien für OSX, Linux und Windows enthält.

        Das benötigte Paket trägt den Namen \texttt{arm-none-eabi-gcc} und enthält ein \glsit{glo:gcc}-Compiler für die \glsit{glo:arm}-Architektur.

    \subsection{Entwicklung}
        Die Entwicklung lässt sich in die drei Bereiche \textit{Vorbereitung und Einrichtung der Toolchain},
        \textit{Integration von \glsit{glo:openocd} in die \glsit{glo:ide}} und \textit{Erstellen eines User-Interfaces} unterteilen.
        \subsubsection{Compiler und CMake}
        Das Einrichten eines Cross-Compilers mit \glsit{glo:cmake} in CLion ist eine nicht-triviale Aufgabe. Im Folgenden sind
        die Schritte aufgelistet, die notwendig sind um ein C-Programm über CLion für die \glsit{glo:arm}-Architektur zu
        kompilieren.
        \begin{enumerate}
            \item Die voreingestellte \glsit{glo:toolchain} muss durch eine neue auf \glsit{glo:arm}-Architekturen ausgelegte \glsit{glo:toolchain}
            ersetzt werden. Hierfür wird eine neue \glsit{glo:cmake}-Datei mit einem beliebigen Namen (in dem Testdurchlauf
            \texttt{cc\_test.cmake} genannt) erstellt, dessen Inhalt in Listing \ref{cccmakeconf} angegeben ist.
            \begin{lstlisting}[
                language=CMake,
                caption=Verkürzte Version der CMake-Konfigurationsdatei für die Cross-Compiler-Toolchain,
                label=cccmakeconf
            ]
include(CMakeForceCompiler)
set(CMAKE_SYSTEM_NAME Generic)
set(CMAKE_SYSTEM_PROCESSOR cortex-m7)

find_program(ARM_CC arm-none-eabi-gcc
        ${TOOLCHAIN_DIR}/bin
        )
find_program(ARM_CXX arm-none-eabi-g++
        ${TOOLCHAIN_DIR}/bin
        )
find_program(ARM_OBJCOPY arm-none-eabi-objcopy
        ${TOOLCHAIN_DIR}/bin
        )
find_program(ARM_SIZE_TOOL arm-none-eabi-size
        ${TOOLCHAIN_DIR}/bin)

# specify the cross compiler
CMAKE_FORCE_C_COMPILER(${ARM_CC} GNU)
CMAKE_FORCE_CXX_COMPILER(${ARM_CXX} GNU)
            \end{lstlisting}
            Die erste Anweisung schließt das CMakeForceCompiler-Modul von \glsit{glo:cmake} ein, welches bestimmte Makros für die
            Benutzung von Cross-Compiling-\glsplit{glo:toolchain} definiert\citep[vgl.~module/CMakeForceCompiler]{CMAKEDOC}.

            Mit \texttt{find\_program()} wird nach den für die \glsit{glo:toolchain} relevanten Programmen gesucht und
            in die Variablen geschrieben, somit ist eine Überprüfung mitintegriert. Die Compiler können mit
            \texttt{CMAKE\_FORCE\_C\_COMPILER} und \texttt{CMAKE\_FORCE\_CXX\_COMPILER} gesetzt werden. Die Variable
            \texttt{\$\{TOOLCHAIN\_DIR\}} zeigt auf den Pfad zum Ordner, welcher die benötigten Compiler beinhaltet.
            \item Als nächstes müssen die \glsit{glo:toolchain}-Einstellungen von \glsit{glo:cmake} übernommen werden. Dafür müssen unter
            \texttt{Settings} $\rightarrow$ \texttt{Build, Execution, Deployment} $\rightarrow$ \texttt{CMake}
             $\rightarrow$ \texttt{CMake options} die beiden \glsit{glo:cmake}-Optionen
            \begin{Verbatim}[frame=single]
-DTOOLCHAIN_DIR:PATH=/usr/bin/
-DCMAKE_TOOLCHAIN_FILE=cc_test.cmake
            \end{Verbatim}
            gesetzt werden.
            \item Schließlich muss der \glsit{glo:cmake}-Cache in der \glsit{glo:ide} gelöscht und die \glsit{glo:cmake}-Dateien neu geladen werden.
        \end{enumerate}

        Um diesen Prozess im Plugin weitesgehend zu automatisieren, muss die \glsit{glo:cmake}-Datei aus Listing \ref{cccmakeconf}
        automatisch in das oberste Verzeichnis des Projekts geschrieben werden. Der Pfad zur \glsit{glo:toolchain} ist
        dabei in Form einer Benutzereinstellung zu setzen.

        \subsubsection{OpenOCD-Frontend}
        Bei der Implementierung des \glsit{glo:openocd}-Frontends sind einige Sachen zu beachten.

        Auf Unix-ähnlichen
        \footnote{Mit ``Unix-ähnlichen Systemen'' sind hauptsächlich die Betriebssysteme Linux und OS X gemeint}
        Systemen benötigt das \glsit{glo:openocd} \glsit{glo:root}-Rechte, um ein Programm-Image auf den Flash-Speicher der Zielhardware
        herunterzuladen. Dazu wird der \texttt{sudo} Befehl genutzt, welcher bestimmten Benutzern die Möglichkeit gibt
        einen Prozess, der \glsit{glo:root}-Rechte benötigt, auszuführen ohne das \glsit{glo:root}-Passwort kennen zu müssen.
        Vorraussetzung ist, dass \texttt{sudo} für den aktuellen Benutzer konfiguriert ist. Ist dies nicht der Fall,
        muss der Benutzer trotzdem ein Passwort eingeben.

        Das Plugin ruft einen nativen \glsit{glo:openocd}-Prozess auf, indem die Java-\glsplit{glo:api}\\\texttt{Process} und
        \texttt{ProcessBuilder} verwendet werden.

        Die benötigten \glsit{glo:openocd}-Kommandos lauten
        \begin{Verbatim}[frame=single]
$ openocd -f interface/cmsis-dap.cfg -f board/<BOARD CFG>
        \end{Verbatim}
        und
        \begin{Verbatim}[frame=single]
$ openocd -f board/<BOARD CFG>
    -c "program <IMAGE> exit <EXIT ADDRESS>"
        \end{Verbatim}

        Der erste Befehl dient dazu, dass das Programm \glsit{glo:openocd} mit dem \glsit{glo:cmsisdap}-Protokoll
        konfiguriert und der \glsit{glo:openocd}-Daemon gestartet wird. Bei jedem \glsit{glo:openocd}-Befehl, bei dem
        der Daemon gestartet wird, muss die Board-Konfiguration angegeben werden.

        Der zweite Befehl dient dazu, den \glsit{glo:openocd}-Daemon zu starten und dabei ein Programm-Image auf den Flash-Speicher
        des Chips herunterzuladen.

        Beide Befehle werden in Java-Code eingebettet und mit den entsprechenden Parametern versehen, die vom Nutzer
        in der \glsit{glo:ide} einstellbar sind.

        Außerdem können beim Starten des Daemons noch die gewünschten Port\hyp{}Nummern gesetzt werden. Diese sind
        standardmäßig so eingestellt, dass der \glsit{glo:gdb}-Server auf Port \texttt{3333}, der \glsit{glo:tcl}-Server auf Port
        \texttt{6666} und der \glsit{glo:telnet}-Server auf Port \texttt{4444} hört. Da weder der \glsit{glo:tcl}- noch der \glsit{glo:telnet}-Server
        in diesem Kontext benötigt wird, wird nicht weiter darauf eingegangen.

        \subsubsection{User-Interface}
        Das User-Interface des Plugins findet sich im Hauptmenü der \glsit{glo:ide} unter dem Menüpunkt
        \texttt{\underline{E}mbedded Systems} und in der Toolbar am rechten Bildschirmrand wieder.
        In der Toolbar können bestimmte Konfigurationen, wie zum Beispiel der Pfad zum Programm-Image, vorgenommen werden.

        Die graphischen Komponenten wurden nach Vorgaben des IntelliJ-\glsit{glo:sdk}s mit der Java-\texttt{Swing}-Bibliothek
        erstellt.

        Beim Registrieren der Aktionen (Starte Daemon, Lade Programm-Image und Generiere \glsit{glo:cmake}-Konfigurationsdatei) in der
        \texttt{Plugin.xml}-Datei, lassen sich Gruppen und sogenannte \textit{anchor-points} angeben. Diese bestimmen,
        an welche Stelle im \glsit{glo:ui} die Aktionen als Menüeinträge erscheinen. In diesem Fall wurde eine neue Gruppe
        erstellt und der Gruppe \texttt{MainMenu} als letztes angehängt.

        Bei jeder Auswahl des Aktionspunktes im Menü wird die in der Aktionsklasse überschriebene Methode
        \texttt{actionPerformed(ActionEvent e) \{...\}} ausgeführt.

        \subsubsection{Aktueller Stand und Ausblick}
        Folgende Funktionalitäten wurden im Plugin umgesetzt:
        \begin{itemize}
            \item Es wird für den Nutzer eine \glsit{glo:cmake}-Konfigurationsdatei generiert, in der die Verwendung der benötigten
            \glsit{glo:toolchain} für das Kompilieren eines Firmware-Programms auf \glsit{glo:arm}-Architekturen eingestellt wird.
            Dabei wird auf die Verfügbarkeit der \glsit{glo:toolchain}-Programme geprüft und CLion versucht bei jedem Laden der
            neuen \glsit{glo:cmake}-Konfiguration ein Testprogramm zu kompilieren und auszuführen um die Compiler auf ihre
            Funktionstüchtigkeit zu prüfen.
            \item Ein kompiliertes Programm kann durch den gestarteten \glsit{glo:openocd}-Daemon per \glsit{glo:usb} auf den Flash-Speicher der
            Zielhardware heruntergeladen werden. Dabei werden die von \glsit{glo:openocd} bereitgestellten Flash-Algorithmen
            genutzt.
            \item Der \glsit{glo:openocd}-Daemon kann, ohne dass die Firmware auf das Board heruntergeladen wird, gestartet werden.
            Dabei kann der Port für den lokalen \glsit{glo:gdb}-Server gesetzt werden. Über die \glsit{glo:ide} kann eine neue
            \textit{Remote-Debugging-Configuration} angelegt werden und anschließend das auf der Zielhardware laufende Programm
            auf Fehler untersucht werden.
        \end{itemize}

        Da es sich bei dem Plugin lediglich um ein \glsit{glo:proofofconcept} handelt, fehlen einige Aspekte und
        Komponenten, die zu der Robustheit und \textit{User-Experience} beitragen.
        \begin{itemize}
            \item Es muss ein Überblick geschaffen werden, welche Hardware von \glsit{glo:openocd} unterstützt wird. Eine Liste
            aller verfügbaren Konfigurationsdateien sollte für den Entwickler ohne umfangreiche Suche einsehbar sein.
            \item Das \glsit{glo:ui} bedarf an vielen Stellen einer Überarbeitung. Beispielsweise sollte dem Nutzer idealerweise
            ein Status des \glsit{glo:openocd}-Daemons, sowie Informationen zum \glsit{glo:gdb}-Server und der Zielhardware angezeigt werden.
            \item Das Konfigurieren der \glsit{glo:toolchain} soll einfacher und mit noch wenigeren Umständen geschehen.
            Zum Beispiel soll der Entwickler nicht mehr den Pfad der \glsit{glo:cmake}-Konfigurationsdatei in CLion
            als CMake-Option eintragen müssen.
            \item Fehlende benötigte Software-Komponenten, wie zum Beispiel das \glsit{glo:openocd} oder die
            \glsit{glo:arm}-Compiler-\glsit{glo:toolchain}, sollen vom Plugin erfasst und automatisch heruntergeladen
            werden.
            \item Überwachung (Monitoring) der Zielhardware und die Möglichkeit verschiedene Laufzeitparameter zu
            ändern, sollten dem Entwickler zur Verfügung stehen.
            \item Die Konfigurationen für das \glsit{glo:openocd}-Plugin müssen persistent gemacht werden, indem sie in eine
            Konfigurationsdatei geschrieben und bei jedem Start wieder ausgelesen werden.
            \item Beim Testen des Plugins wurden trotz \glsit{glo:jvm} unterschiedliche Verhaltensweisen beim Ausführen nativer
            Prozesse festgestellt. Beispiele hierfür sind die verschiedenen Rechte-Verwaltungen oder die Angabe von
            Dateipfaden (Leerzeichen, Anführungszeichen, etc.).
        \end{itemize}

        Letztendlich stellt das Plugin einen ersten Ansatz dar, um eine frei verfügbare Alternative für die Integrierung
        von Entwicklungswerkzeugen für Embedded-Systeme in eine Entwicklungsumgebung zu schaffen.

        Das \glsit{glo:openocd} steht dabei für ein Entwicklungswerkzeug, welches durch eine eigens implementierte
        Lösung ersetzt werden soll (siehe Abschnitt \ref{sec:prospect}).